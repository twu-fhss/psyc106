% Options for packages loaded elsewhere
\PassOptionsToPackage{unicode}{hyperref}
\PassOptionsToPackage{hyphens}{url}
%
\documentclass[
]{book}
\usepackage{amsmath,amssymb}
\usepackage{iftex}
\ifPDFTeX
  \usepackage[T1]{fontenc}
  \usepackage[utf8]{inputenc}
  \usepackage{textcomp} % provide euro and other symbols
\else % if luatex or xetex
  \usepackage{unicode-math} % this also loads fontspec
  \defaultfontfeatures{Scale=MatchLowercase}
  \defaultfontfeatures[\rmfamily]{Ligatures=TeX,Scale=1}
\fi
\usepackage{lmodern}
\ifPDFTeX\else
  % xetex/luatex font selection
\fi
% Use upquote if available, for straight quotes in verbatim environments
\IfFileExists{upquote.sty}{\usepackage{upquote}}{}
\IfFileExists{microtype.sty}{% use microtype if available
  \usepackage[]{microtype}
  \UseMicrotypeSet[protrusion]{basicmath} % disable protrusion for tt fonts
}{}
\makeatletter
\@ifundefined{KOMAClassName}{% if non-KOMA class
  \IfFileExists{parskip.sty}{%
    \usepackage{parskip}
  }{% else
    \setlength{\parindent}{0pt}
    \setlength{\parskip}{6pt plus 2pt minus 1pt}}
}{% if KOMA class
  \KOMAoptions{parskip=half}}
\makeatother
\usepackage{xcolor}
\usepackage{longtable,booktabs,array}
\usepackage{calc} % for calculating minipage widths
% Correct order of tables after \paragraph or \subparagraph
\usepackage{etoolbox}
\makeatletter
\patchcmd\longtable{\par}{\if@noskipsec\mbox{}\fi\par}{}{}
\makeatother
% Allow footnotes in longtable head/foot
\IfFileExists{footnotehyper.sty}{\usepackage{footnotehyper}}{\usepackage{footnote}}
\makesavenoteenv{longtable}
\usepackage{graphicx}
\makeatletter
\def\maxwidth{\ifdim\Gin@nat@width>\linewidth\linewidth\else\Gin@nat@width\fi}
\def\maxheight{\ifdim\Gin@nat@height>\textheight\textheight\else\Gin@nat@height\fi}
\makeatother
% Scale images if necessary, so that they will not overflow the page
% margins by default, and it is still possible to overwrite the defaults
% using explicit options in \includegraphics[width, height, ...]{}
\setkeys{Gin}{width=\maxwidth,height=\maxheight,keepaspectratio}
% Set default figure placement to htbp
\makeatletter
\def\fps@figure{htbp}
\makeatother
\setlength{\emergencystretch}{3em} % prevent overfull lines
\providecommand{\tightlist}{%
  \setlength{\itemsep}{0pt}\setlength{\parskip}{0pt}}
\setcounter{secnumdepth}{5}
\usepackage{booktabs}
\usepackage{amsthm}
\makeatletter
\def\thm@space@setup{%
  \thm@preskip=8pt plus 2pt minus 4pt
  \thm@postskip=\thm@preskip
}
\makeatother

\usepackage{tcolorbox}
\tcbuselibrary{breakable}

\newtcolorbox{blackbox}{
  colback=black,
  coltext=white,
  colframe=black,
  boxsep=5pt,
  arc=4pt,
  breakable
  }
\newtcolorbox{bonus}{
  colback=blue!15,
  colframe=blue!15,
  coltext=black!80,
  boxsep=5pt,
  arc=4pt,
  breakable
  }
\newtcolorbox{reflect}{
  colback=green!5,
  colframe=green!5,
  coltext=black!80,
  boxsep=5pt,
  arc=4pt,
  breakable
  }
\newtcolorbox{assessment}{
  colback=blue!5,
  colframe=blue!5,
  coltext=black!80,
  boxsep=5pt,
  arc=4pt,
  breakable
  }
  
\newtcolorbox{progress}{
  colback=purple!10,
  colframe=purple!10,
  coltext=black!80,
  boxsep=5pt,
  arc=4pt,
  breakable
  }
\newtcolorbox{video}{
  colback=yellow!5,
  colframe=yellow!5,
  coltext=black!80,
  boxsep=5pt,
  arc=4pt,
  breakable
  }
\newtcolorbox{caution}{
  colback=red!5,
  colframe=red!5,
  coltext=black!80,
  boxsep=5pt,
  arc=4pt,
  breakable
  }
\newtcolorbox{feedback}{
  colback=black!5,
  colframe=black!5,
  coltext=black!80,
  boxsep=5pt,
  arc=4pt,
  breakable
  }
\ifLuaTeX
  \usepackage{selnolig}  % disable illegal ligatures
\fi
\usepackage[]{natbib}
\bibliographystyle{apalike}
\IfFileExists{bookmark.sty}{\usepackage{bookmark}}{\usepackage{hyperref}}
\IfFileExists{xurl.sty}{\usepackage{xurl}}{} % add URL line breaks if available
\urlstyle{same}
\hypersetup{
  pdftitle={Introduction to Psychology},
  pdfauthor={Todd Dutka},
  hidelinks,
  pdfcreator={LaTeX via pandoc}}

\title{Introduction to Psychology}
\author{Todd Dutka}
\date{}

\begin{document}
\maketitle

{
\setcounter{tocdepth}{1}
\tableofcontents
}
\hypertarget{welcome}{%
\chapter*{Welcome}\label{welcome}}
\addcontentsline{toc}{chapter}{Welcome}

This is the course book for {[}insert{]}. This book is divided into 6 units of study to help you engage with the materials. The course resources and learning activities are designed not only to help prepare you for the course assessments, but also to give you opportinities to practice various skills.

Below you will find information about how to navigate this book. Please also refer the schedule in Moodle, as well as the Asseessment section in Moodle for instructions on required readings and assignments.

\hypertarget{course-notes}{%
\section*{Course Notes}\label{course-notes}}
\addcontentsline{toc}{section}{Course Notes}

You should be reading this information in the context of a Trinity Western University course offered via Moodle. If this is not the case, then this may be an unauthorized reproduction of the course. Please contact \href{mailto:elearning@twu.ca}{\nolinkurl{elearning@twu.ca}} if you have concerns.

These notes will be your guide through the learning activities and assessment strategies necessary for you to succeed in the course, so it is important for you to engage to the best of your ability and take advantage of the resources available to you through Trinity Western University.

Assessment tasks are managed in other sections of the Moodle course, so be sure to familiarize yourself with those requirements and resources.

\hypertarget{how-this-course-is-built}{%
\section*{How this Course is Built}\label{how-this-course-is-built}}
\addcontentsline{toc}{section}{How this Course is Built}

This course is primarily designed to be completed asynchronously, meaning that there are no scheduled times or places that you are required to meet, even online. You can work according to your own schedule \emph{within the six weeks you have to complete the course}. That said, this is a full university level course and there are timelines that we strongly recommend that you meet to ensure that you are succeeding in building your knowledge through the course.

It would be to your significant disadvantage to submit everything at the end of the course.

Asynchronous courses require learners to be well-organized and self-motivated, and we have included supports for you to help you develop strong learning habits that will ensure your success.

For example, there are several self-check quizzes throughout the course. These quizzes are not graded, but they can be powerful tools for you to ensure you understand key ideas and concepts. We suggest you take each quiz without the aid of your notes and textbook and multiple times until you have mastered the content. This strategy taps into three powerful learning structures that have been shown to be highly effective.

\begin{enumerate}
\def\labelenumi{\arabic{enumi}.}
\tightlist
\item
  \textbf{Effortful recall.} By intentionally trying to recall information without external aids, you are strengthening the neural pathways in your brain that lead to building new connections between ideas. One way to make recall easier is to connect key ideas to other things that you know or have experienced. For example, you might be studying World War II, and you connect the date that Canadians participated in the D-Day operation with something else meaningful to you that happened on June 6, like maybe the date you bought your first car.
\item
  \textbf{Spaced repetition.} By spreading out your attempts on the quiz (leaving a few days between attempts) you can maximize the effects of the first strategy (effortful recall) and ensure that your second or third attempts truly reflect what you know about the topic. We suggest leaving 1-3 days between attempt 1 and 2, then 4-5 days between attempt 2 and 3. You can use a tool like Trello, Notion, or Asana (free versions), or even a task list on your phone to set up a spaced repetition schedule.
\item
  \textbf{Interleaving.} This is the practice of studying a particular topic for a relatively short period of time (maybe 30-40 mins), then switching to a different topic for the same period, before going back to the original topic. We will help build this into your learning by including items from unit 1 in your unit 2-6 quizzes. You can also practice this by taking regular breaks in your work, or even by retaking a unit 1 quiz while you are working in unit 2.
\end{enumerate}

These three strategies are very effective at helping people \emph{remember} key facts about a particular topic, an important first step in learning at the university level. However, you will be asked to do much more than just remember facts. Your ultimate goal is to develop \textbf{evaluative judgement}, or the ability for you to judge for yourself the quality of your (or your peers') responses to prompts.

The discussion forums are a key way for you to do this. We have set up the forums in such a way that you will need to present a response to any given prompt before you see other learners' responses. We strongly encourage you to use this structure to formulate your own ideas before you present them in the forum, and then to use the responses of your peers to help you evaluate your own response.

Using these self-check activities in this way is designed to help you to succeed on the course assignments, upon which your final grade will be determined. These assignments will require you to \textbf{use} the facts of the course to generate unique responses to the prompts, based on your past experiences, knowledge, and ability to evaluate the quality of your own work.

\hypertarget{how-to-navigate-this-book}{%
\subsection*{How To Navigate This Book}\label{how-to-navigate-this-book}}
\addcontentsline{toc}{subsection}{How To Navigate This Book}

To move quickly to different portions of the book, click on the appropriate chapter or section in the table of contents on the left. The buttons at the top of the page allow you to show/hide the table of contents, search the book, change font settings, download a pdf or ebook copy of this book, or get hints on various sections of the book.
\includegraphics{assets/course-intro/menu.png}

The faint left and right arrows at the sides of each page (or bottom of the page if it's narrow enough) allow you to step to the next/previous section. Here's what they look like:
\includegraphics{assets/course-intro/left_arrow.png} \includegraphics{assets/course-intro/right_arrow.png}

You can also download an offline copy of this book in various formats, such as pdf or an ebook. If you are having any accessibility or navigation issues with this book, please reach out to your instructor or our online team at \href{mailto:elearning@twu.ca}{\nolinkurl{elearning@twu.ca}}.

\hypertarget{course-units}{%
\subsection*{Course Units}\label{course-units}}
\addcontentsline{toc}{subsection}{Course Units}

This course is organized into 6 units. Each unit of the course will provide you with the following information:

\begin{itemize}
\tightlist
\item
  A general overview of the key concepts that will be addressed during the unit.\\
\item
  Specific learning outcomes and topics for the unit.\\
\item
  Learning activities to help you engage with the concepts. These often include key readings, videos, and reflective prompts.\\
\item
  The Assessment section provides details on assignments you will need to complete throughout the course to demonstrate your understanding of the course learning outcomes.
\end{itemize}

\begin{caution}
Note that assessments, including assignments and discussion posts will be submitted in Moodle. See the Assessment tab in Moodle for the assignment dropboxes.
\end{caution}

\hypertarget{course-activities}{%
\subsection*{Course Activities}\label{course-activities}}
\addcontentsline{toc}{subsection}{Course Activities}

Below is some key information on features you will see throughout the course.~

\begin{reflect}
\textbf{\emph{Learning Activity}}\\
This box will prompt you to engage in course concepts, often by viewing resources and reflecting on your experience and/or learning. Most learning activities are ungraded and are designed to help prepare you for the assessment in this course.
\end{reflect}

\begin{assessment}
\textbf{\emph{Assessment}}\\
This box will signify an assignment or discussion post you will submit in Moodle. Note that these demonstrate your understanding of the course learning outcomes. Be sure to review the grading rubrics for each assignment.
\end{assessment}

\begin{progress}
\textbf{\emph{Checking Your Learning}}\\
This box is for checking your understanding, to make sure you are ready for what follows. Ways to check your learning might include self-check quizzes or questions for discussion. These activities are not graded but are critical for you to be able to begin to develop evaluative judgement in this domain of knowledge.
\end{progress}

\begin{caution}
\textbf{\emph{Note}}\\
This box signifies key notes. It may also warn you of possible problems or pitfalls you may encounter!
\end{caution}

\hypertarget{thought-and-language}{%
\chapter{Thought and Language}\label{thought-and-language}}

\hypertarget{overview}{%
\section*{Overview}\label{overview}}
\addcontentsline{toc}{section}{Overview}

\textbf{\emph{Welcome to Psych 106}}

We begin this course by reviewing some important information on scientific research methodologies. While this review will be self-guided, it is critical that you understand the important elements that constitute valid and reliable scientific research as it is these methodologies that serve as the backbone for psychological studies.

After taking time to review scientific research methodologies, we will turn our attention to the subject of Thought and Language. Developing an understanding of Thought and Language helps us better understand how we think, how we organize our thoughts and our knowledge of the world around us, and how we communicate and act out on this information.

\hypertarget{topics}{%
\subsection*{Topics}\label{topics}}
\addcontentsline{toc}{subsection}{Topics}

This unit is divided into the following topics:

\begin{enumerate}
\def\labelenumi{\arabic{enumi}.}
\tightlist
\item
  Review- Scientific Research\\
\item
  Thinking and Problem Solving\\
\item
  Cognitive Biases\\
\item
  Language and Thought\\
\item
  Animal Language
\end{enumerate}

\hypertarget{learning-outcomes}{%
\subsection*{Learning Outcomes}\label{learning-outcomes}}
\addcontentsline{toc}{subsection}{Learning Outcomes}

When you have completed this unit, you should be able to:

\begin{itemize}
\tightlist
\item
  Define key terminology related to principles of scientific research, research designs, and statistics.\\
\item
  Explain the five characteristics of quality scientific research, and the pros and cons of descriptive, correlational, and experimental research designs.\\
\item
  Determine how biases might influence the outcome of a study and how experiments help demonstrate cause-and-effect relationships.\\
\item
  Apply the concepts of reliability and validity to examples and concepts of experimental methods to research examples.\\
\item
  Assess whether anecdotes, authority figures, and common sense are reliably truthful sources of information.\\
\item
  Understand what it means for variables to be positively or negatively correlated and how and why psychologists use significance tests.
\end{itemize}

\hypertarget{activity-checklist}{%
\subsection*{Activity Checklist}\label{activity-checklist}}
\addcontentsline{toc}{subsection}{Activity Checklist}

Here is a checklist of learning activities you will benefit from in completing this unit. You may find it useful for planning your work.

\hypertarget{unit-1}{%
\subsection*{\texorpdfstring{\textbf{\emph{Unit 1:}}}{Unit 1:}}\label{unit-1}}
\addcontentsline{toc}{subsection}{\textbf{\emph{Unit 1:}}}

\textbf{Read and Reflect} \{-\}

\begin{itemize}
\tightlist
\item
  Read \emph{Krause et al.~(2021). Revel for An Introduction to Psychological Science, 3rd Canadian Edition}\\
\item
  Review \textbf{UNIT 1 Slides}
\end{itemize}

CLICK HERE

An Introduction to Psychological Science - Chapter 2: Reading and Evaluating Scientific Research

Learning Objectives

\begin{itemize}
\tightlist
\item
  Know the key terminology related to principles of scientific research.\\
\item
  Understand the five characteristics of quality scientific research.\\
\item
  Understand how biases might influence the outcome of a study.\\
\item
  Apply the concepts of reliability and validity to examples.\\
\item
  Analyze whether anecdotes, authority figures, and common sense are reliably truthful sources of information.
\end{itemize}

Five Characteristics of Quality Scientific Research

\begin{itemize}
\tightlist
\item
  Based on measurements that are objective, valid, and reliable\\
\item
  Generalizable\\
\item
  Use of techniques that reduce bias\\
\item
  Made public\\
\item
  Can be replicated
\end{itemize}

Scientific Measurement: Objectivity(1 of 2)

\begin{itemize}
\tightlist
\item
  Objective measurements (p.~31)

  \begin{itemize}
  \tightlist
  \item
    e.g.~weight\\
  \end{itemize}
\item
  Variable (p.~31)\\
\item
  Measuring variables - Examples

  \begin{itemize}
  \tightlist
  \item
    Functional magnetic resonance imaging (fMRI)\\
  \item
    Blood or saliva\\
  \item
    Self-reporting
  \end{itemize}
\end{itemize}

Scientific Measurement: Objectivity(1 of 2)

\includegraphics{assets/unit_1/PSYC106-Chs2-ResearchandThoughtandLanguage-3rdEd.png}

\emph{slide showing Operational Definitions}

Scientific Measurement: Reliability and Validity

\begin{itemize}
\tightlist
\item
  Reliability (p.~32)

  \begin{itemize}
  \tightlist
  \item
    Consistent and stable\\
  \end{itemize}
\item
  Validity (p.~32)

  \begin{itemize}
  \tightlist
  \item
    True measurements
  \end{itemize}
\end{itemize}

Generalizability of Results (1 of 2)

\begin{itemize}
\tightlist
\item
  Generalizability (p.~33)

  \begin{itemize}
  \tightlist
  \item
    Outside the laboratory\\
  \end{itemize}
\item
  Study large groups

  \begin{itemize}
  \tightlist
  \item
    Population (p.~33)

    \begin{itemize}
    \tightlist
    \item
      Sample (p.~33)
    \end{itemize}
  \end{itemize}
\end{itemize}

Generalizability of Results (2 of 2)

\begin{itemize}
\tightlist
\item
  Best reflection of population

  \begin{itemize}
  \tightlist
  \item
    Random sample (p.~33)\\
  \end{itemize}
\item
  Settle for easier sample

  \begin{itemize}
  \tightlist
  \item
    Convenience sample (p.~33)\\
  \end{itemize}
\item
  Location of study

  \begin{itemize}
  \tightlist
  \item
    Laboratory research\\
  \item
    Naturalistic research\\
  \item
    Ecological validity (p.~33)
  \end{itemize}
\end{itemize}

Sources of Bias in Psychological

\begin{itemize}
\tightlist
\item
  Research\\
\item
  Researcher Bias\\
\item
  Subject/Participant Bias\\
\item
  Hawthorne effect (p.~35)\\
\item
  Social Desirability (p.~35)
\end{itemize}

Working the Scientific Literacy Model: Demand Characteristics and
Participant Behaviour (1 of 2)

\begin{itemize}
\tightlist
\item
  What do we know about how bias affects research participants?

  \begin{itemize}
  \tightlist
  \item
    Demand characteristics (p.~36)\\
  \end{itemize}
\item
  How can science test the effects of demand characteristics on behaviour?

  \begin{itemize}
  \tightlist
  \item
    Backpack scenario
  \end{itemize}
\end{itemize}

Working the Scientific Literacy Model: Demand Characteristics and Participant Behaviour (2 of 2)

\begin{itemize}
\tightlist
\item
  How can we critically evaluate the issue of bias in research?

  \begin{itemize}
  \tightlist
  \item
    Researcher bias

    \begin{itemize}
    \tightlist
    \item
      Bright rats vs.~dull rats\\
    \end{itemize}
  \end{itemize}
\item
  Why is this relevant?

  \begin{itemize}
  \tightlist
  \item
    Bias compromises studies\\
  \item
    Placebo effect (p.~35)
  \end{itemize}
\end{itemize}

Psych @ The Hospital: The Placebo Effect

\begin{itemize}
\tightlist
\item
  Debate about placebo effect

  \begin{itemize}
  \tightlist
  \item
    ``All in their head''\\
  \item
    Actual physiological response\\
  \end{itemize}
\item
  Brain activity in regions involved in pain

  \begin{itemize}
  \tightlist
  \item
    Multiple ways for placebos to affect our responses to pain
  \end{itemize}
\end{itemize}

Techniques That Reduce Bias

\begin{itemize}
\tightlist
\item
  Anonymity\\
\item
  Confidentiality\\
\item
  Inform participants\\
\item
  Single-blind study (p.~37)\\
\item
  Double-blind study (p.~37)
\end{itemize}

Sharing the Results

\begin{itemize}
\tightlist
\item
  Academic journals

  \begin{itemize}
  \tightlist
  \item
    Peer review (p.~37)\\
  \item
    Replication (p.~38)
  \end{itemize}
\end{itemize}

Five Characteristics of Poor Research (1 of 2)

\begin{itemize}
\tightlist
\item
  Lack of falsifiable hypotheses (p.~38)

  \begin{itemize}
  \tightlist
  \item
    Testability requires faisifiability\\
  \end{itemize}
\item
  Anecdotal evidence (p.~38)

  \begin{itemize}
  \tightlist
  \item
    weight loss commercials\\
  \end{itemize}
\item
  Biased selection of data
\end{itemize}

Five Characteristics of Poor Research (2 of 2)

\begin{itemize}
\tightlist
\item
  Appeal to authority (p.~39)

  \begin{itemize}
  \tightlist
  \item
    Corresponding data?\\
  \item
    Biased expert?\\
  \end{itemize}
\item
  Appeal to common sense (p.~39)

  \begin{itemize}
  \tightlist
  \item
    Earth is centre of universe
  \end{itemize}
\end{itemize}

2.2 Learning Objectives

\begin{itemize}
\tightlist
\item
  Know the key terminology related to research designs.\\
\item
  Understand what it means when variables are positively or negatively correlated.\\
\item
  Understand how experiments help demonstrate cause-and-effect relationships.\\
\item
  Apply the terms and concepts of experimental methods to research examples.\\
\item
  Analyze the pros and cons of descriptive, correlational, and experimental research designs.
\end{itemize}

Descriptive Research (1 of 2)

\begin{itemize}
\tightlist
\item
  Descriptive data

  \begin{itemize}
  \tightlist
  \item
    From observations\\
  \item
    No attempt to explain why\\
  \end{itemize}
\item
  Qualitative Research (p.~42)
\end{itemize}

Descriptive Research (2 of 2)

\begin{itemize}
\tightlist
\item
  Case study (p.~42)

  \begin{itemize}
  \tightlist
  \item
    Extensive details\\
  \item
    Lacks generalizability\\
  \end{itemize}
\item
  Naturalistic observation (p.~44)\\
\item
  Self-reporting (p.~45)

  \begin{itemize}
  \tightlist
  \item
    Participant makes the observations
  \end{itemize}
\end{itemize}

\includegraphics{assets/unit_1/slide_21.png}

\emph{Slide showing correlations depicted in scatterplots}

Myths in Mind: Beware of Illusory Correlations

\begin{itemize}
\tightlist
\item
  Illusory correlations (p.~47)

  \begin{itemize}
  \tightlist
  \item
    Crime increases when the moon is full\\
  \item
    Opposites attract\\
  \item
    Gamblers on a ``hot streak''\\
  \item
    Stereotypes
  \end{itemize}
\end{itemize}

\includegraphics{assets/unit_1/slide_23.png}

\emph{Slide showing - Elements of an Experiment}

Experimental Research: The Quasi-Experimental Method

\begin{itemize}
\tightlist
\item
  Quasi-experimental research (p.~49)

  \begin{itemize}
  \tightlist
  \item
    Random assignment not always possible

    \begin{itemize}
    \tightlist
    \item
      Comparing men and women\\
    \end{itemize}
  \item
    Cannot determine cause-and-effect
  \end{itemize}
\end{itemize}

2.4 Learning Objectives

\begin{itemize}
\tightlist
\item
  Know the key terminology of statistics.\\
\item
  Understand how and why psychologists use significance tests.\\
\item
  Apply your knowledge to interpret the most frequently used types of graphs.\\
\item
  Analyze the choice of central tendency statistics based on the shape of the distribution.
\end{itemize}

Descriptive Statistics

\begin{itemize}
\tightlist
\item
  Descriptive statistics (p.~60)

  \begin{itemize}
  \tightlist
  \item
    Frequency\\
  \item
    Central tendency\\
  \item
    Variability
  \end{itemize}
\end{itemize}

\includegraphics{assets/unit_1/slide_27.png}

\emph{Slide showing - Graphing Psychological Data}

\includegraphics{assets/unit_1/slide_28.jpg}

\emph{Slide showing - Skewed Distributions}

\includegraphics{assets/unit_1/slide_29.png}

\emph{Slide showing - Central Tendency in Symmetrical Distributions}

\includegraphics{assets/unit_1/slide_30.png}

\emph{Slide showing - Visualizing Variability}

\includegraphics{assets/unit_1/slide_31.png}

\emph{Slide showing - Testing a Simple Hypothesis}

\includegraphics{assets/unit_1/slide_32.png}

\emph{Slide showing - How Variability Affects Hypothesis Testing}

True or False?

\begin{itemize}
\tightlist
\item
  T F 1. People more easily detect male prejudice against females than female against males or female against females.\\
\item
  T F 2. In general, people underestimate how much they really know.\\
\item
  T F 3. It takes less compelling evidence to change our beliefs than it did to create them in the first place.\\
\item
  T F 4. The babbling of an infant at 4 months of age makes it clear whether the infant is French, Korean, or Ethiopian.\\
\item
  T F 5. Some people can write but not read.\\
\item
  T F 6. Many bilinguals report that they have different senses of self, depending on which language they are using.\\
\item
  T F 7. Imagining a physical activity triggers action in the same brain areas that are triggered when actually performing that activity.\\
\item
  T F 8. Only human beings seem capable of insight (the sudden realization of a problem's solution).\\
\item
  T F 9. Honeybees do a dance to communicate the direction and distance of a new food source to other bees.\\
\item
  T F 10. Apes are capable of communicating meaning by using symbols.
\end{itemize}

Thinking

\begin{itemize}
\tightlist
\item
  Thinking, or \emph{cognition}, refers to a process that involves knowing, understanding, remembering, and communicating.\\
\item
  Gr. \(\Phi\rho\omega\nu\epsilon\omega\) (pr. phrones) - to think, to mind; to be of opinion; to take thought, be considerate; to entertain sentiments or inclinations of a specific kind, to be minded; to be in a certain frame of mind; to imagine; to heed, pay regard to; to incline to; be set upon, mind
\end{itemize}

A little more Greek

\begin{itemize}
\tightlist
\item
  Mind (Gr. \(\Nu\omicron\upsilon\varsigma\)) - the mind, intellect; understanding, intelligent faculty; intellect, judgment; opinion, sentiment; mind, thought, conception; settled state of mind; frame of mind.
\end{itemize}

The limits of intuition

\begin{itemize}
\tightlist
\item
  A bat and a ball cost \$1.10 in total. The bat costs \$1 more than the ball. How much does the ball cost?\\
\item
  A man bought a horse for \$60 and sold it for \$70. Then he bought the same horse back for \$80 and again sold it, for \$90. How much money did he make in the horse business?
\end{itemize}

\includegraphics{assets/unit_1/slide_39.png}

\emph{Slide showing - Modules}

8.1 Learning Objectives

\begin{itemize}
\tightlist
\item
  Know the key terminology associated with concepts and categories.\\
\item
  Understand theories of how people organize their knowledge about the world.\\
\item
  Understand how experience and culture can shape the way we organize our knowledge.\\
\item
  Apply your knowledge to identify prototypical examples.\\
\item
  Analyze the claim that the language we speak determines how we think.
\end{itemize}

Concepts and Categories

\begin{itemize}
\tightlist
\item
  Concept (p.~294)

  \begin{itemize}
  \tightlist
  \item
    Divided into smaller groups\\
  \end{itemize}
\item
  Categories (p.~294)
\end{itemize}

\includegraphics{assets/unit_1/slide_42.png}

\emph{Slide showing - Using the Definition of a Triangle to Categorize Shapes}

\includegraphics{assets/unit_1/slide_43.png}

\emph{Slide showing - Categorizing Objects According to the Definition of Bird}

\includegraphics{assets/unit_1/slide_44.png}

\emph{Slide showing - A Prototypical Bird}

\includegraphics{assets/unit_1/slide_45.png}

\emph{Slide showing - A Semantic Network Diagram for the Category ``Animal''}

Working the Scientific Literacy Model: Priming and Semantic
Networks (1 of 2)

\begin{itemize}
\tightlist
\item
  What do we know about semantic networks?

  \begin{itemize}
  \tightlist
  \item
    Priming (p.~297)\\
  \end{itemize}
\item
  How can scientists explain priming effects?

  \begin{itemize}
  \tightlist
  \item
    Lexical Decision Task
  \end{itemize}
\end{itemize}

Working the Scientific Literacy Model: Priming and Semantic
Networks (2 of 2)

\begin{itemize}
\tightlist
\item
  Can we critically evaluate this information?

  \begin{itemize}
  \tightlist
  \item
    Strength of priming varies\\
  \item
    Experiments difficult to replicate\\
  \end{itemize}
\item
  Why is this relevant?

  \begin{itemize}
  \tightlist
  \item
    Advertising
  \end{itemize}
\end{itemize}

Categorization and Experience

\begin{itemize}
\tightlist
\item
  Categorization is based on experience

  \begin{itemize}
  \tightlist
  \item
    Efficient process\\
  \item
    But can also result in errors
  \end{itemize}
\end{itemize}

Categories and the Brain (1 of 2)

\begin{itemize}
\tightlist
\item
  Categories, Memories, and the Brain\\
\item
  Category-specific visual agnosia (CSVA)

  \begin{itemize}
  \tightlist
  \item
    Living vs.~non-living categories
  \end{itemize}
\end{itemize}

\includegraphics{assets/unit_1/slide_50.png}

\emph{Slide showing - Naming Errors for a CSVA Patient}

\includegraphics{assets/unit_1/slide_51.png}

\emph{Slide showing - Your Culture and Your Point of View}

\includegraphics{assets/unit_1/slide_52.png}

\emph{Slide showing - Brain Activity Varies by Culture}

Myths in Mind: How Many Words for Snow?

\begin{itemize}
\tightlist
\item
  Inuit have many words for snow

  \begin{itemize}
  \tightlist
  \item
    Aput = snow on the ground\\
  \item
    Gana = falling snow\\
  \item
    Exaggerated to dozens of words\\
  \end{itemize}
\item
  Canadians have many words for snow

  \begin{itemize}
  \tightlist
  \item
    Sticky snow\\
  \item
    Drifting snow\\
  \item
    Yellow snow
  \end{itemize}
\end{itemize}

8.2 Learning Objectives

\begin{itemize}
\tightlist
\item
  Know the key terminology of problem solving and decision making.\\
\item
  Understand the characteristics that problems have in common.\\
\item
  Understand how obstacles to problem solving are often self-imposed.\\
\item
  Apply your knowledge to determine if you tend to be a maximizer or a satisficer.\\
\item
  Analyze whether human thought is primarily logical or intuitive.
\end{itemize}

Defining and Solving Problems (1 of 2)

\begin{itemize}
\tightlist
\item
  Problem solving (p.~304)

  \begin{itemize}
  \tightlist
  \item
    Algorithms (p.~304)\\
  \item
    Heuristics (p.~304)
  \end{itemize}
\end{itemize}

\includegraphics{assets/unit_1/slide_56.png}

\emph{Slide showing - Problem Solving in Hangman}

\includegraphics{assets/unit_1/slide_57.png}

\emph{Slide showing - The Nine-Dot Problem}

\includegraphics{assets/unit_1/slide_58.png}

\emph{Slide showing - The Five-Daughter Problem}

\includegraphics{assets/unit_1/slide_59.png}

\emph{Slide showing - The Two-String Problem}

Representativeness and Availability

\begin{itemize}
\tightlist
\item
  Conjunction fallacy (p.~308)\\
\item
  Representativeness heuristic (p.~308)\\
\item
  Availability heuristic (p.~308)
\end{itemize}

Anchoring Effects

\begin{itemize}
\tightlist
\item
  Anchoring effect (p.~310)

  \begin{itemize}
  \tightlist
  \item
    In what year did British Columbia become part of Canada?\\
  \item
    More affected when generated by individual
  \end{itemize}
\end{itemize}

Framing Effects (1 of 2)

\begin{itemize}
\tightlist
\item
  Decision-making influenced by how problem is framed (p.~310)\\
\item
  Example: Vaccine A vs.~Vaccine B
\end{itemize}

\includegraphics{assets/unit_1/slide_63.png}

\emph{Slide showing - Framing Effects}

Belief Perseverance and Confirmation Bias

\begin{itemize}
\tightlist
\item
  Belief perseverance (p.~310)\\
\item
  Confirmation bias (p.~311)
\end{itemize}

Can dramatically influence beliefs, especially for complex, emotionally-charged issues (e.g.~politics)

What do we know about maximizing and satisficing?

\begin{itemize}
\tightlist
\item
  Two types of consumers

  \begin{itemize}
  \tightlist
  \item
    Satisficers = ``good enough*\\
  \item
    Maximizers = evaluate every option\\
  \end{itemize}
\item
  Paradox of choice
\end{itemize}

How can scientists explain maximizing and satisficing?

\includegraphics{assets/unit_1/slide_66.png}

\emph{Slide showing - Satisfaction of Maximizers and Satisficers}

Working the Scientific Literacy Model: Maximizing and Satisficing in Complex Decisions (3 of 3)

\begin{itemize}
\tightlist
\item
  Can we critically evaluate this information?

  \begin{itemize}
  \tightlist
  \item
    Maximizers might expect more\\
  \item
    Correlational research
  \end{itemize}
\item
  Why is this relevant?

  \begin{itemize}
  \tightlist
  \item
    Planning for the future
  \end{itemize}
\end{itemize}

8.3 Learning Objectives

\begin{itemize}
\tightlist
\item
  Know the key terminology from the study of language.\\
\item
  Understand how language is structured.\\
\item
  Understand how genes and the brain are involved in language use.\\
\item
  Apply your knowledge to distinguish between units of language such as phonemes and morphemes.\\
\item
  Analyze whether species other than humans are able to use language.
\end{itemize}

\includegraphics{assets/unit_1/slide_69.png}

\emph{Slide showing - Two Language Centres of the Brain}

Properties of Language

\begin{itemize}
\tightlist
\item
  Language (p.~317)\\
\item
  Unique features

  \begin{itemize}
  \tightlist
  \item
    Communicate objects and events not in present time and place\\
  \item
    Produce new meanings\\
  \item
    Passed down naturally to children
  \end{itemize}
\end{itemize}

Phonemes and Morphemes: The Basic Ingredients of Language

\begin{itemize}
\tightlist
\item
  Phonemes (p.~318)

  \begin{itemize}
  \tightlist
  \item
    ``T''
  \end{itemize}
\item
  Morphemes (p.~318)

  \begin{itemize}
  \tightlist
  \item
    Pig, ish, or pigish\\
  \item
    Productivity
  \end{itemize}
\item
  Semantics (p.~318)
\end{itemize}

\includegraphics{assets/unit_1/slide_72.png}

\emph{Slide showing - How syntax helps us to understand language}

\includegraphics{assets/unit_1/slide_73.png}

\emph{Slide showing - Pragmatic Rules Guiding Language Use}

The Development of Language (1 of 2)

Infants, sound perception, and language acquisition

\begin{itemize}
\tightlist
\item
  Identifying Sounds\\
\item
  Fast mapping (p.~320)
\end{itemize}

\includegraphics{assets/unit_1/slide_75.png}

\emph{Slide showing - Milestones in Language Acquisition and Speech}

Sensitive Periods for Language

\begin{itemize}
\tightlist
\item
  Sensitive period

  \begin{itemize}
  \tightlist
  \item
    Brains are primed to develop language skills\\
  \item
    Ability fades starting seventh year\\
  \item
    Same with sign language
  \end{itemize}
\end{itemize}

The Bilingual Brain

\begin{itemize}
\tightlist
\item
  Costs

  \begin{itemize}
  \tightlist
  \item
    Smaller vocabulary\\
  \item
    Word access
  \end{itemize}
\item
  Benefits

  \begin{itemize}
  \tightlist
  \item
    Executive functions\\
  \item
    Health benefits
  \end{itemize}
\end{itemize}

Working the Scientific Literacy Model: Genes and Language (1 of 4)

\begin{itemize}
\tightlist
\item
  What do we know about genes and language?

  \begin{itemize}
  \tightlist
  \item
    Language evolved to solve problems\\
  \item
    Number of genes involved
  \end{itemize}
\item
  Which scientific evidence supports a genetic basis of language?

  \begin{itemize}
  \tightlist
  \item
    FOXP2 gene
  \end{itemize}
\end{itemize}

\includegraphics{assets/unit_1/slide_79.png}

\emph{Slide showing - Inheritance Pattern for the Mutated FOXP2 Gene in the KE Family}

\includegraphics{assets/unit_1/slide_80.png}

\emph{Slide showing - Brain Scans Taken While Members of the KE Family Completed a Speech Task}

Working the Scientific Literacy Model: Genes and Language (4 of 4)

\begin{itemize}
\tightlist
\item
  Can we critically evaluate this evidence?

  \begin{itemize}
  \tightlist
  \item
    Many genes work together\\
  \item
    FOXP2 not unique to humans

    \begin{itemize}
    \tightlist
    \item
      Language is unique to humans
    \end{itemize}
  \end{itemize}
\item
  Why is this relevant?

  \begin{itemize}
  \tightlist
  \item
    Links between genes and language
  \end{itemize}
\end{itemize}

Can Animals Use Language?

\begin{itemize}
\tightlist
\item
  Chimpanzee Viki

  \begin{itemize}
  \tightlist
  \item
    Cross-fostered (p.~324)\\
  \item
    Four words
  \end{itemize}
\item
  Chimpanzee Washoe

  \begin{itemize}
  \tightlist
  \item
    ASL

    \begin{itemize}
    \tightlist
    \item
      200 signs\\
    \item
      Generalized words
    \end{itemize}
  \end{itemize}
\item
  Bonobo Kanzi

  \begin{itemize}
  \tightlist
  \item
    Lexigrams

    \begin{itemize}
    \tightlist
    \item
      350 symbols\\
    \item
      3,000 spoken words
    \end{itemize}
  \end{itemize}
\end{itemize}

\begin{caution}
\textbf{Note:} the slides are intended to supplement the information found in your textbook. If you are having trouble viewing them, they can also be downloaded by scrolling to the bottom of the screen and clicking on the ``Unit 1- Slides'' link.*
\end{caution}

\hypertarget{learning-activities}{%
\subsection*{Learning Activities:}\label{learning-activities}}
\addcontentsline{toc}{subsection}{Learning Activities:}

\begin{reflect}
\textbf{Chapter 2 Review Quiz}

\begin{itemize}
\tightlist
\item
  Practice Quiz to self-assess your own comprehension of important terms from Chapter 2.\\
\item
  Not for formal evaluation.
\end{itemize}

\textbf{Problem Solving Activity}

\begin{itemize}
\tightlist
\item
  Solve some problems by utilizing some of the cognitive strategies we learned about in this topic.
\end{itemize}

\textbf{Problem Solving Practice}

\begin{itemize}
\tightlist
\item
  Explore problem solving activities and reflect on the strategies you incorporate as you discover solutions.
\end{itemize}

\textbf{Introduction to Visualization}

\begin{itemize}
\tightlist
\item
  Article introduces visualization and provides an opportunity to practice this skill.
\end{itemize}

\textbf{Learning Lab Preparation}

\begin{itemize}
\tightlist
\item
  Each topic will provide a question or scenario for you to consider prior to attending your Learning Lab. Be sure to carefully consider each prompt as you will be expected to contribute to the group discussion.
\end{itemize}
\end{reflect}

\hypertarget{resources}{%
\subsection*{Resources}\label{resources}}
\addcontentsline{toc}{subsection}{Resources}

Here are some additional resources that will help you complete this unit:

\begin{itemize}
\tightlist
\item
  Krause, M., Corts, D., Smith, S. C., \& Dolderman, D. (2018). \emph{Revel for An Introduction to Psychological Science, 2nd Canadian Edition.} Pearson Ed.\\
\item
  Other resources will be provided online.
\end{itemize}

\hypertarget{what-is-psychology}{%
\section{What is Psychology}\label{what-is-psychology}}

We begin our course with a quick challenge: \textbf{\emph{In your own words, define ``psychology.''}}

According to your definition, how is psychology different from other academic areas that would study humans (\emph{for example, philosophy, literature, or history})? If you said ``Pyschology is different because it uses the scientific method''- give yourself a pat on the back

You will begin your study of the scientific method by reading your textbook. The parable below (\emph{from Philipchalk's Social Psychology textbook}), however, helps illustrate the scientific method with three ``helpful'' approaches to a problem, including a simple experiment:

\emph{Once upon a time there were three brothers. One day while they were working in their father's field, they saw an old man coming along the road. The old man greeted them, and then struggled on along the road, limping terribly. After that, every day at the same time, the brothers greeted the old man and watched as he hobbled by. When a month had passed, they were so impressed that they each did something. The first brother wrote a compelling story about perseverance in the face of the ravages of old age. It encouraged many people. The second brother painted a moving portrait of the old man, stooped over and limping along. People were inspired. The third brother, who had observed the old man very closely, asked him one day if he could exchange shoes with him. The old man was surprised, but he gladly agreed. When the old man walked away he did not limp. The next day the third brother gave the old man his shoes back and watched as he limped on his way. On the third day, the brother again exchanged shoes with the old man. Then he took the old man's shoes to a shoemaker and had them repaired. When the brother gave them back to the old man he was delighted. The old man put on the shoes, thanked the brother, and walked away without a limp.}

Although each brother made a positive contribution, the third brother solved the man's problem because he discovered its cause. To do this, \textbf{\emph{he used the scientific method and he conducted an experiment (Philipchalk, 1994).}}

I think psychology is one of the most interesting areas of study there is, first, because it studies people, people like you and me, and we're interesting Second, I like psychology because it is so broad. Psychologists, as you will soon see, study everything from nerve conduction in single cells, all the way to the influence of groups on our behaviour---and everything in between. Finally, psychologists don't just speculate and theorize, they look for evidence for their ideas. If they don't find sufficient evidence, they change their ideas; and I like that. Which leads us back to the scientific method and how psychology began.

\hypertarget{learning-activity}{%
\subsection*{Learning Activity}\label{learning-activity}}
\addcontentsline{toc}{subsection}{Learning Activity}

\begin{reflect}
\textbf{Chapter 2 Review Quiz}

In order to review some of the major concepts from the text, take the following unmarked quiz. Although you will not be evaluated on these terms, they will assist you in the assignments for this course.
\end{reflect}

\hypertarget{thinking-and-problem-solving}{%
\section{Thinking and Problem Solving}\label{thinking-and-problem-solving}}

\hypertarget{thinking}{%
\subsection*{Thinking}\label{thinking}}
\addcontentsline{toc}{subsection}{Thinking}

``So God created man in his own image, in the image of God created he him; male and female created he them.'' (Genesis 1:27)

``I will praise thee; for I am fearfully and wonderfully made.'' (Psalm 139:14)

The human image of God means many things. It seems that one aspect of this image is our thinking ability, including our ability to solve problems and speak. How important is our thinking ability in our reflection of God's image? What does your answer mean for people with less ability? What about people who lose abilities due to accident or disease (e.g., Alzheimer's patients)?

\hypertarget{algorithms-heuristics}{%
\subsection*{Algorithms \& Heuristics}\label{algorithms-heuristics}}
\addcontentsline{toc}{subsection}{Algorithms \& Heuristics}

Algorithms and heuristics can be confusing. An algorithm is a guaranteed route to a solution, but it may be the long way around to success. If you knew a person lived somewhere in a large residence hall, an algorithm for finding that individual would be to knock on every door until you located the person. Heuristics, on the other hand, sug­gest that you would first ask friends where to locate the person, or check a list, then knock on the appropriate door to locate the person. Another way to think of heuristics is the phrase ``rule of thumb.'' Can you think of some rules of thumb that you have learned from your various job experiences? They may have to do with how long to cook a hamburger, or when to refill a machine, or how to get a date.

Consider the following example and explanation taken from Invitation to Social Psychology by Ron Philipchalk:

\emph{In the second part of the book, they tell you how to crack a safe. There are all kinds of ninny-pinny, dopey things, like ``It might be a good idea to try a date for the combination, because lots of people like to use dates.'' Or ``Think of the psychology of the owner of the safe, and what he might use for the combination.'' And ``The secretary is often worried that she might forget the combination of the safe, so she might write it down in one of the following places---along the edge of her desk drawer, on a list of names and addresses . . .'' and so on . . . .}

\emph{I also did a certain amount of systematic study. For instance, a typical combination was 69-32-21. How far off could a number be when you're opening the safe? If the number was 69, would 68 work? Would 67 work? On the particular locks we had, the answer was yes for both, but 66 wouldn't work. You could be off by two in either direction. That meant you only had to try one out of five numbers, so you could try zero, five, ten, fifteen, and so on. With twenty such numbers on a wheel of 100, that was 8000 possibilities instead of the 1,000,000 you would get if you had to try every single number. . . .}

\emph{I practiced all the time on my own safe so I could do this process as fast as I could and not get lost in my mind as to which number I was pushing and mess up the first number. Like a guy who practices sleight of hand, I got it down to an absolute rhythm so I could try the 400 possible back numbers in less than half an hour. That meant I could open a safe in a maximum of eight hours---with an average time of four hours. (Surely You're Joking Mr.~Feynman, p.~140)}

Mr.~Feynman's safecracking system succeeds because he methodically works through every possible combination. By logical analysis he has discovered which 8,000 possibilities out of 1,000,000 he needs to try. We call this type of logical step-by-step procedure for solving problems an algorithm (Newell \& Simon, 1972; Simon, 1981). If you use the correct algorithm your success is guaranteed.

But sometimes it can take a long time to discover the correct algorithm. And employing an algorithm is often time consuming. Mr.~Feynman spent days developing his system and it took hours to open a safe. You could certainly open a safe much faster if you found the combination on the edge of the secretary's drawer.

The shortcuts Mr.~Feynman calls ``ninny-pinny, dopey things'' are examples of heuristics. Heuristics are rule-of-thumb strategies for solving problems, shortcuts we develop from our experience. Heuristics often help us eliminate improbable alternatives and guide us to the most likely solution to a problem. Despite his appreciation for algorithms, Mr.~Feynman discovered heuristics can be useful. He found, for example, that safe owners often did not bother to change the factory set combination when they received a new safe. In one office building the temporary factory combinations 25-0-25 or 50-25-50 opened one safe in five

Algorithms and heuristics are examples of cognitive strategies---mental plans we use to make decisions and solve problems.

\hypertarget{learning-activities-1}{%
\subsection*{Learning Activities}\label{learning-activities-1}}
\addcontentsline{toc}{subsection}{Learning Activities}

\begin{reflect}
\textbf{Read and Reflect}

In addition to the content above, you are also responsible for reading through the following:

\emph{Krause et. al (2018). Revel for An Introduction to Psychological Science, 2nd Canadian Edition. Chapter 8}

While all of these pages may not relate directly to this unit's discussion, consistent reading will help you keep pace, as well as provide necessary background knowledge when you need it.

\textbf{Problem Solving Activity}

In this lesson, we spent time exploring cognitive strategies used to solve problems and make decisions. We now have an opportunity to practice this on our own Take a look at the following problems and see if you can find a solution. As you work through the problems, think about what cognitive strategies you are implementing as you make each decision:

\textbf{\emph{Problem A}}

\begin{itemize}
\tightlist
\item
  Take a look at the following Roman Numeral: \textbf{IX}
\item
  Now add one line to the Roman numeral \textbf{IX} to make it six
\end{itemize}

Click here for the solution.

The answer is to add a curved line shaped like an ``S'' (i.e., ``SIX''). In writing, the problem looks simple, but you might want to try it aloud on a friend. There is a mental set that one must add a straight line and have some form of Roman numeral on the page.

\textbf{\emph{Problem B}}

Your task is to plant trees on Arbor Day. You have ten trees that must be planted in five rows of four trees each. How would you plant the trees?

Click here for the solution.

The answer is that you would arrange them at the vertices and cross-points of a 5-pointed star

\textbf{Problem Solving Practice}

Below is a website that provides more opportunity to work through, and solve, some problems. Specifically, this resource explores the idea of \textbf{\emph{Assumptions}} and the role assumptions play when solving problems. Furthermore, this is a valuable resource as it also explores other important techniques to be implemented when solving problems.

Click on the following link and read through the information as you continue to practice your problem solving:

\href{https://www.virtualsalt.com/crebook4.htm}{\textbf{Virtual Salt}}

\hypertarget{learning-lab-preparation}{%
\subsubsection*{Learning Lab Preparation}\label{learning-lab-preparation}}
\addcontentsline{toc}{subsubsection}{Learning Lab Preparation}

Your Learning Lab for this unit will focus on group discussion as explore the topics of this unit in more detail. As you prepare for your Learning Lab, one possible scenario that discussion will focus on is below- please prepare some thoughts to share with the group:

In the largest sense, society is breaking into two classes:

\emph{The first class are people who know how to think. These people realize that most problems are open to examination and creative solution. If a problem appears in the lives of these people, their intellectual training will quickly lead them to a solution or an alternative statement of the problem. These people are the source of the most important product in today's economy -- ideas.}

\emph{The second class, the vast majority of society, are people who cannot think for themselves. I call these people `idea consumers' -- metaphorically speaking, they wander around in a gigantic open-air mall of facts and ideas. The content of their experience is provided by television, the Internet and other shallow data pools. These people believe collecting images and facts makes them educated and competent, and all their experiences reinforce this belief. The central, organizing principle of this class is that ideas come from somewhere else, from magical persons, geniuses, `them.'}

Consider the following prompts to help better prepare for the discussion:

\begin{itemize}
\tightlist
\item
  \textbf{\emph{Do you agree or disagree with this claim?}}\\
\item
  \textbf{\emph{Do you know people that fit in the second category? What causes this difference? How might it be changed?}}
\end{itemize}
\end{reflect}

\hypertarget{cognitive-biases}{%
\section{Cognitive Biases}\label{cognitive-biases}}

As if the biases mentioned in the textbook are not enough, here are a few more to watch out for, taken from Invitation to Social Psychology by Ron Philipchalk.

\hypertarget{the-gamblers-fallacy}{%
\subsection*{The Gambler's Fallacy}\label{the-gamblers-fallacy}}
\addcontentsline{toc}{subsection}{The Gambler's Fallacy}

Jill and Bob are the parents of three boys. Jill is pregnant again, and she and Bob are hoping the baby is a girl. In fact, they are confident the baby must be a girl because their previous three children were boys. If you agree with Jill and Bob that the baby is more likely to be a girl than a boy, then you---along with Jill and Bob---may be committing the gambler's fallacy. No matter how many boys have been born, the likelihood of a girl being born is the same as it always was, approximately 50 \% (assuming no biological abnormality or medical intervention).

The gambler's fallacy arises from our failure to recognize the independence of unconnected events. The result of a coin toss does not depend on the outcome of previous tosses; a child's sex at conception is not affected by the sex of prior conceptions; the cards dealt in a hand are not influenced by the distribution of cards on the previous deal; and so on. Each event in these sequences is independent of the others, although we tend to think that somehow there must be a connection.

\hypertarget{the-anchoring-and-adjustment-heuristic}{%
\subsection*{The Anchoring and Adjustment Heuristic}\label{the-anchoring-and-adjustment-heuristic}}
\addcontentsline{toc}{subsection}{The Anchoring and Adjustment Heuristic}

First impressions of a person exert a powerful influence on the way we interpret subsequent information about that person. This effect may be an example of a more general principle called the anchoring and adjustment heuristic. Information we use to establish a starting value (or anchor point) tends to be more influential in our decisions than subsequent information we use to adjust this value (Tversky \& Kahneman, 1974).

Daniel Cervone and his colleagues found, for example, that initial success or failure on a task can establish an anchor for feelings of self-efficacy. Students who initially succeed on a task and later fail have higher feelings of self-efficacy than students who initially fail and later succeed even though their overall level of success is the same. Final judgments of self-efficacy are biased in the direction of initial judgments (Cervone \& Palmer, 1990; Peake \& Cervone, 1989).

Salespeople often use the anchoring and adjustment heuristic to their advantage. Some real estate agents routinely show their clients an over-priced and unattractive house first in order to set an anchor point which, in effect, says, ``The kind of house you want is going to cost a lot.'' Once established, this expectation of high price changes very slowly and the clients are relieved to find an acceptable house in their price range (Northcraft \& Neale, 1987).

Car dealers too like us to set our sights high. Their so-called list price establishes an anchor or reference point that overshadows our subsequent evaluations, as I recently discovered. In looking for a certain model of car, I was attracted to a particular vehicle with an asking price of \$3,800 (``reduced from \$4,200''). I believed this price was too high, so I bargained with the vendor. Eventually, I bought the car for \$2,800. Did the high original asking price affect my decision? Yes, it probably did. Subsequent events indicated I still paid too much. I later bought an identical model in only slightly poorer condition for \$2,000. I was a victim of the anchoring and adjustment heuristic.

\textbf{\emph{Contrast Effects}}

My car purchase also illustrates a related distortion in judgment, the contrast effect. In contrast to the original price of \$4,200 my offer of \$2,800 seemed like a bargain. John Lynch, Jr.~and his colleagues (1991) found a similar effect with students. The students rated low-priced cars as less expensive when they were considered alongside high-priced cars (contrast effect), compared to when they were considered along with other low-priced cars (no contrast).

Research by Douglass Kenrick and his colleagues indicates that we also show contrast effects in evaluating other people. In one study (1980), male college students rated the attractiveness of potential blind dates. Subjects who gave their ratings after watching a TV show with attractive female actresses rated the potential dates as less attractive than did subjects who rated their potential dates before watching the show. In another study (1989), after viewing centerfold erotica, men found average women---and even their own wives---less attractive.

\hypertarget{heuristics-biases}{%
\subsection*{Heuristics \& Biases}\label{heuristics-biases}}
\addcontentsline{toc}{subsection}{Heuristics \& Biases}

By now you may be wondering why we fall prey to so many cognitive biases and errors. Well don't worry; our biases are actually a side effect of our cognitive efficiency. Most of our biases result from using heuristics, rules of thumb, or mental shortcuts that work very well. Sometimes they let us down, but overall, they improve the speed with which we handle mental problems---much like Mr.~Feynman's safe-cracking tricks. As we noted in the previous discussion, you could certainly open a safe much faster if you found the combination on the edge of the secretary's drawer. However, you won't always find the combination there, so limiting yourself to this approach would produce a kind of ``cognitive bias'' in your safe-cracking strategy.

\hypertarget{resources-online-articles-of-interest}{%
\subsection*{RESOURCES: Online Articles of Interest}\label{resources-online-articles-of-interest}}
\addcontentsline{toc}{subsection}{RESOURCES: Online Articles of Interest}

For additional information and examples, click on the link below:

\href{https://en.wikipedia.org/wiki/List_of_cognitive_biases}{\textbf{Cognitive Biases}}

\begin{reflect}
\hypertarget{learning-lab-preparation-1}{%
\subsection*{Learning Lab Preparation}\label{learning-lab-preparation-1}}
\addcontentsline{toc}{subsection}{Learning Lab Preparation}

Another focus of our discussion during our Learning Lab for this unit, will focus on biases. In order to prepare for participation in this discussion, consider the guiding prompt below- be sure to have some thoughts to contribute to the discussion:

\textbf{\emph{Give an example from your own experience of one of the cognitive biases, discussed here or in the textbook, that you have fallen prey to.}}
\end{reflect}

\hypertarget{language-and-thought}{%
\section{Language and Thought}\label{language-and-thought}}

\hypertarget{linguistic-relativity}{%
\subsection*{Linguistic Relativity}\label{linguistic-relativity}}
\addcontentsline{toc}{subsection}{Linguistic Relativity}

Benjamin Whorf's linguistic relativity hypothesis suggests that our language affects the way we see the world. Do you know any examples of weather terms, for example, that are unique to one area? Could knowing these terms help you to notice differences in weather that outsiders might not notice? What about in sports? Sports fans usually know terms to de­scribe certain strokes or plays. Does knowledge of these terms affect percep­tion? Can you think of some examples? What does it mean to ``clothesline'' someone, or ``post-up,'' or ``birdie?''

\hypertarget{imaginary-practice}{%
\subsection*{Imaginary Practice}\label{imaginary-practice}}
\addcontentsline{toc}{subsection}{Imaginary Practice}

Mental practice is now widely accepted in many areas. The following excerpt is taken from the \href{https://www.golfpsych.com}{\textbf{GolfPsych}} website. You may find further examples there as well.

``You can practice the mental aspects of your game anytime. We encourage our clients to do imagery practice of playing well in upcoming tournaments. This imaginary practice includes seeing the course, situation, doing a full mental pre-shot routine and seeing a good shot. You should also be feeling the way you do when you play your best. In addition, you should be practicing deep breathing and quieting your mind off-course. This is an extremely valuable tool that must be practiced to be effective. The mental game is much more than thinking positive thoughts. Take our Personality Assessment and get your own GolfPsych Report to receive our recommendations for you based on your personality and our research. Reading our book will also help you understand all aspects of a good mental game. During your practices and before your rounds you should also be practicing your mental skills.''

\hypertarget{learning-activity-1}{%
\subsection*{Learning Activity}\label{learning-activity-1}}
\addcontentsline{toc}{subsection}{Learning Activity}

\begin{reflect}
\hypertarget{introduction-to-visualization}{%
\subsubsection*{Introduction to Visualization}\label{introduction-to-visualization}}
\addcontentsline{toc}{subsubsection}{Introduction to Visualization}

In this Topic, we learned about the notion of mental practice. Below is an article that will take you through the process of visualization. Take some time to read the article and practice for yourself. Pay careful attention to your thoughts, your focus, your feelings as you engage in the process.

\begin{itemize}
\tightlist
\item
  \href{https://www.forbes.com/sites/bhaligill/2017/06/22/new-to-visualization-here-are-5-steps-to-get-you-started/\#60dafcdc6e3f}{\textbf{Introduction to Visualization}}
\end{itemize}

\hypertarget{learning-lab-preparation-2}{%
\subsubsection*{Learning Lab Preparation}\label{learning-lab-preparation-2}}
\addcontentsline{toc}{subsubsection}{Learning Lab Preparation}

The subject of our focus for this Topic has been on the power of language in influencing how we see the world. Our discussion during Learning Lab this week will focus on the importance of language in the Bible (and other religious writings) and how it ``shapes'' how we see the world.

To prepare for this discussion, consider the following prompt:

\textbf{\emph{Give some examples of the importance of language in the Bible or other religious writings.}}
\end{reflect}

\hypertarget{animal-language}{%
\section{Animal Language}\label{animal-language}}

\hypertarget{language}{%
\subsection*{Language}\label{language}}
\addcontentsline{toc}{subsection}{Language}

As a Christian I am pleased to see in psychology the resurgence of interest in studying some of what Ronald Koteskey (1980) calls humanity's ``God-like'' characteristics, creativity, imagery, and particularly language.

The use of words is extremely important in Christian scripture. God spoke the creation into existence; Jesus is called the Word; the significance of Babel and Pentecost are closely linked to the importance of language; and there is great power associated with an individual's name. In addition, Christians have usually considered the ability to communicate with words to be part of the image of God in man. However, recently several researchers claim to have taught animals, usually chimpanzees or apes, to communicate through language. Using sign language, blocks, or keyboards and computer- generated voices, the animals have signaled their needs and even generated word combinations.

But is this truly language? There is no doubt that the animals are using symbols as signs to stand for objects and actions. However, there are significant questions being raised about the comparison with human language.

Christians need to think carefully about what they mean when they talk about the image of God in man. The area of human learning and psycholinguistics offers some intriguing questions for thoughtful Christians. What is the origin of human speech-is it learned (as Skinner would say) or largely innate (as Chomsky would say)? Is human speech unique? Do the studies of language in animals necessitate a redefinition of the uniqueness of man? (Based on Psychology and Christianity by Ron Philipchalk, p.~102)

\hypertarget{resources-online-articles-of-interest-1}{%
\subsubsection*{RESOURCES: Online Articles of Interest}\label{resources-online-articles-of-interest-1}}
\addcontentsline{toc}{subsubsection}{RESOURCES: Online Articles of Interest}

To add to our exploration of this topic, take a moment to read the following articles:

\begin{itemize}
\item
  \href{http://tuvalu.santafe.edu/~johnson/articles.chimp.html}{\textbf{Chimp Talk Debate: Is It Really Language?}}
\item
  \href{https://www.massey.ac.nz/~alock/hbook/ristau.htm}{\textbf{Animal Language and Cognition Projects}}
\end{itemize}

\hypertarget{learning-lab-preparation-3}{%
\subsubsection*{Learning Lab Preparation}\label{learning-lab-preparation-3}}
\addcontentsline{toc}{subsubsection}{Learning Lab Preparation}

\begin{reflect}
Finally, take a moment to consider the following questions:

\textbf{\emph{How important for our understanding of who we are as humans is the distinctiveness of our language ability? Is it a sign of the image of God?}}

Be prepared to share your thoughts during discussion at your Learning Lab.
\end{reflect}

\hypertarget{assessment}{%
\section*{Assessment}\label{assessment}}
\addcontentsline{toc}{section}{Assessment}

\begin{assessment}
While there is no ``formal'' assignment that you will be responsible for submitting for Unit 1, you will be expected to participate in discussion during your Learning Lab. Your facilitator will be providing a participation mark based on your contributions. Below is some information to consider prior to attending your Learning Lab:

\emph{Active participation in group exercises, reflection, and critical discourse is an essential component of this course. You are expected to show respect for all members of the course, both in your speech and actions. Contribute by actively observing and listening, raising thoughtful questions, examining relevant issues, building on others' ideas, analyzing and evaluating the group's thinking, synthesizing key points, and expanding the group's perspectives. Take care not to dominate a conversation, giving space for others to speak. When in small groups help maintain the focus, flow, and quality of conversations, and take the initiative to invite others (particularly those who are quiet) to speak.}

\textbf{Rubric for Participation in Learning Labs}

\begin{longtable}[]{@{}
  >{\raggedright\arraybackslash}p{(\columnwidth - 4\tabcolsep) * \real{0.3333}}
  >{\raggedright\arraybackslash}p{(\columnwidth - 4\tabcolsep) * \real{0.3333}}
  >{\raggedright\arraybackslash}p{(\columnwidth - 4\tabcolsep) * \real{0.3333}}@{}}
\toprule\noalign{}
\begin{minipage}[b]{\linewidth}\raggedright
Emerging (0-64\%)
\end{minipage} & \begin{minipage}[b]{\linewidth}\raggedright
Developing (65-89\%)
\end{minipage} & \begin{minipage}[b]{\linewidth}\raggedright
Mastering (90-100\%)
\end{minipage} \\
\midrule\noalign{}
\endhead
\bottomrule\noalign{}
\endlastfoot
Never to almost never: Demonstrates active listening (as indicated by disengaged body language and no to rare comments that build on others' remarks),Initiates any contributions in class or small groups, Makes insightful or constructive comments, Helps maintain a supportive space for others to speak. & Sometimes to fairly often: Demonstrates active listening (as indicated by somewhat to often engaged body language and comments that build on others' remarks), Initiates a contribution at least once in a class or small group discussion; Makes insightful or constructive comments, Helps maintain a supportive space for others to speak. & Very often to nearly always: Demonstrates active listening (as indicated by fully engaged body language and comments that build on others' remarks), Initiates more than one contribution in a class or small group discussion, Makes insightful or constructive comments, Creates a space for others to speak and takes initiative to include others. \\
\end{longtable}
\end{assessment}

\hypertarget{checking-your-learning}{%
\section*{Checking your Learning}\label{checking-your-learning}}
\addcontentsline{toc}{section}{Checking your Learning}

\begin{progress}
Before you move on to the next unit, check that you are able to:

\begin{itemize}
\tightlist
\item
  Define key terminology related to principles of scientific research, research designs, and statistics.
\item
  Explain the five characteristics of quality scientific research, and the pros and cons of descriptive, correlational, and experimental research designs.
\item
  Determine how biases might influence the outcome of a study and how experiments help demonstrate cause-and-effect relationships.
\item
  Apply the concepts of reliability and validity to examples and concepts of experimental methods to research examples.
\item
  Assess whether anecdotes, authority figures, and common sense are reliably truthful sources of information.
\item
  Understand what it means for variables to be positively or negatively correlated and how and why psychologists use significance tests.
\end{itemize}
\end{progress}

\hypertarget{intelligence-testing}{%
\chapter{Intelligence Testing}\label{intelligence-testing}}

\hypertarget{overview-1}{%
\section*{Overview}\label{overview-1}}
\addcontentsline{toc}{section}{Overview}

In this unit, you will learn about techniques and tools for measuring intelligence, different theories as to what constitutes intelligence, and the biological, environmental, and behavioural factors that influence intelligence.

\hypertarget{topics-1}{%
\subsection*{Topics}\label{topics-1}}
\addcontentsline{toc}{subsection}{Topics}

This unit is divided into the following topics:

\begin{enumerate}
\def\labelenumi{\arabic{enumi}.}
\tightlist
\item
  What is Intelligence?
\item
  Extremes of Intelligence
\item
  Nature-Nature and IQ
\end{enumerate}

\hypertarget{learning-outcomes-1}{%
\subsection*{Learning Outcomes}\label{learning-outcomes-1}}
\addcontentsline{toc}{subsection}{Learning Outcomes}

By the end of this unit, students will be able to:

\begin{itemize}
\tightlist
\item
  Know and define the key terminology associated with understanding intelligence, intelligence testing, and heredity, environment, and intelligence.\\
\item
  Understand the reasoning behind the eugenics movements and its use of intelligence tests, why intelligence is divided into fluid and crystallized types, and the genetic basis of intelligence.\\
\item
  Apply the concepts of entity theory and incremental theory to help kids succeed in school, to identify examples from the triarchic and multiple theories of intelligence, and to recognize environmental and behavioural effects on intelligence to understand how to enhance your own cognitive abilities.\\
\item
  Analyze why it is difficult to remove all cultural bias from intelligence testing and whether teachers should spend time tailoring lessons to each individual student's learning style.
\end{itemize}

\hypertarget{activity-checklist-1}{%
\subsection*{Activity Checklist}\label{activity-checklist-1}}
\addcontentsline{toc}{subsection}{Activity Checklist}

\begin{reflect}
Here is a checklist of learning activities you will benefit from in completing this unit. You may find it useful for planning your work.

\textbf{Read and Reflect}

\begin{itemize}
\tightlist
\item
  Read \emph{Krause et al.~(2021). Revel for An Introduction to Psychological Science, 3rd Canadian Edition}\\
\item
  Review \textbf{Unit 2 - Slides}
\end{itemize}

CLICK HERE

An Introduction to Psychological Science - Chapter 9:Intelligence Testing

\begin{itemize}
\tightlist
\item
  Chapter 9 -- Intelligence Testing

  \begin{itemize}
  \tightlist
  \item
    Biblical Word Study Related to Ch. 8
  \end{itemize}
\end{itemize}

Proverbs 2:6

\begin{itemize}
\item
  \textsuperscript{6} For the LORD gives wisdom, and from his mouth come knowledge and understanding.
\end{itemize}

Greek Word Study

\begin{itemize}
\tightlist
\item
  Wisdom

  \begin{itemize}
  \tightlist
  \item
    Gr. \(\Sigma\omicron\phi\iota\alpha\) (Sophia) (Noun) -- wisdom in general, knowledge; ability; practical wisdom, prudence; learning, science; scientific skill; professed wisdom, human philosophy, superior knowledge and enlightenment; in N.T. divine wisdom, Christian enlightenment\\
  \end{itemize}
\item
  Wise

  \begin{itemize}
  \tightlist
  \item
    Gr. \(\Sigma\omicron\phi\omicron\varsigma\) (Sophos) (Adjective) -- wise generally; shrewd, clever; learned, intelligent; in N.T. divinely instructed; furnished with Christian wisdom, spiritually enlightened
  \end{itemize}
\item
  Intelligence

  \begin{itemize}
  \tightlist
  \item
    Gr.\(\Sigma\upsilon\nu\eta\sigma\iota\varsigma\) (Noun) -- pr. a sending together, a junction, as of streams; met. understanding, intelligence, discernment; the understanding, intellect, mind\\
  \end{itemize}
\item
  Intelligent

  \begin{itemize}
  \tightlist
  \item
    Gr. \(\Sigma\upsilon\nu\eta\tau\omicron\sigma\) (Synetos) (Adj.) -- intelligent, discerning, wise, prudent (cautious; worldly wise; exercising sound judgment)
  \end{itemize}
\end{itemize}

The Psychology of Wisdom

\begin{itemize}
\tightlist
\item
  Difficult life dilemmas:

  \begin{itemize}
  \tightlist
  \item
    ``A 15yearold girl wants to get married right away. What should one/she do and consider?''
  \end{itemize}
\item
  Or, ``Imagine a good friend of yours calls you up and tells you that she can't go on anymore and has decided to commit suicide. What would one/you be thinking about? How would one deal with this situation?''\\
\item
  Or, ``A 60 year old widow has recently completed a college degree and opened a business, only to learn that her son has been left alone with two small children to care for. What should she do?''
\end{itemize}

\includegraphics{assets/unit_2/slide_9.png}

\emph{Modules}

9.1 Learning Objectives

\begin{itemize}
\tightlist
\item
  Know the key terminology associated with intelligence and intelligence testing.\\
\item
  Understand the reasoning behind the eugenics movements and its use of intelligence tests.\\
\item
  Apply the concepts of entity theory and incremental theory to help kids succeed in school.\\
\item
  Analyze why it is difficult to remove all cultural bias from intelligence testing.
\end{itemize}

Different Approaches to Intelligence Testing (1 of 2)

\begin{itemize}
\tightlist
\item
  Sir Francis Galton

  \begin{itemize}
  \tightlist
  \item
    Anthropometrics (p.~329)\\
  \end{itemize}
\item
  Alfred Binet

  \begin{itemize}
  \tightlist
  \item
    Intelligence (p.~330)\\
  \item
    Mental age (p.~330)\\
  \end{itemize}
\item
  Lewis Terman

  \begin{itemize}
  \tightlist
  \item
    Stanford-Binet Test (p.~330)\\
  \end{itemize}
\item
  William Stern

  \begin{itemize}
  \tightlist
  \item
    Intelligence Quotient (IQ) (p.~330)
  \end{itemize}
\end{itemize}

\includegraphics{assets/unit_2/slide_12.png}

\emph{Slide showing - The Normal Distribution of Scores for a Standardized Intelligence Test}

\includegraphics{assets/unit_2/slide_13.png}

\emph{Slide showing - Subscales of the Wechsler Adult Intelligence Scale}

\includegraphics{assets/unit_2/slide_14.png}

\emph{Slide showing - The Wechsler Adult Intelligence Scale}

\includegraphics{assets/unit_2/slide_15.png}

\emph{Slide showing - Sample Problem from Raven's Progressive Matrices}

The Checkered Past of Intelligence Testing

\begin{itemize}
\tightlist
\item
  IQ Testing and Eugenics

  \begin{itemize}
  \tightlist
  \item
    Historical context\\
  \item
    Social Darwinism\\
  \item
    Eugenics
  \end{itemize}
\end{itemize}

The Race and IQ Controversy

\begin{itemize}
\tightlist
\item
  Racial differences in IQ scores Problems with the racial superiority interpretation

  \begin{itemize}
  \tightlist
  \item
    Culturally biased test content\\
  \item
    Culturally biased test process\\
  \item
    Stereotype threat (p.~336)
  \end{itemize}
\end{itemize}

Working the Scientific Literacy Model: Beliefs About Intelligence (1 of 3)

\begin{itemize}
\tightlist
\item
  What do we know about the kinds of beliefs that may affect test scores?

  \begin{itemize}
  \tightlist
  \item
    Entity theory (p.~336)\\
  \item
    Incremental theory (p.~336)\\
  \end{itemize}
\item
  How can science test whether beliefs affect performance?
\end{itemize}

\includegraphics{assets/unit_2/slide_19.png}

\emph{Slide showing - Personal Beliefs Influence Grades}

Working the Scientific Literacy Model: Beliefs About Intelligence (3 of 3)

\begin{itemize}
\tightlist
\item
  Can we critically evaluate this research?\\
\item
  Why is this relevant?
\end{itemize}

9.2 Learning Objectives

\begin{itemize}
\tightlist
\item
  Know the key terminology related to understanding intelligence.\\
\item
  Understand why intelligence is described as a hierarchy.\\
\item
  Understand intelligence differences between males and females.\\
\item
  Apply your knowledge to identify examples that reflect fluid vs.~crystallized intelligence.\\
\item
  Analyze whether teachers should spend time tailoring lessons to each individual student's learning style.
\end{itemize}

Intelligence as a Single, General Ability (1 of 3)

\begin{itemize}
\tightlist
\item
  Spearman's general intelligence

  \begin{itemize}
  \tightlist
  \item
    General intelligence factor, ``g'' (p.~340)
  \end{itemize}
\end{itemize}

\includegraphics{assets/unit_2/slide_23.png}

\emph{Slide showing - General Intelligence Is Related to Many Different Life Outcomes}

\includegraphics{assets/unit_2/slide_24.png}

\emph{Slide showing - General Intelligence Is Related to Many Different Life Outcomes}

Spearman's General Intelligence

\begin{itemize}
\tightlist
\item
  Does ``g'' tell us the whole story?
\end{itemize}

Intelligence as Multiple, Specific Abilities

\begin{itemize}
\tightlist
\item
  Spearman

  \begin{itemize}
  \tightlist
  \item
    Two factors: ``g'' and ``s''\\
  \end{itemize}
\item
  Thurstone

  \begin{itemize}
  \tightlist
  \item
    7 primary mental abilities\\
  \end{itemize}
\item
  Hierarchical model of intelligence

  \begin{itemize}
  \tightlist
  \item
    Nesting
  \end{itemize}
\end{itemize}

\includegraphics{assets/unit_2/slide_26.png}

\emph{Slide showing - Fluid and Crystallized Intelligence}

\includegraphics{assets/unit_2/slide_27.png}

\emph{Slide showing - Measuring Fluid Intelligence}

\includegraphics{assets/unit_2/slide_28.png}

\emph{Slide showing - Measuring Crystallized Intelligence}

Working the Scientific Literacy Model: Testing for Fluid and Crystallized Intelligence (4 of 4)

\begin{itemize}
\tightlist
\item
  Can we critically evaluate crystalized and fluid intelligence?

  \begin{itemize}
  \tightlist
  \item
    Gf is a blend of several different cognitive abilities\\
  \item
    Gf and Gc are not entirely separable
  \end{itemize}
\end{itemize}

Why is this relevant?

\begin{itemize}
\tightlist
\item
  Stereotypes related to intelligence and age
\end{itemize}

\includegraphics{assets/unit_2/slide_30.png}

\emph{Slide showing - Gardner's Proposed Forms of Intelligence}

\includegraphics{assets/unit_2/slide_31.png}

\emph{Slide showing - Gardner's Proposed Forms of Intelligence}

Myths in Mind: Learning Styles

\begin{itemize}
\tightlist
\item
  Visual learners should learn more with visual materials?

  \begin{itemize}
  \tightlist
  \item
    Lack of supporting evidence\\
  \end{itemize}
\item
  Focus on learning the meaning
\end{itemize}

PSYCH @ The NHL

\begin{itemize}
\tightlist
\item
  Head injuries in the NHL\\
\item
  Chronic traumatic encephalopathy\\
\item
  ImPACT

  \begin{itemize}
  \tightlist
  \item
    Regular testing checks for declines on specific abilities
  \end{itemize}
\end{itemize}

The Battle of the Sexes?

\begin{itemize}
\tightlist
\item
  Differences in intelligence?

  \begin{itemize}
  \tightlist
  \item
    No sex differences found\\
  \item
    Male scores have greater variability
  \end{itemize}
\item
  Do males and females have unique cognitive abilities?

  \begin{itemize}
  \tightlist
  \item
    Females: verbal, memory, emotions\\
  \item
    Males: visuospatial\\
  \item
    Stereotype threat
  \end{itemize}
\end{itemize}

9.3 Learning Objectives

\begin{itemize}
\tightlist
\item
  Know the key terminology related to heredity, environment, and intelligence.\\
\item
  Understand different approaches to studying the genetic basis of intelligence.\\
\item
  Apply your knowledge of environmental and behavioural effects on intelligence to understand how to enhance your own cognitive abilities.\\
\item
  Analyze the belief that older children are more intelligent than their younger siblings.
\end{itemize}

\includegraphics{assets/unit_2/slide_36.png}

\emph{Slide showing - Intelligence and Genetic Relatedness}

Working the Scientific Literacy Model: Brain Size and Intelligence (1 of 3)

\begin{itemize}
\tightlist
\item
  What do we know about brain size and intelligence?

  \begin{itemize}
  \tightlist
  \item
    Once believed brain size was related to intelligence

    \begin{itemize}
    \tightlist
    \item
      Contributed to prejudice
    \end{itemize}
  \end{itemize}
\end{itemize}

\includegraphics{assets/unit_2/slide_38.png}

\emph{Slide showing - Does Intelligence Increase with Brain Size?}

Working the Scientific Literacy Model: Brain Size and Intelligence (3 of 3)

\begin{itemize}
\tightlist
\item
  Can we critically evaluate the issue?

  \begin{itemize}
  \tightlist
  \item
    Which abilities underlie the correlation?\\
  \item
    Third-variable problem
  \end{itemize}
\item
  Why is this relevant?

  \begin{itemize}
  \tightlist
  \item
    Brain size and IQ used to understand clinical conditions

    \begin{itemize}
    \tightlist
    \item
      Prolonged anorexia nervosa and alcohol abuse
    \end{itemize}
  \end{itemize}
\end{itemize}

Environmental Influences on Intelligence

\begin{itemize}
\tightlist
\item
  Nutrition\\
\item
  Socioeconomic Status (SES)\\
\item
  Stress\\
\item
  Birth Order\\
\item
  Education
\end{itemize}

\includegraphics{assets/unit_2/slide_41.png}

\emph{Slide showing - The Flynn Effect}

Behavioural Influences on Intelligence

\begin{itemize}
\tightlist
\item
  Brain training programs\\
\item
  Nootropic drugs (p.~358)
\end{itemize}
\end{reflect}

\begin{caution}
\textbf{Note:} the slides are intended to supplement the information found in your textbook. If you are having trouble viewing them, they can also be downloaded by scrolling to the bottom of the screen and clicking on the ``Unit 2 - Slides'' link.*
\end{caution}

\begin{reflect}
\textbf{Intelligence Testing}

\begin{itemize}
\tightlist
\item
  Take some intelligence tests for yourself. As you complete them, consider how they might be considered beneficial, and controversial, as measures of intelligence.
\end{itemize}

\textbf{I.Q. Testing}

\begin{itemize}
\tightlist
\item
  Take some I.Q. Tests. Consider how valid and reliable these results are and think about how meaningful the results are.
\end{itemize}

\textbf{Ch. 9 Key Terms Quiz}

\begin{itemize}
\tightlist
\item
  Practice quiz to assess how well you know key terms from Chapter 9.\\
\item
  Not for formal evaluation.\\
\item
  Each topic will provide a question or scenario for you to consider prior to attending your Learning Lab. Be sure to carefully consider each prompt as you will be expected to contribute to the group discussion.
\end{itemize}
\end{reflect}

\hypertarget{resources-1}{%
\subsection*{Resources}\label{resources-1}}
\addcontentsline{toc}{subsection}{Resources}

Here are some additional resources that will help you complete this unit:

\begin{itemize}
\tightlist
\item
  Krause, M., Corts, D., Smith, S. C., \& Dolderman, D. (2018). \emph{Revel for An Introduction to Psychological Science, 2nd Canadian Edition.} Pearson Ed.\\
\item
  Other resources will be provided online.
\end{itemize}

\hypertarget{what-is-intelligence}{%
\section{What is Intelligence?}\label{what-is-intelligence}}

\hypertarget{intelligence}{%
\subsection*{Intelligence}\label{intelligence}}
\addcontentsline{toc}{subsection}{Intelligence}

As David Myers points out, intelligence is a slippery concept. We all have an idea of what it refers to, but we cannot agree on a single definition. Perhaps the most helpful advice is to remember, as Myers points out, that ``intelligence is a socially constructed concept. Cultures deem `intelligent' whatever attributes enable success in those cultures'' (2010). Historically, in North American culture, the idea of intelligence performance on an IQ test. The composition of these tests reflected North American culture's emphasis on particular mental abilities, specifically, those associated with success in an academic setting. More recently, we have come to realize that there are many kinds of intelligence. In this unit we will consider some varieties of intelligence, or multiple intelligences.

\hypertarget{types-of-intelligence}{%
\subsection*{Types of Intelligence}\label{types-of-intelligence}}
\addcontentsline{toc}{subsection}{Types of Intelligence}

\textbf{\emph{Emotional Intelligence}}

I have to admit that when I first heard of emotional intelligence (EI) I was skeptical. It sounded like popular psychology - someone trying to make a buck preying on our need for self-knowledge. However, upon further investigation I found that EI was linked to social intelligence (the ability to understand and relate to people), a concept developed by the pioneering psychologist E.L. Thorndike in 1920. And upon still further investigation, I found EI made a lot of sense.

\hypertarget{sternbergs-three-components-of-intelligence}{%
\section{Sternberg's Three Components of Intelligence}\label{sternbergs-three-components-of-intelligence}}

Robert Sternberg wanted to show that intelligence was more than just one general ability (known as \emph{g theory}). He believed our intelligence is best classified into three areas that predict real - world success: \textbf{\emph{analytical, creative, and practical.}} The following article does a great job explaining the Triarchic Theory of Intelligence and its three sub-theories. It also makes note of the criticisms that have been brought against this theory. \emph{To better understand Sternberg's Three Components of Intelligence follow the link below:}

\begin{itemize}
\tightlist
\item
  \href{https://www.thoughtco.com/triarchic-theory-of-intelligence-4172497}{\textbf{Three Components of Intelligence}}
\end{itemize}

\hypertarget{gardners-eight-types-of-intelligence}{%
\section{Gardner's Eight Types of Intelligence}\label{gardners-eight-types-of-intelligence}}

Howard Gardner put together a robust, research-based theory of Multiple Intelligences. He put forth an understanding of intelligence promoting our abilities are best classified into nine independent intelligences, which include a broad range of skills beyond traditional school smarts. This illuminating read that can help you understand what your primary intelligences are. Follow the link below:

\begin{itemize}
\tightlist
\item
  \href{https://www.institute4learning.com/resources/articles/multiple-intelligences/}{\textbf{Eight Types of Intelligence}}
\end{itemize}

\hypertarget{resources-online-articles-of-interest-2}{%
\subsection*{Resources: Online Articles of Interest}\label{resources-online-articles-of-interest-2}}
\addcontentsline{toc}{subsection}{Resources: Online Articles of Interest}

To add to your exploration of this topic, take a moment to read the following articles:

\begin{itemize}
\item
  \href{http://www.eiconsortium.org/}{\textbf{Emotional Intelligence Consortium Website}}
\item
  \href{http://eqi.org/}{\textbf{EQI Web Resource}}
\end{itemize}

\hypertarget{learning-activity-2}{%
\subsection*{Learning Activity}\label{learning-activity-2}}
\addcontentsline{toc}{subsection}{Learning Activity}

\begin{reflect}
\hypertarget{intelligence-testing-1}{%
\subsubsection*{Intelligence Testing}\label{intelligence-testing-1}}
\addcontentsline{toc}{subsubsection}{Intelligence Testing}

In this unit we investigated Intelligence Testing. One of the important concepts we learned was that Intelligence Testing can be controversial due to its socially constructed nature.

Below is a link to a website that will allow you to take some Intelligence Tests for yourself. As you work through each test, and see the results, think about why we might consider these types of tests beneficial, and why they might be considered controversial.

\begin{itemize}
\tightlist
\item
  \href{https://www.queendom.com/tests/}{\textbf{Emotional Intelligence Test}}
\end{itemize}

\hypertarget{learning-lab-preparation-4}{%
\subsubsection*{Learning Lab Preparation}\label{learning-lab-preparation-4}}
\addcontentsline{toc}{subsubsection}{Learning Lab Preparation}

Our Learning Lab for this unit will focus on Intelligence Testing. As we have seen, Intelligence Testing is best implemented after careful consideration - this will be the focus of our discussion during this unit's Learning Lab. To help you prepare, consider the following questions:

\begin{itemize}
\tightlist
\item
  \textbf{\emph{How do you feel about the EMOTIONAL INTELLIGENCE TESTS? Did you learn anything? If so, what were the important areas that were illumined by the test?}}\\
\item
  \textbf{\emph{Is `emotional intelligence' a valid concept worth measuring?}}\\
\item
  \textbf{\emph{Do you believe E-IQ is more important than ``intelligence'' as measured by IQ scores for success and happiness in your life?''}}
\end{itemize}
\end{reflect}

\hypertarget{extremes-of-intelligence}{%
\section{Extremes of Intelligence}\label{extremes-of-intelligence}}

\hypertarget{iq-testing}{%
\subsection*{IQ Testing}\label{iq-testing}}
\addcontentsline{toc}{subsection}{IQ Testing}

In this section we continue to build upon our understanding of intelligence and testing to focus on Intelligence tests and those who score at the ``extremes.'' Intelligence tests are the most common tool used to measure intelligence. Intelligence test are one method of assessing an individual's mental aptitudes and comparing them with those of others using numerical scores. These scores are then plotted on a normal (bell) curve to estimate where a person's intelligence rates in relation to a standardized population. The extremes of intelligence is the understanding that on one end of the continuum are those with intellectual disabilities and on the other end are those who are geniuses.

\hypertarget{resources-2}{%
\subsection*{Resources}\label{resources-2}}
\addcontentsline{toc}{subsection}{Resources}

To supplement our understanding of this topic, take a moment to explore the following resources:

\begin{itemize}
\item
  \href{https://thearc.org/}{\textbf{The Arc}}
\item
  \href{https://www.mensa.org}{\textbf{MENSA International}}
\end{itemize}

\hypertarget{learning-activity-3}{%
\subsection*{Learning Activity}\label{learning-activity-3}}
\addcontentsline{toc}{subsection}{Learning Activity}

\begin{reflect}
\hypertarget{iq-testing-1}{%
\subsubsection*{IQ Testing}\label{iq-testing-1}}
\addcontentsline{toc}{subsubsection}{IQ Testing}

After considering the above, and having read the textbook's discussion of intelligence testing, you might want to try some tests yourself. The value in doing this is not that you will get an accurate idea of your IQ score, but rather that you might get a better understanding for some of the problems in testing. As you try some of the tests at the following sites, remind yourself of the problems of test standardization, validity, and reliability. How well do you think these tests measure up?

\begin{itemize}
\tightlist
\item
  \href{https://www.queendom.com/tests/index.htm}{\textbf{Queendom IQ Test}}\\
\item
  \href{http://www.davideck.com/}{\textbf{davideck.com}}
\end{itemize}

\hypertarget{learning-lab-preparation-5}{%
\subsubsection*{Learning Lab Preparation}\label{learning-lab-preparation-5}}
\addcontentsline{toc}{subsubsection}{Learning Lab Preparation}

As we prepare for our Learning Lab this week, we consider intelligence from a Christian perspective. Read the following passage and carefully consider the questions below to help prepare for our discussion:

\emph{The Bible says, in James; Chapter 2:}

\emph{My brothers, as believers in our glorious Lord Jesus Christ, don't show favoritism. Suppose a man comes into your meeting wearing a gold ring and fine clothes, and a poor man in shabby clothes also comes in. If you show special attention to the man wearing fine clothes and say, ``Here's a good seat for you,'' but say to the poor man, ``You stand there'' or ``Sit on the floor by my feet,'' have you not discriminated among yourselves and become judges with evil thoughts?}

\begin{itemize}
\tightlist
\item
  \textbf{\emph{Could we be guilty of favoritism in esteeming more intelligent people above less intelligent people? In society? In Church?}}
\end{itemize}
\end{reflect}

\hypertarget{nature-nature-and-iq}{%
\section{Nature-Nature and IQ}\label{nature-nature-and-iq}}

While the debate rages over the relative contributions of nature and nurture to IQ, no one denies that heredity (nature) plays some role. The question arises then, ``So what?'' Are we going to try to control (and presumably increase) IQ through genetic engineering or some other method of eugenics (\emph{Eugenics is the search for hereditary factors that give people an evolutionary advantage; translated it can mean ``good genes'' or ``good origin''})? Are we going to control who should have children and how many, allowing the most intelligent parents to have more children and restricting the less intelligent? When the genetic basis for IQ (or some other component of intelligence) is identified, will parents select embryos with greater potential? For more on this topic see the following quote and the website from which it came.

``If we are concerned for the future of the (hopefully) millions of generations still to be born, we must realize that their fate lies to a considerable extent in the breeding practices of those who are currently alive.'' \emph{(Intelligence and Eugenics)}

\hypertarget{what-will-you-do}{%
\section{What Will You Do?}\label{what-will-you-do}}

If or when you have children, will you ban screens (TV, smart phone, tablet, laptop) as a ``brain rotter'' and read to them every day? Or will you just let nature take its course and allow both screens and reading?

\hypertarget{test-biases}{%
\section{Test Biases?}\label{test-biases}}

IQ tests are generally valid for their original purpose---as predictors of aca­demic performance. Controversy arises when IQ scores are taken to mean over­all intelligence and even overall worth. IQ scores consistently predict that some cultural and racial groups will do better at school than will other groups. These differ­ences are an indication of bias not in the IQ tests but in the back­grounds and academic settings that first create and then magnify differences.

\hypertarget{resources-3}{%
\subsection*{Resources}\label{resources-3}}
\addcontentsline{toc}{subsection}{Resources}

To supplement our understanding of this topic, take a moment to read through the following:

\begin{itemize}
\tightlist
\item
  \href{https://darkwing.uoregon.edu/~adoption/topics/naturenurturestudies.htm}{\textbf{The Adoption History Project}}\\
\item
  \href{http://unisci.com/stories/20012/0417014.htm}{\textbf{Nature-Nature and IQ}}
\end{itemize}

\hypertarget{learning-activity-4}{%
\subsection*{Learning Activity}\label{learning-activity-4}}
\addcontentsline{toc}{subsection}{Learning Activity}

\begin{reflect}
\hypertarget{ch.-9-key-terms-quiz}{%
\subsubsection*{Ch. 9 Key Terms Quiz}\label{ch.-9-key-terms-quiz}}
\addcontentsline{toc}{subsubsection}{Ch. 9 Key Terms Quiz}

In order to review some of the major terms from Chapter 9 in your textbook, take the following unmarked quiz. Although you will not be evaluated on these terms, they will assist you in the assessments for this course:

\hypertarget{learning-lab-preparation-6}{%
\subsubsection*{Learning Lab Preparation}\label{learning-lab-preparation-6}}
\addcontentsline{toc}{subsubsection}{Learning Lab Preparation}

Consider the following scenario (and questions) as you continue to prepare for this unit's Learning Lab. You will be asked to share your thoughts so come prepared

*If you suggest that Asians have darker skin than Caucasians, you are not considered racist; this is an obvious fact with a genetic basis. However, if you suggest that Asians are more intelligent than Caucasians (as IQ tests show), or that African Americans are less intelligent, watch out

\begin{itemize}
\tightlist
\item
  \textbf{\emph{What is different about these two claims that makes us accept one and not the other? Is it the role of nature versus nurture? Or is it more closely tied to the high value our culture places on intelligence, and especially IQ scores?}}\\
\item
  \textbf{\emph{If IQ were unimportant would it matter if one racial or gender group tended to score higher than another group? Would you be considered racist or sexist for suggesting this?}}
\end{itemize}
\end{reflect}

\hypertarget{assessment-1}{%
\section*{Assessment}\label{assessment-1}}
\addcontentsline{toc}{section}{Assessment}

\begin{assessment}
While there is no ``formal'' assignment that you will be responsible for submitting for Unit 2, you will be expected to participate in discussion during your Learning Lab. Your facilitator will be providing a participation mark based on your contributions. Below is some information to consider prior to attending your Learning Lab:

\emph{Active participation in group exercises, reflection, and critical discourse is an essential component of this course. You are expected to show respect for all members of the course, both in your speech and actions. Contribute by actively observing and listening, raising thoughtful questions, examining relevant issues, building on others' ideas, analyzing and evaluating the group's thinking, synthesizing key points, and expanding the group's perspectives. Take care not to dominate a conversation, giving space for others to speak. When in small groups help maintain the focus, flow, and quality of conversations, and take the initiative to invite others (particularly those who are quiet) to speak.}

\textbf{Rubric for Participation in Learning Labs}

\begin{longtable}[]{@{}
  >{\raggedright\arraybackslash}p{(\columnwidth - 4\tabcolsep) * \real{0.3333}}
  >{\raggedright\arraybackslash}p{(\columnwidth - 4\tabcolsep) * \real{0.3333}}
  >{\raggedright\arraybackslash}p{(\columnwidth - 4\tabcolsep) * \real{0.3333}}@{}}
\toprule\noalign{}
\begin{minipage}[b]{\linewidth}\raggedright
Emerging (0-64\%)
\end{minipage} & \begin{minipage}[b]{\linewidth}\raggedright
Developing (65-89\%)
\end{minipage} & \begin{minipage}[b]{\linewidth}\raggedright
Mastering (90-100\%)
\end{minipage} \\
\midrule\noalign{}
\endhead
\bottomrule\noalign{}
\endlastfoot
Never to almost never: Demonstrates active listening (as indicated by disengaged body language and no to rare comments that build on others' remarks), Initiates any contributions in class or small groups, Makes insightful or constructive comments, Helps maintain a supportive space for others to speak. & Sometimes to fairly often: Demonstrates active listening (as indicated by somewhat to often engaged body language and comments that build on others' remarks), Initiates a contribution at least once in a class or small group discussion; Makes insightful or constructive comments, Helps maintain a supportive space for others to speak. & Very often to nearly always: Demonstrates active listening (as indicated by fully engaged body language and comments that build on others' remarks), Initiates more than one contribution in a class or small group discussion, Makes insightful or constructive comments, Creates a space for others to speak and takes initiative to include others. \\
\end{longtable}
\end{assessment}

\hypertarget{checking-your-learning-1}{%
\section*{Checking your Learning}\label{checking-your-learning-1}}
\addcontentsline{toc}{section}{Checking your Learning}

\begin{progress}
Before you move on to the next unit, check that you are able to:

\begin{itemize}
\tightlist
\item
  Define the key terminology associated with understanding intelligence, intelligence testing, and heredity, environment, and intelligence.\\
\item
  Understand the reasoning behind the eugenics movements and its use of intelligence tests, why intelligence is divided into fluid and crystallized types, and the genetic basis of intelligence.\\
\item
  Apply the concepts of entity theory and incremental theory to help kids succeed in school, to identify examples from the triarchic and multiple theories of intelligence, and to recognize environmental and behavioral effects on intelligence to understand how to enhance your own cognitive abilities.\\
\item
  Analyze why it is difficult to remove all cultural bias from intelligence testing and whether teachers should spend time tailoring lessons to each individual student's learning style.
\end{itemize}
\end{progress}

\hypertarget{the-developing-person---part-1}{%
\chapter{The Developing Person - Part 1}\label{the-developing-person---part-1}}

\hypertarget{overview-2}{%
\section*{Overview}\label{overview-2}}
\addcontentsline{toc}{section}{Overview}

Now that we have covered over some broad and influential topics in psychology, we begin our focus on human development. The focus of the content for this unit, will be on Chapter 10 in your textbook. As you turn ahead, you will notice that Chapter 10 contains a large amount of information - because of this, we will be covering it in this unit, and the next (Unit 4).

In this Unit (Part 1), you will learn about various strategies for researching human development, normal and abnormal prenatal development, and various cognitive, physical, and social developmental factors for infancy, childhood, and adolescence.

\hypertarget{topics-2}{%
\subsection*{Topics}\label{topics-2}}
\addcontentsline{toc}{subsection}{Topics}

This unit is divided into the following topics:

\begin{enumerate}
\def\labelenumi{\arabic{enumi}.}
\tightlist
\item
  Prenatal Development\\
\item
  Infancy and Childhood\\
\item
  Adolescence
\end{enumerate}

\hypertarget{learning-outcomes-2}{%
\subsection*{Learning Outcomes}\label{learning-outcomes-2}}
\addcontentsline{toc}{subsection}{Learning Outcomes}

By the end of this unit, student's will be able to:

\begin{itemize}
\tightlist
\item
  Define the key terminology related to prenatal and infant physical development, infancy and childhood, and adolescent development.\\
\item
  Understand advantages and disadvantages to different research designs in developmental psychology.\\
\item
  Understand the cognitive changes that occur during infancy and childhood, and the importance of attachment and the different styles of attachment.\\
\item
  Understand the process of identity formation, relationships, and moral emotions during adolescence.\\
\item
  Apply your understanding to identify the best ways expectant parents can ensure the health of their developing fetus, how to promote learning, and how to categorize moral reasoning.\\
\item
  Analyze the effects of preterm birth, how to effectively discipline children, and adolescent judgment and risk taking.
\end{itemize}

\hypertarget{activity-checklist-2}{%
\subsection*{Activity Checklist}\label{activity-checklist-2}}
\addcontentsline{toc}{subsection}{Activity Checklist}

\begin{reflect}
Here is a checklist of learning activities you will benefit from in completing this unit. You may find it useful for planning your work:

\textbf{Read and Reflect}

\begin{itemize}
\tightlist
\item
  Read \emph{Krause et al.~(2021). Revel for An Introduction to Psychological Science, 3rd Canadian Edition}\\
\item
  Continue our study of development - in particular, some of the major changes adolescents go through in their growth.\\
\item
  Review \emph{Unit 3 - Slides}
\end{itemize}

CLICK HERE

An Introduction to Psychological Science - Chapter 10:Lifespan Development

Learning Objectives

\textsuperscript{12} The righteous will flourish like a palm tree, they will grow like a cedar of Lebanon;\\
\textsuperscript{13} planted in the house of the Lord, they will flourish in the courts of our God.\\
\textsuperscript{14} They will still bear fruit in old age,they will stay fresh and green,\\
\textsuperscript{15} proclaiming, ``The Lord is upright; he is my Rock, and there is no wickedness in him.''

\begin{itemize}
\tightlist
\item
  Video: \href{http://www.ted.com/talks/annie_murphy_paul_what_we_learn_before_we_re_born.html}{Annie Murphy Paul: What we learn before we're born}
\end{itemize}

\includegraphics{assets/unit_3/slide_3.png}

\emph{Slide showing - Introduction}

Modules

\begin{itemize}
\tightlist
\item
  Physical Development from Conception through Infancy\\
\item
  Infancy and Childhood: Cognitive and Emotional Development\\
\item
  Adolescence\\
\item
  Adulthood and Aging
\end{itemize}

Learning Objectivess

\begin{itemize}
\tightlist
\item
  Know the key terminology related to prenatal and infant physical development.\\
\item
  Understand pros and cons to different research designs in developmental psychology.\\
\item
  Apply your understanding to identify the best ways expectant parents can ensure the health of their developing fetus.\\
\item
  Analyze the effects of preterm birth.
\end{itemize}

Developmental Psychology

\begin{itemize}
\tightlist
\item
  Developmental psychology (p.~362)

  \begin{itemize}
  \tightlist
  \item
    Early development influences later behaviours
  \end{itemize}
\end{itemize}

\includegraphics{assets/unit_3/slide_7.png}

\emph{Slide showing - Cross-Sectional and Longitudinal Methods}

Patterns of Development: Stages and Continuity

\begin{itemize}
\tightlist
\item
  Stages

  \begin{itemize}
  \tightlist
  \item
    Abrupt transitions\\
  \end{itemize}
\item
  Continuous

  \begin{itemize}
  \tightlist
  \item
    Slow changes
  \end{itemize}
\end{itemize}

\includegraphics{assets/unit_3/slide_9.png}

\emph{Slide showing - Phases of Prenatal Development}

\includegraphics{assets/unit_3/slide_10.png}

\emph{Slide showing - Fetal Brain Development}

Nutrition, Teratogens, and Fetal Development

\begin{itemize}
\tightlist
\item
  Teratogen (p.~366)

  \begin{itemize}
  \tightlist
  \item
    Alcohol\\
  \item
    Cigarettes\\
  \end{itemize}
\item
  Fetal Alcohol Syndrome (p.~366)

  \begin{itemize}
  \tightlist
  \item
    1.5 in 1000 worldwide

    \begin{itemize}
    \tightlist
    \item
      Likely higher\\
    \end{itemize}
  \end{itemize}
\item
  Stress
\end{itemize}

Working the Scientific Literacy Model: The Long-Term Effects of Premature Birth (1 of 2)

\begin{itemize}
\tightlist
\item
  What do we know about premature birth?

  \begin{itemize}
  \tightlist
  \item
    Preterm infants (p.~368)\\
  \item
    25 weeks: 50\% survival\\
  \item
    30 weeks: 95\% survival\\
  \end{itemize}
\item
  How can science be used to help preterm infants?

  \begin{itemize}
  \tightlist
  \item
    NIDCAP
  \end{itemize}
\end{itemize}

Working the Scientific Literacy Model: The Long-Term Effects of Premature Birth (2 of 2)

\begin{itemize}
\tightlist
\item
  Can we critically evaluate this research?

  \begin{itemize}
  \tightlist
  \item
    Small sample size\\
  \item
    Why does the program work?\\
  \end{itemize}
\item
  Why is this relevant?

  \begin{itemize}
  \tightlist
  \item
    9\% of infants are born preterm\\
  \item
    Simple interventions available:

    \begin{itemize}
    \tightlist
    \item
      Massage\\
    \item
      Kangaroo care
    \end{itemize}
  \end{itemize}
\end{itemize}

Myths in Mind: Vaccinations and Autism

\begin{itemize}
\tightlist
\item
  1990 claim that MMR vaccine linked to autism

  \begin{itemize}
  \tightlist
  \item
    One dose given at year 1\\
  \item
    Second does before starting school\\
  \item
    Many parents refused\\
  \end{itemize}
\item
  Lack of scientific support

  \begin{itemize}
  \tightlist
  \item
    Article retracted 2010
  \end{itemize}
\end{itemize}

Sensory Development in Infancy

\begin{itemize}
\tightlist
\item
  Sensory before birth

  \begin{itemize}
  \tightlist
  \item
    4 months gestation, brain receiving signals from eyes and ears\\
  \item
    7-8 months gestation, fetus actively listening\\
  \end{itemize}
\item
  Vision at birth

  \begin{itemize}
  \tightlist
  \item
    30 cm or less\\
  \item
    20/20 by 12 months\\
  \end{itemize}
\item
  Smell at birth

  \begin{itemize}
  \tightlist
  \item
    Cringe at foul odours\\
  \item
    Discriminate mother's breastmilk
  \end{itemize}
\end{itemize}

\includegraphics{assets/unit_3/slide_16.png}

\emph{Slide showing - Experimental Stimuli for Studying Visual Habituation in Infants}

\includegraphics{assets/unit_3/slide_17.png}

\emph{Slide showing - The Visual Cliff}

\includegraphics{assets/unit_3/slide_18.png}

\emph{Slide showing - A Few Key Infant Reflexes}

\includegraphics{assets/unit_3/slide_19.png}

\emph{Slide showing - Motor Skills Develop in Stages}

\includegraphics{assets/unit_3/slide_20.png}

\emph{Slide showing - The Processes of Synaptic Pruning}

Learning Objectives

\begin{itemize}
\tightlist
\item
  Know the key terminology associated with infancy and childhood.\\
\item
  Understand the cognitive changes that occur during infancy and childhood.\\
\item
  Understand the importance of attachment and the different styles of attachment.\\
\item
  Apply the concept of scaffolding and the zone of proximal development to understand how to best promote learning.\\
\item
  Analyze how to effectively discipline children in order to promote moral behaviour.
\end{itemize}

The Importance of Sensitive Periods

\begin{itemize}
\tightlist
\item
  Sensitive period (p.~375)

  \begin{itemize}
  \tightlist
  \item
    Language fluency\\
  \item
    Perception\\
  \item
    Balance\\
  \item
    Recognition of parents\\
  \item
    Identifying with a particular culture
  \end{itemize}
\end{itemize}

\includegraphics{assets/unit_3/slide_23.png}

\emph{Slide showing - Piaget's Stages of Cognitive Development}

The Sensorimotor Stage: Objects and the Physical World

\begin{itemize}
\tightlist
\item
  Sensorimotor stage (p.~375)

  \begin{itemize}
  \tightlist
  \item
    Birth to 2 years\\
  \item
    Object permanence (p.~376)

    \begin{itemize}
    \tightlist
    \item
      Hidden toy test
    \end{itemize}
  \end{itemize}
\end{itemize}

\includegraphics{assets/unit_3/slide_25.png}

\emph{Slide showing - Testing Conservation}

\includegraphics{assets/unit_3/slide_26.png}

\emph{Slide showing - Scale Errors and Testing for Scale Model Comprehension}

The Concrete Operational Stage: Using Logical Thought

\begin{itemize}
\tightlist
\item
  Concrete operational stage (p.~377)

  \begin{itemize}
  \tightlist
  \item
    7 to 11 years\\
  \item
    Transitivity
  \end{itemize}
\end{itemize}

The Formal Operational Stage: Abstract and Hypothetical Thought

\begin{itemize}
\tightlist
\item
  Formal operational stage (p.~378)

  \begin{itemize}
  \tightlist
  \item
    11 years to adulthood\\
  \item
    Scientific thinking
  \end{itemize}
\end{itemize}

Working the Scientific Literacy Model: Evaluating Piaget (1 of 3)

\begin{itemize}
\tightlist
\item
  What do we know about cognitive abilities in infants?

  \begin{itemize}
  \tightlist
  \item
    Core knowledge hypothesis (p.~378)\\
  \item
    Habituation (p.~378)\\
  \item
    Dishabituation (p.~378)
  \end{itemize}
\end{itemize}

\includegraphics{assets/unit_3/slide_31.png}

\emph{Slide showing - Testing Infants' Understanding of Quantity}

Complementary Approaches to Piaget

\begin{itemize}
\tightlist
\item
  Vygotsky

  \begin{itemize}
  \tightlist
  \item
    Zone of proximal development (p.~380)

    \begin{itemize}
    \tightlist
    \item
      Scaffolding (p.~380)

      \begin{itemize}
      \tightlist
      \item
        Cultural differences
      \end{itemize}
    \end{itemize}
  \end{itemize}
\end{itemize}

Social Development and Attachment (1 of 2)

\begin{itemize}
\tightlist
\item
  Attachment (p.~381)\\
\item
  Harry Harlow's monkey experiments\\
\item
  Strange situation test (p.382)
\end{itemize}

\includegraphics{assets/unit_3/slide_34.png}

\emph{Slide showing - The Strange Situation}

Social Development and Attachment (1 of 2)

\begin{itemize}
\tightlist
\item
  Types of Attachment\\
\item
  Secure attachment

  \begin{itemize}
  \tightlist
  \item
    Insecure attachment\\
  \item
    Disorganized\\
  \item
    Anxious/Ambivalent\\
  \item
    Avoidant
  \end{itemize}
\end{itemize}

Social Development and Attachment (1 of 2)

\begin{itemize}
\tightlist
\item
  Parenting and Attachment\\
\item
  Attachment behavioural system (p.~383)\\
\item
  Caregiving behavioural system (p.~383)\\
\item
  Conditional approaches\\
\item
  Introjection (p.~383)\\
\item
  Inductive discipline (p.~383)
\end{itemize}

Self-Awareness (1 of 3)

\begin{itemize}
\tightlist
\item
  Self-awareness (p.~384)

  \begin{itemize}
  \tightlist
  \item
    Reflection in mirror\\
  \end{itemize}
\item
  Egocentric (p.~384)
\end{itemize}

\includegraphics{assets/unit_3/slide_38.png}

\emph{Slide showing - Piaget's Test for Egocentric Perspective in Children}

Self-Awareness (3 of 3)

\begin{itemize}
\tightlist
\item
  Theory of mind (p.~384)\\
\item
  False-belief task
\end{itemize}

Psychosocial Development Across the Lifespan (1 of 2)

\begin{itemize}
\tightlist
\item
  Infancy

  \begin{itemize}
  \tightlist
  \item
    Sense of security\\
  \end{itemize}
\item
  Toddlerhood

  \begin{itemize}
  \tightlist
  \item
    Exploring autonomy\\
  \end{itemize}
\item
  Early Childhood

  \begin{itemize}
  \tightlist
  \item
    Pushing boundaries and experimenting\\
  \end{itemize}
\item
  Childhood

  \begin{itemize}
  \tightlist
  \item
    Active engagement
  \end{itemize}
\end{itemize}

Learning Objectives

\begin{itemize}
\tightlist
\item
  Know the key terminology concerning adolescent development.\\
\item
  Understand the process of identity formation during adolescence.\\
\item
  Understand the importance of relationships in adolescence.\\
\item
  Understand the functions of moral emotions.\\
\item
  Apply your understanding of the categories of moral reasoning.\\
\item
  Analyze the relationship between brain development and adolescent judgment and risk taking.
\end{itemize}

\includegraphics{assets/unit_3/slide_42.png}

\emph{Slide showing - Physical Changes That Accompany Puberty in Male and Female Adolescents}

Emotional Challenges in Adolescence

\begin{itemize}
\tightlist
\item
  Intense and volatile emotions\\
\item
  Cognitive reframing\\
\item
  Ability to delay gratification (p.~392)
\end{itemize}

Working the Scientific Literacy Model: Adolescent Risk and Decision Making (1 of 4)

\begin{itemize}
\tightlist
\item
  What do we know about adolescence and decision making?

  \begin{itemize}
  \tightlist
  \item
    Ongoing changes in prefrontal cortex

    \begin{itemize}
    \tightlist
    \item
      Region involved in impulse control, mood, planning, organizing, and reasoning
    \end{itemize}
  \end{itemize}
\end{itemize}

\includegraphics{assets/unit_3/slide_45.png}

\emph{Slide showing - Extended Brain Development}

Working the Scientific Literacy Model: Adolescent Risk and Decision Making (3 of 4)

\begin{itemize}
\tightlist
\item
  Can we critically evaluate this explanation for risky decision making?

  \begin{itemize}
  \tightlist
  \item
    Still capable of making good decisions\\
  \item
    Temperament and personality\\
  \item
    Situational factors\\
  \end{itemize}
\item
  Why is this relevant?

  \begin{itemize}
  \tightlist
  \item
    Major public health problem
  \end{itemize}
\end{itemize}

\includegraphics{assets/unit_3/slide_47.png}

\emph{Slide showing - What Drives Teenagers to Take Risks?}

Cognitive Development: Moral Reasoning vs.~Emotions

\begin{itemize}
\tightlist
\item
  Formal operational stage

  \begin{itemize}
  \tightlist
  \item
    Abstract thinking\\
  \item
    Scientific thinking\\
  \item
    Perspective taking
  \end{itemize}
\end{itemize}

Kohlberg's Moral Development: Learning Right from Wrong (1 of 2)

\begin{itemize}
\tightlist
\item
  A trolley is hurtling down the tracks toward a group of five unsuspecting people. You are standing next to a lever that, if pulled, would direct the trolley onto another track, thereby saving the five individuals. However, on the second track stands a single, unsuspecting person, who would be struck by the diverted trolley.
\end{itemize}

\includegraphics{assets/unit_3/slide_50.png}

\emph{Slide showing - Kohlberg's Stages of Moral Reasoning}

Moral Development

\begin{itemize}
\tightlist
\item
  Social intuitionist model

  \begin{itemize}
  \tightlist
  \item
    Julie and Steven are brother and sister. They are travelling together in France on summer vacation from college. One night they are staying alone in a cabin near the beach. They decide that it would be interesting and fun if they shared a romantic kiss. At the very least it would be a new experience for each of them. They both enjoy the experience, but they decide not to do it again. They keep that night as a special secret, which makes them feel even closer to each other.
  \end{itemize}
\end{itemize}

Social Development: Identity and Relationships

\begin{itemize}
\tightlist
\item
  Identity (p.~395)

  \begin{itemize}
  \tightlist
  \item
    Personal qualities\\
  \item
    Social qualities\\
  \item
    Future goals\\
  \end{itemize}
\item
  Adolescence identity crisis

  \begin{itemize}
  \tightlist
  \item
    Curiosity, questioning, and exploration\\
  \end{itemize}
\item
  Peer groups\\
\item
  Romantic relationships
\end{itemize}

Learning Objectives

\begin{itemize}
\tightlist
\item
  Know the key terminology concerning adulthood and aging.\\
\item
  Know the key areas of growth experiences by emerging adults.\\
\item
  Understand age-related disorders such as Alzheimer's disease.\\
\item
  Understand how cognitive abilities change with age.\\
\item
  Apply your attitudes about marriage.\\
\item
  Analyze the stereotype that old age is a time of unhappiness.
\end{itemize}

Physical Changes in Adulthood

\begin{itemize}
\tightlist
\item
  Age brackets:

  \begin{itemize}
  \tightlist
  \item
    Young adulthood: 18-40 years\\
  \item
    Middle adulthood: 40-65 years\\
  \item
    Older adulthood: 65 years and onward\\
  \end{itemize}
\item
  Menopause (p.~399)
\end{itemize}

Psychosocial Development Across the Lifespan (2 of 2)

\begin{itemize}
\tightlist
\item
  Ages 25 to 40

  \begin{itemize}
  \tightlist
  \item
    Separate from parents
  \item
    Work on intimate relationships
  \item
    Failure can result in isolation
  \end{itemize}
\item
  Ages 45 to 65
\item
  Producing something of value
\item
  Work and/or family
\item
  Ages 65+
\item
  Reflect on life of fulfillment (or not)
\end{itemize}

Love and Marriage

\begin{itemize}
\tightlist
\item
  Most adults pursue some kind of long-term relationship
\item
  Marriage associated with longer life, happiness
\item
  Gottman

  \begin{itemize}
  \tightlist
  \item
    Conflict and communication
  \item
    ``Four horsemen of the Apocalypse''
  \end{itemize}
\end{itemize}

Parenting

\begin{itemize}
\tightlist
\item
  Shift in identity, lifestyle
\item
  Children affect marriage
\item
  Empty nest myth
\end{itemize}

\includegraphics{assets/unit_3/slide_58.png}

\emph{Slide showing - Emotion, Memory, and Aging}

PSYCH @ The Driver's Seat

\begin{itemize}
\tightlist
\item
  Driving skills and age
\item
  UFOV Speed of Processing training

  \begin{itemize}
  \tightlist
  \item
    Computer-based
  \item
    Decreases accident risk
  \end{itemize}
\end{itemize}

\includegraphics{assets/unit_3/slide_60.png}

\emph{Slide showing - How Alzheimer's Disease Affects the Brain}

Working the Scientific Literacy Model: Aging and Cognitive Change (1 of 3)

\begin{itemize}
\tightlist
\item
  What do we know about cognitive abilities?

  \begin{itemize}
  \tightlist
  \item
    Fluid intelligence declines
  \item
    Crystalized intelligence remains largely intact
  \end{itemize}
\item
  How can science explain age-related differences in cognitive abilities?

  \begin{itemize}
  \tightlist
  \item
    Activation of brain areas
  \end{itemize}
\end{itemize}

Working the Scientific Literacy Model: Aging and Cognitive Change (2 of 3)

\begin{itemize}
\tightlist
\item
  Can we critically evaluate our assumptions about age-related cognitive changes?

  \begin{itemize}
  \tightlist
  \item
    Too simplistic to say memory declines

    \begin{itemize}
    \tightlist
    \item
      Many different types of memory
    \end{itemize}
  \item
    Compensation
  \end{itemize}
\item
  Why is this relevant?

  \begin{itemize}
  \tightlist
  \item
    Control over how one ages
  \end{itemize}
\end{itemize}

\includegraphics{assets/unit_3/slide_63.png}

\emph{Slide showing - Memory and Aging}

\textbf{Note:} The slides are intended to supplement the information found in your textbook. If you are having trouble viewing them, they can also be downloaded by scrolling to the bottom of the screen and clicking on the ``Unit 3 - Slides'' link.

\textbf{Designer Babies}

\begin{itemize}
\tightlist
\item
  Explore and reflect upon this contemporary and controversial issue. Designer babies pose many ethical issues and requires careful consideration.
\end{itemize}

\textbf{Cognitive Change}

\begin{itemize}
\tightlist
\item
  Reflect on your own development before taking a ``test'' to develop additional insights into your own developmental trajectory.
\end{itemize}

\textbf{Terminology Practice}

\begin{itemize}
\tightlist
\item
  Take this flip-card activity to self-evaluate how well you know some of the important terms from Chapter 10.
\end{itemize}

\textbf{\emph{Learning Lab Preparation}}

\begin{itemize}
\tightlist
\item
  Each topic will provide a question or scenario for you to consider prior to attending your Learning Lab. Be sure to carefully consider each prompt as you will be expected to contribute to the group discussion.
\end{itemize}
\end{reflect}

\hypertarget{resources-4}{%
\subsection*{Resources}\label{resources-4}}
\addcontentsline{toc}{subsection}{Resources}

Here are some additional resources that will help you complete this unit:

\begin{itemize}
\tightlist
\item
  Krause, M., Corts, D., Smith, S. C., \& Dolderman, D. (2018). \emph{Revel for An Introduction to Psychological Science, 2nd Canadian Edition.} Pearson Ed.\\
\item
  Other resources will be provided online.
\end{itemize}

\hypertarget{prenatal-development}{%
\section{Prenatal Development}\label{prenatal-development}}

\hypertarget{physical-development}{%
\subsection*{Physical Development}\label{physical-development}}
\addcontentsline{toc}{subsection}{Physical Development}

Prenatal development is a time of rapid growth and change. This rapid change continues throughout the first few years of life. Development during early life is clearly a function both of nature and of the environment.

\textbf{\emph{Questions to Consider}}

After you have read the first few pages of this chapter you should be able to answer the following questions:

\begin{itemize}
\tightlist
\item
  \textbf{\emph{How is the gender of an offspring determined?}}
\item
  \textbf{\emph{What differentiates zygotes from embryos and embryos from fetuses?}}
\end{itemize}

\emph{(These questions are intended for personal reflection - you are not intended to submit anything for assessment)}

\hypertarget{designer-babies}{%
\subsection*{Designer Babies}\label{designer-babies}}
\addcontentsline{toc}{subsection}{Designer Babies}

It is beginning to look inevitable that, however fierce the debate, the technology to make designer babies will happen - maybe just 20 years from now. Geneticists claim to have found the gene for good-parenting, genes for obesity, Alzheimer's, red hair, and even happiness. Incredibly, scientists have even constructed an artificial human chromosome, which could carry any genes a geneticist - or prospective parents - desired.

Embryo A technique called Pre-implantation Genetic Diagnosis (PGD) is already being used to screen embryos for genetic diseases. Embryos created outside the body using in vitro fertilization are tested to see whether they carry a genetic disorder before being transferred to the uterus. It's deeply controversial whether parents should ever be allowed to select embryos just because they're genetically different.

At the moment the technique is used for therapeutic purposes only, to screen for children who may have a deadly genetic disease. Even if some parents and their doctors were willing to use PGD for cosmetic or enhancement purposes, which remains absolutely taboo, the technique is limited in a crucial way - PGD can only select an embryo with genes inherited from the parents.

\textbf{\emph{Bottled Genes?}} \emph{One day parents may be able to pick any gene they desire from a range of bottled genes and have it put into their embryos. (quoted from ``Designer Babies'' website)}

\hypertarget{learning-activities-2}{%
\subsection*{Learning Activities}\label{learning-activities-2}}
\addcontentsline{toc}{subsection}{Learning Activities}

\hypertarget{designer-babies-1}{%
\subsubsection*{Designer Babies}\label{designer-babies-1}}
\addcontentsline{toc}{subsubsection}{Designer Babies}

This activity involves some reading and reflection around the topic of genetic engineering. As this is a contemporary issue, it will be valuable to familiarize yourself with some of the complexities of this technology and think critically about some of the ethical challenges. Your task is to read the following resources and carefully consider the implications of this technology:

\begin{itemize}
\tightlist
\item
  \href{https://www.statnews.com/2019/09/16/could-editing-the-dna-of-embryos-with-crispr-help-save-people-who-are-already-alive/}{\textbf{Editing the DNA of Embryos with CRISPR}}\\
\item
  \href{https://www.geneticsandsociety.org/internal-content/designer-babies-crispr-genetic-engineering}{\textbf{Designer Babies, CRISPR, \& Genetic Engineering}}
\end{itemize}

\hypertarget{learning-lab-preparation-7}{%
\subsubsection*{Learning Lab Preparation}\label{learning-lab-preparation-7}}
\addcontentsline{toc}{subsubsection}{Learning Lab Preparation}

Prior to your Learning Lab, take some time to think about the following scenario and questions. You will be asked to share your thoughts in this week's Learning Lab:

\emph{Modern techniques of conception and human genetic engineer­ing raise important new issues for human development. A pam­phlet containing the following message was left at doorsteps in TWU professor Philipchalk's neighborhood:}

\includegraphics{assets/unit_3/U3_T1_LearningACtivitity.JPG}

\emph{Image showing an example of surrogacy message }

(\emph{You may also wish to comment on the ``Designer Babies'' topic above.})

\begin{enumerate}
\def\labelenumi{\arabic{enumi}.}
\tightlist
\item
  \textbf{\emph{How do you feel about this request?}}
\item
  \textbf{\emph{What problems might you anticipate?}}
\end{enumerate}

\hypertarget{infancy-and-childhood}{%
\section{\texorpdfstring{\textbf{Infancy and Childhood}}{Infancy and Childhood}}\label{infancy-and-childhood}}

\hypertarget{a-childs-view-of-god}{%
\subsection*{A Child's View of God}\label{a-childs-view-of-god}}
\addcontentsline{toc}{subsection}{A Child's View of God}

One evening on a camping trip several years ago, my wife and I listened outside the tent as our five-year-old Joelle and three-year-old Matthew tried to get to sleep. Always the ``mother,'' Joelle attempted to dispel her little brother's fear of bears and other wild creatures by reminding him that Jesus was watching over them. Not content with generalities, Matt responded, ``Does Jesus got a gun?'' \emph{(Psychology and Christianity by Ronald Philipchalk, p.~141)}

As any Sunday School teacher knows, children see God differently from adults---often in very concrete terms (protection requires a gun!). Studying cognitive development can help us to understand as well as teach children at their own level.

\hypertarget{the-process-of-cognitive-change}{%
\subsection*{The Process of Cognitive Change}\label{the-process-of-cognitive-change}}
\addcontentsline{toc}{subsection}{The Process of Cognitive Change}

Our textbook provides a good summary of the structure (stages) of cognitive development. The section, however, does not address the process by which a person moves from one stage to the next. Piaget believed that the key to cognitive development is something called cognitive conflict or cognitive disequilibrium. For cognitive development to proceed, the individual must constantly re-evaluate his or her schemas. According to Piaget, we develop schemas from an early age of life. Schemas are our cognitive representations of the world. Schemas help us to organize our experiences. They also allow us to make predictions about what outcomes might result from particular behaviours. Schemas are very important in helping us to understand and to adapt to the world.

Although schemas are important in helping us to understand the world, they are not always accurate. People at all ages can have mental representations of the world that are not correct.

\textbf{\emph{Can you think of any examples of inaccurate schemas?}}

Although people at all ages can have inaccurate mental representations of the world, children are especially prone to view the world in an incorrect way. The reason children may view the world in an incorrect way is because the structure of their cognitive processing is developing. Piaget believed that inaccurate schemas are changed only when they are challenged in the cognitive structure of the child. This challenge has been termed cognitive conflict.

Basically, the process of cognitive change works as follows:

\begin{itemize}
\tightlist
\item
  People are motivated to maintain a state of cognitive equilibrium.\\
\item
  When a child encounters information from the world and the information is inconsistent with his or her schema, the new piece of information creates a state of disequilibrium or cognitive conflict.\\
\item
  \textbf{\emph{Equilibrium}} may be restored through one of the two processes of adapation called assimilation and accommodation.\\
\item
  \textbf{\emph{Assimilation}} occurs when a child re-organizes the new information in such a way as to make the new piece of information consistent with his or her preexisting schema of the world.\\
\item
  \textbf{\emph{Accommodation}} occurs when a child alters his or her schema such that the new piece of information can now be incorporated into the new schema.
\end{itemize}

Thus the process of accommodation produces the greatest cognitive change. Can you think of examples of both assimilation and accommodation? Here is an example:

\textbf{Equilibrium-Preexisting Schema:} Child has grown up in an environment where all people he interacted with were of the same race (mom, dad, siblings, grandma, grandpa, etc.) Child has seen people of other racial groups, but has never interacted with them. Child develops the schema that people tend to like others who are of the same race as him or her.

\textbf{Cognitive Conflict Produced:} At four years of age the child begins to attend preschool. At this time he starts to interact with children of various races. The child begins to develop a friendship with a child of a different race. This friendship creates cognitive conflict for the child: ``How can I like someone who is a different color?'' To resolve this cognitive conflict, the child has two options:

\textbf{\emph{Option A}}- \textbf{Assimilation:} In order to maintain his or her preexisting schema, the child re-organizes the information such that the other child is not perceived to be so dissimilar after all: ``Maybe he is a different color from me, but we both speak English. We must not be so dissimilar after all.''

\textbf{\emph{Option B}}- \textbf{Accommodation:} The child's preexisting schema is altered such that the new information can be incorporated into a new way of perceiving the world, ``Maybe I can be friends with someone who is different from me.''

\hypertarget{cognitive-equilibrium-is-restored}{%
\subsection*{Cognitive Equilibrium is Restored}\label{cognitive-equilibrium-is-restored}}
\addcontentsline{toc}{subsection}{Cognitive Equilibrium is Restored}

Although cognitive equilibrium is restored via either assimilation or accommodation, assimilation serves to maintain an inaccurate schema (that differences inhibit the development of friendships) whereas accommodation serves to produce cognitive change and hence produces a more accurate representation of the world (that differences do not inhibit the development of friendships).

\hypertarget{learning-activities-3}{%
\subsection*{Learning Activities}\label{learning-activities-3}}
\addcontentsline{toc}{subsection}{Learning Activities}

\begin{reflect}
\hypertarget{cognitive-change}{%
\subsubsection*{Cognitive Change}\label{cognitive-change}}
\addcontentsline{toc}{subsubsection}{Cognitive Change}

The first three links below are articles that are intended to give you an opportunity to reflect upon your own considerations around development. The last link is a test - along with the first three links, it is intended to provide some insights about your developmental trajectory in light of your crisis resolution, attachment style, and parenting styles:

\begin{itemize}
\tightlist
\item
  \href{http://www.childdevelopmentinfo.com/development/erickson.shtml}{\textbf{Erik Erikson's Stages of Social-Emotional Development}}\\
\item
  \href{http://www.attachmentdisorder.net/}{\textbf{The Link Between Substance Abuse and Attachment Disorder}}\\
\item
  \href{https://www.mayoclinic.org/healthy-lifestyle/childrens-health/in-depth/stepfamilies/art-20047046}{\textbf{Stepfamilies: How to Help Your Child Adjust}}\\
\item
  \href{https://www.3smartcubes.com/pages/tests/parentingstyle/parentingstyle_instructions/}{\textbf{What is Your Parenting Style}}
\end{itemize}

\hypertarget{learning-lab-preparation-8}{%
\subsubsection*{Learning Lab Preparation}\label{learning-lab-preparation-8}}
\addcontentsline{toc}{subsubsection}{Learning Lab Preparation}

Prior to your Learning Lab, take some time to think about the following questions. You will be asked to share your thoughts in this week's Learning Lab:

\begin{itemize}
\tightlist
\item
  \textbf{\emph{What is God like for children of different levels of cognitive development? If you can, give some examples from children you know\ldots.}}
\item
  \textbf{\emph{Would children even have an idea of God if they were not taught it?}}
\end{itemize}
\end{reflect}

\hypertarget{adolescence}{%
\section{Adolescence}\label{adolescence}}

John has just turned 13. Over the past year he has experienced may changes. He has grown over six inches and he has developed acne over his face and back. Not only is he changing physically, he is also experiencing a wave of emotional, spiritual, cognitive, and sexual changes. John has become self-focused and very self-critical. In addition, he is beginning to think abstractly and to challenge adults' ``dominion'' on knowledge. John is also on a quest to understand ``who he is'' and ``what his place is in the world''. John's quest for an identity makes him more vulnerable to peer pressure and to the influence of radical groups and cults. During this time that we call adolescence, John will make many decisions that will have a profound effect on the direction his life will take.

\emph{Does any of the above sound familiar?}

Before you begin reading the textbook section on adolescence, think back to your own adolescence. As you think about your experience of adolescence, use the following questions to guide your reflection:

\begin{itemize}
\tightlist
\item
  What physical changes did you experience in adolescence?
\item
  How did these physical changes make you feel?
\item
  In what ways did your view of the world change during adolescence?
\item
  How did your way of treating other people change during adolescence?
\item
  What was most important to you during adolescence?
\item
  To what extent is ``who you are today'' a function of ``who you became during adolescence''?
\end{itemize}

\hypertarget{no-adolescence}{%
\subsection*{No Adolescence?}\label{no-adolescence}}
\addcontentsline{toc}{subsection}{No Adolescence?}

In other times and in other cultures to­day, adolescence does not exist as a significant and distinct period of develop­ment. This might seem surprising and difficult to imagine. Think of how modern society would be different, or if it could even exist, without a period of adolescence. What are the advantages and disadvantages of having an adolescent period?

\hypertarget{identity}{%
\subsection*{Identity}\label{identity}}
\addcontentsline{toc}{subsection}{Identity}

The concept of identity is a rich topic for consideration. The most familiar aspect of identity is occupational identity, since much of ``who we are'' in our society rests on the kind of work we do. Perhaps you can readily relate to this in your choice of major. Less familiar, but equally important, is ideological identity. Ideologi­cal identity, including both religious and political orientations, may un­dergo a tremen­dous upheaval during your student years. Do you have the same political beliefs as your parents? What about religious beliefs? Conflict and questioning of parental beliefs and values may be a necessary part of establishing your personal iden­tity---even if the beliefs and values you ultimately adopt are the same as those of your parents.

\hypertarget{learning-activities-4}{%
\subsection*{Learning Activities}\label{learning-activities-4}}
\addcontentsline{toc}{subsection}{Learning Activities}

\begin{reflect}
\hypertarget{read-and-reflect}{%
\subsubsection*{Read and Reflect}\label{read-and-reflect}}
\addcontentsline{toc}{subsubsection}{Read and Reflect}

In this section we explore adolescence and the changes in development we go through. Below are some resources that help to help support your understanding:

\begin{itemize}
\tightlist
\item
  \href{https://www.sciencedirect.com/journal/journal-of-adolescence}{\textbf{Journal of Adolescence}}\\
\item
  \href{https://childdevelopmentinfo.com/child-development/erickson/\#gs.d8mpcv}{\textbf{Erik Erikson's Stages of Social-Emotional Development}}\\
\item
  \href{http://drjamesdobson.org/quiz/parenting-quiz/4-tips-for-parenting-teenagers}{\textbf{4 Tips for Parenting Teenagers}}
\end{itemize}

\hypertarget{terminology-practice}{%
\subsubsection*{Terminology Practice}\label{terminology-practice}}
\addcontentsline{toc}{subsubsection}{Terminology Practice}

In order to review some of the major terms from Chapter 10 in your textbook, practice using the activity below. Although you will not be evaluated on these terms, they will assist you in the assessments for this course:

\hypertarget{learning-lab-preparation-9}{%
\subsubsection*{Learning Lab Preparation}\label{learning-lab-preparation-9}}
\addcontentsline{toc}{subsubsection}{Learning Lab Preparation}

Prior to your Learning Lab, take some time to think about the following questions. You will be asked to share your thoughts in this week's Learning Lab:

\begin{itemize}
\tightlist
\item
  \textbf{\emph{Would you want to live your adolescence over again if you could? Why or why not?}}
\end{itemize}
\end{reflect}

\hypertarget{assessment-2}{%
\section*{Assessment}\label{assessment-2}}
\addcontentsline{toc}{section}{Assessment}

\begin{assessment}
While there is no ``formal'' assignment that you will be responsible for submitting for Unit 3, you will be expected to participate in discussion during your Learning Lab. Your facilitator will be providing a participation mark based on your contributions. Below is some information to consider prior to attending your Learning Lab:

\emph{Active participation in group exercises, reflection, and critical discourse is an essential component of this course. You are expected to show respect for all members of the course, both in your speech and actions. Contribute by actively observing and listening, raising thoughtful questions, examining relevant issues, building on others' ideas, analyzing and evaluating the group's thinking, synthesizing key points, and expanding the group's perspectives. Take care not to dominate a conversation, giving space for others to speak. When in small groups help maintain the focus, flow, and quality of conversations, and take the initiative to invite others (particularly those who are quiet) to speak.}

\textbf{Rubric for Participation in Learning Labs}

\begin{longtable}[]{@{}
  >{\raggedright\arraybackslash}p{(\columnwidth - 4\tabcolsep) * \real{0.3333}}
  >{\raggedright\arraybackslash}p{(\columnwidth - 4\tabcolsep) * \real{0.3333}}
  >{\raggedright\arraybackslash}p{(\columnwidth - 4\tabcolsep) * \real{0.3333}}@{}}
\toprule\noalign{}
\begin{minipage}[b]{\linewidth}\raggedright
Emerging (0-64\%)
\end{minipage} & \begin{minipage}[b]{\linewidth}\raggedright
Developing (65-89\%)
\end{minipage} & \begin{minipage}[b]{\linewidth}\raggedright
Mastering (90-100\%)
\end{minipage} \\
\midrule\noalign{}
\endhead
\bottomrule\noalign{}
\endlastfoot
Never to almost never: Demonstrates active listening (as indicated by disengaged body language and no to rare comments that build on others' remarks),Initiates any contributions in class or small groups, Makes insightful or constructive comments, Helps maintain a supportive space for others to speak. & Sometimes to fairly often: Demonstrates active listening (as indicated by somewhat to often engaged body language and comments that build on others' remarks), Initiates a contribution at least once in a class or small group discussion; Makes insightful or constructive comments, Helps maintain a supportive space for others to speak. & Very often to nearly always: Demonstrates active listening (as indicated by fully engaged body language and comments that build on others' remarks), Initiates more than one contribution in a class or small group discussion, Makes insightful or constructive comments, Creates a space for others to speak and takes initiative to include others. \\
\end{longtable}
\end{assessment}

\hypertarget{checking-your-learning-2}{%
\section*{Checking your Learning}\label{checking-your-learning-2}}
\addcontentsline{toc}{section}{Checking your Learning}

\begin{progress}
Before you move on to the next unit, check that you are able to:

\begin{itemize}
\tightlist
\item
  Define the key terminology related to prenatal and infant physical development, infancy and childhood, and adolescent development.\\
\item
  Understand advantages and disadvantages to different research designs in developmental psychology.\\
\item
  Understand the cognitive changes that occur during infancy and childhood, and the importance of attachment and the different styles of attachment.\\
\item
  Understand the process of identity formation, relationships, and moral emotions during adolescence.\\
\item
  Apply your understanding to identify the best ways expectant parents can ensure the health of their developing fetus, how to promote learning, and how to categorize moral reasoning.\\
\item
  Analyze the effects of preterm birth, how to effectively discipline children, and adolescent judgment and risk taking.
\end{itemize}
\end{progress}

\hypertarget{the-developing-person---part-2}{%
\chapter{The Developing Person - Part 2}\label{the-developing-person---part-2}}

\hypertarget{overview-3}{%
\section*{Overview}\label{overview-3}}
\addcontentsline{toc}{section}{Overview}

Building upon Unit 3, this unit (Part 2) will focus on the cognitive, physical, and social changes faced during young (and emerging) adulthood, middle adulthood, and late adulthood. \emph{Please note as well, that although it is not covered in the text, Topic 2 discusses the important, and inevitable, subject of dying and death.}

\hypertarget{topics-3}{%
\subsection*{Topics}\label{topics-3}}
\addcontentsline{toc}{subsection}{Topics}

This unit is divided into the following topics:

\begin{enumerate}
\def\labelenumi{\arabic{enumi}.}
\tightlist
\item
  Adulthood\\
\item
  Death and Dying
\end{enumerate}

\hypertarget{learning-outcomes-3}{%
\subsection*{Learning Outcomes}\label{learning-outcomes-3}}
\addcontentsline{toc}{subsection}{Learning Outcomes}

By the end of this unit, student's will be able to:

\begin{itemize}
\tightlist
\item
  Define the key terminology concerning adulthood and aging.\\
\item
  Describe the key areas of growth experiences by emerging adults.\\
\item
  Explain age-related disorders such as Alzheimer's disease.\\
\item
  Describe how cognitive abilities change with age.\\
\item
  Apply effective communication principles to the challenge of improving your own relationships.\\
\item
  Analyze the stereotype that old age is a time of unhappiness.
\end{itemize}

\hypertarget{activity-checklist-3}{%
\subsection*{Activity Checklist}\label{activity-checklist-3}}
\addcontentsline{toc}{subsection}{Activity Checklist}

Here is a checklist of learning activities you will benefit from in completing this unit. You may find it useful for planning your work:

\begin{reflect}
\textbf{Read and Reflect}

\begin{itemize}
\tightlist
\item
  Read \emph{Krause et al.~(2021). Revel for An Introduction to Psychological Science, 3rd Canadian Edition}\\
\item
  Review \href{PSYC106-CH10LifespanDevelopment-3rdEd.pptx}{\emph{Unit 4 - Slides}}\\
  //todo \#4
\end{itemize}

\textbf{Note:} the slides are intended to supplement the information found in your textbook. If you are having trouble viewing them, they can also be downloaded by scrolling to the bottom of the screen and clicking on the ``Unit 4 - Slides'' link.*

\textbf{Relationships and Aging}

\begin{itemize}
\tightlist
\item
  Read about Erikson's Eight Psychosocial Stages that highlight the importance of relationships in healthy aging.
\end{itemize}

\textbf{Death and Dying}

\begin{itemize}
\tightlist
\item
  Take a moment to read through some resources that are focused on aging and dying. These readings are intended to provoke thought around the inevitability of death.
\end{itemize}

\textbf{\emph{Learning Lab Preparation}}

\begin{itemize}
\tightlist
\item
  Each topic will provide a question or scenario for you to consider prior to attending your Learning Lab. Be sure to carefully consider each prompt as you will be expected to contribute to the group discussion.
\end{itemize}

\hypertarget{assessment-3}{%
\subsubsection*{Assessment}\label{assessment-3}}
\addcontentsline{toc}{subsubsection}{Assessment}

\begin{itemize}
\item
  At the end of this unit, not only will you be assessed on your participation during the Learning Lab discussions, but you will also be responsible for submitting your first writing assignment.
\item
  \textbf{\emph{Writing Assignment \#1}} is outlined on the ``Assessment'' page - expectations for this assignment, as well as a grading rubric, can be found by clicking on that tab.
\item
  After completing this assignment, you will also be expected to complete \textbf{\emph{Quiz \#1.}} Again, information about the quiz can be found by clicking on the ``Assessment'' tab.
\end{itemize}
\end{reflect}

\hypertarget{resources-5}{%
\subsection*{Resources}\label{resources-5}}
\addcontentsline{toc}{subsection}{Resources}

Here are some additional resources that will help you complete this unit:

\begin{itemize}
\tightlist
\item
  Krause, M., Corts, D., Smith, S. C., \& Dolderman, D. (2018). \emph{Revel for An Introduction to Psychological Science, 2nd Canadian Edition.} Pearson Ed.
\item
  Other resources will be provided online.
\end{itemize}

\hypertarget{adulthood}{%
\section{Adulthood}\label{adulthood}}

\hypertarget{psychosocial-development}{%
\subsection*{Psychosocial Development}\label{psychosocial-development}}
\addcontentsline{toc}{subsection}{Psychosocial Development}

There are many changes we experience during the time we call adulthood. According to Erikson's theory of psychosocial development, adults move through three important stages during which they must resolve important psychosocial conflicts:

\begin{itemize}
\tightlist
\item
  Early Adulthood\\
\item
  Middle Adulthood\\
\item
  Late Adulthood
\end{itemize}

\textbf{\emph{In Early Adulthood}} (20-40 years of age) the psychosocial crisis is one of intimacy versus isolation. Although it is often perceived as critical whether or not the person develops an intimate relationship with a lifelong partner, Erikson believed that although this is one component of intimacy in this stage, other types of intimacy are also important---like developing intimate relationships with workmates, colleagues, and even one's children. If one fails to develop a sense of intimacy during this stage, he or she will feel isolation.

\textbf{\emph{In Middle Adulthood}} (40-65 years of age) the psychosocial crises is one of generativity versus stagnation. What is generativity? Erikson used the term gererativity to describe the process by which a person feels like he or she is making a lasting contribution in the world. Usually, generativity involves giving something back to the next generation. People at this stage can feel generative in many ways---coaching a softball team, being a scout leader, teaching children in Sunday School, been a big brother/sister, etc. If a person fails to develop a sense of generativity during this stage, he or she will feel a sense of stagnation.

\textbf{\emph{In Late Adulthood}} (65+ years of age) the psychosocial crisis is one of ego integrity versus despair. If the individual looks back on his or her life and feels a sense of pride in the accomplishments he or she has made, then the individual will feel a sense of ego integrity. However, if the person looks back on his or her life and does not see the significance of his or her accomplishments, then he or she will feel a sense of despair at the meaninglessness of his or her life.

\hypertarget{guilt}{%
\subsection*{Guilt}\label{guilt}}
\addcontentsline{toc}{subsection}{Guilt}

One of the most significant developments in childhood, from both a secular and Christian point of view, is a conscience with its attendant guilt. Is guilt simply a conditioned emotional response (as the behaviorist might say)? Is it the result of conflict with the superego (as a psychoanalyst would say)? Is it the failure to live up to our self-concept (as the humanist might say)? Is it the voice of the Holy Spirit? Or is it some combination of these? (For help on this issue and an important distinction between false and true guilt see Counts and Narramore (1970), Narramore (1984), Tournier (1962).) \emph{(from Psychology and Christianity, by Ronald Philipchalk, p.~146)}

The textbook discusses moral development in childhood and adolescence. However, many adults are troubled by questions of right and wrong, and in particular, by feelings of guilt. The Christian authors noted above suggest that many guilt feelings are not true guilt but false guilt carried over from childhood experiences. Understanding how conscience and guilt feelings develop can help to liberate us from unnecessary false guilt inappropriately attributed to God.

\hypertarget{learning-activities-5}{%
\subsection*{Learning Activities}\label{learning-activities-5}}
\addcontentsline{toc}{subsection}{Learning Activities}

\hypertarget{read-and-reflect-1}{%
\subsubsection*{Read and Reflect}\label{read-and-reflect-1}}
\addcontentsline{toc}{subsubsection}{Read and Reflect}

This activity involves some reading and reflection around Erikson's Eight Psychosocial Stages and an article from Harvard University illuminating the importance of relationships in healthy aging.

\begin{itemize}
\tightlist
\item
  \href{https://childdevelopmentinfo.com/child-development/erickson/\#gs.d8mpcv}{\textbf{Erik Erikson's Stages of Social-Emotional Development}}\\
\item
  \href{https://news.harvard.edu/gazette/story/2017/04/over-nearly-80-years-harvard-study-has-been-showing-how-to-live-a-healthy-and-happy-life/}{\textbf{Good Genes are Nice, but Joy is Better}}
\end{itemize}

\hypertarget{learning-lab-preparation-10}{%
\subsubsection*{Learning Lab Preparation}\label{learning-lab-preparation-10}}
\addcontentsline{toc}{subsubsection}{Learning Lab Preparation}

Prior to your Learning Lab, take some time to think about the following questions. You will be asked to share your thoughts in this week's Learning Lab:

\begin{itemize}
\tightlist
\item
  \textbf{\emph{What do you hope to accomplish during your ``adult development'' years?}}

  \begin{itemize}
  \tightlist
  \item
    \emph{(If you are past these years, what are you most pleased with?)}
  \end{itemize}
\end{itemize}

\hypertarget{death-and-dying}{%
\section{Death and Dying}\label{death-and-dying}}

\hypertarget{death}{%
\subsection*{Death}\label{death}}
\addcontentsline{toc}{subsection}{Death}

Secular psychologists see death as final. Christians, however, see resurrection beyond, with death being but another step in that direction. What implications do these beliefs have for the process of dying?

As medical technology has advanced death has become more and more difficult to define. We need to focus our attention less on preserving the physical and more on preserving the personhood of the individual. This means giving greater attention to our concept of the dying person created in the image of God (as the abortion issue has forced us to do at the other end of life). When is personhood sacrificed to technical efficiency? Should we advocate a more ``natural death?'' What is ``natural death?'' How far does one go in ``allowing'' natural death? \emph{(from Psychology and Christianity, by Ronald Philipchalk, p.~146)}

\hypertarget{hospice}{%
\subsection*{Hospice}\label{hospice}}
\addcontentsline{toc}{subsection}{Hospice}

You matter because of who you are. You matter to the last moment of your life, and we will do all we can not only to help you die peacefully, but also to live until you die.'' \emph{(Dame Cicely Saunders)}

During the Crusades of the Middle Ages a hospice provided lodging for travelers: a place of refuge and comfort. So in 1967, when Dame Cicley Saunders opened a facility in London to provide care and comfort to dying people and their families, St.~Christopher's hospice was an appropriate name. \emph{(Courtesy of the Langley Hospice Society)}

\emph{If you would like to know more about the hospice movement in this area, you can contact the Langley Hospice Society at 604-530-1115.}

\hypertarget{cultural-variations}{%
\subsection*{Cultural Variations}\label{cultural-variations}}
\addcontentsline{toc}{subsection}{Cultural Variations}

The following issues are often subject to cul­tural variations:

\begin{itemize}
\tightlist
\item
  Adolescence is unknown in some cultures\\
\item
  Adolescent struggles and conflict are much less in some cultures (e.g., )\\
\item
  Stage theories may not apply in other cultures\\
\item
  Ageism, especially with regard to intellectual abilities, is reduced, unknown, or even reversed in some cultures where the wisdom of old age is venerated\\
\item
  The ``social clock'' may be set differently in other cultures\\
\item
  Attitudes toward death vary greatly between cultures
\end{itemize}

\hypertarget{assessment-4}{%
\section{Assessment}\label{assessment-4}}

\begin{assessment}
In addition to your participation in this unit's Learning Lab, you will also be responsible for two formal assessments. The first, is your \textbf{Writing Assignment \#1} - more information can be found on this assignment by scrolling down the page. The second assessment that will take place is \textbf{Quiz \#1} - more information can be found by scrolling down the page.

\emph{Active participation in group exercises, reflection, and critical discourse is an essential component of this course. You are expected to show respect for all members of the course, both in your speech and actions. Contribute by actively observing and listening, raising thoughtful questions, examining relevant issues, building on others' ideas, analyzing and evaluating the group's thinking, synthesizing key points, and expanding the group's perspectives. Take care not to dominate a conversation, giving space for others to speak. When in small groups help maintain the focus, flow, and quality of conversations, and take the initiative to invite others (particularly those who are quiet) to speak.}

\textbf{Rubric for Participation in Learning Labs}

\begin{longtable}[]{@{}
  >{\raggedright\arraybackslash}p{(\columnwidth - 4\tabcolsep) * \real{0.3333}}
  >{\raggedright\arraybackslash}p{(\columnwidth - 4\tabcolsep) * \real{0.3333}}
  >{\raggedright\arraybackslash}p{(\columnwidth - 4\tabcolsep) * \real{0.3333}}@{}}
\toprule\noalign{}
\begin{minipage}[b]{\linewidth}\raggedright
Emerging (0-64\%)
\end{minipage} & \begin{minipage}[b]{\linewidth}\raggedright
Developing (65-89\%)
\end{minipage} & \begin{minipage}[b]{\linewidth}\raggedright
Mastering (90-100\%)
\end{minipage} \\
\midrule\noalign{}
\endhead
\bottomrule\noalign{}
\endlastfoot
Never to almost never: Demonstrates active listening (as indicated by disengaged body language and no to rare comments that build on others' remarks),Initiates any contributions in class or small groups, Makes insightful or constructive comments, Helps maintain a supportive space for others to speak. & Sometimes to fairly often: Demonstrates active listening (as indicated by somewhat to often engaged body language and comments that build on others' remarks), Initiates a contribution at least once in a class or small group discussion; Makes insightful or constructive comments, Helps maintain a supportive space for others to speak. & Very often to nearly always: Demonstrates active listening (as indicated by fully engaged body language and comments that build on others' remarks), Initiates more than one contribution in a class or small group discussion, Makes insightful or constructive comments, Creates a space for others to speak and takes initiative to include others. \\
\end{longtable}
\end{assessment}

\hypertarget{writing-assignment-1}{%
\section{Writing Assignment \#1}\label{writing-assignment-1}}

Your first assignment for the course will involve reading a formal research article and extracting important information. Research articles play an important role in psychology as they present new findings and research that help us better understand the subject field. It is, however, important for us to understand what information is important as we weigh the value of the article we are reading. This assignment attempts to help us better understand those considerations.

Your first task is to read \href{assets/unit_4/Assessment_Empirical_Short.pdf}{\textbf{How to Read Empirical Articles}}

As you can see, it is important to understand \textbf{\emph{why}} you are reading an article and it is important to understand \textbf{\emph{what}} you want to get out of it.

Next, your task is to read \href{assets/unit_4/Assessment_Empirical_Long.pdf}{\textbf{How to Read a Journal Article in Social Psychology}}

\begin{itemize}
\tightlist
\item
  This article should help support your understanding of formal research articles and how to read them to support your research and learning.
\end{itemize}

\hypertarget{writing-assignment}{%
\subsubsection*{Writing Assignment}\label{writing-assignment}}
\addcontentsline{toc}{subsubsection}{Writing Assignment}

Finally, it is time to apply what we have learned. Download the following article:

\href{assets/unit_4/Assessment_FOMO_Article.pdf}{\textbf{No More FOMO}}

\hypertarget{writing-assignment-2}{%
\subsection*{Writing Assignment}\label{writing-assignment-2}}
\addcontentsline{toc}{subsection}{Writing Assignment}

After reading the article above, use the content to respond to the following questions:

\begin{enumerate}
\def\labelenumi{\arabic{enumi}.}
\tightlist
\item
  \textbf{Does this study reflect basic or applied research? How do you know?}\\
\item
  \textbf{What level(s) of analysis (biological, psychological, environmental) are being employed in this study? Explain and identify the specific details of the study that show this.}\\
\item
  \textbf{What were the findings of previous research that led the researchers to do this study? What was the research question that they attempted to answer? What was (were) the researchers' hypothesis(es)?}\\
\item
  \textbf{How was this study conducted? Was it an experiment or a non-experiment? How can you tell? What were the variables under investigation in this study? Identify the dependent and independent variables, if applicable.}\\
\item
  \textbf{What were the main findings of this study? How do they relate to the original study hypothesis(es)?}\\
\item
  \textbf{How do the researchers explain their results? What are some other possible explanations? What are some strengths of this study? Limitations?}\\
\item
  \textbf{What are the possible implications of these results in the real world?}
\end{enumerate}

\hypertarget{rubric-for-writing-assignment-1}{%
\subsubsection*{Rubric for Writing Assignment \#1}\label{rubric-for-writing-assignment-1}}
\addcontentsline{toc}{subsubsection}{Rubric for Writing Assignment \#1}

\textbf{In addition to the criteria included in the rubric below, it is expected that students submit their assignments with an APA style title page.}

Estimate that each response will be between one paragraph to ½ a page (depending on what the question is asking and how much information is in the article). The total amount of content will be between 2 ½ to 3 pages (not including the title page).

\textbf{Exceeds Expectations} \textbar{} \textbf{Meets Expectations} \textbar{} \textbf{Minimally Meets Expectations} \textbar{} \textbf{Does Not Meet Expectations} \textbar{}

\textbar\textbar--\textbar--\textbar-\textbar{}
Responses are accurate and address the questions in a comprehensive manner\textbar Responses are accurate and address the questions but could demonstrate more meaningful comprehension of content\textbar Responses may lack accuracy in addressing the questions and/or may demonstrate limited comprehension of content\textbar Responses do not accurately address the questions or do not demonstrate any comprehension of content\textbar{}
Clear, precise and well-reasoned responses\textbar{} Mostly clear, precise and well-reasoned responses \textbar Some clear, precise and well-reasoned responses \textbar{} Responses lack clarity, logic and/or precision\textbar{}
\textbar Spelling and grammar are accurate.\textbar{} Minor and/or few spelling or grammatical errors.\textbar{} Several spelling or grammatical errors.\textbar{}

\textbf{\emph{To submit your completed assignment, scroll to the bottom of the page and click on the ``Writing Assignment \#1'' tab - follow the directions there.}}

\hypertarget{quiz-1}{%
\subsection*{Quiz \#1}\label{quiz-1}}
\addcontentsline{toc}{subsection}{Quiz \#1}

Upon completing your first Writing Assignment, you will next complete your first quiz.

Your instructor will provide additional instructions. Student's will be provided with one attempt. You will be provided information about when the quiz will open so you can begin.

\textbf{\emph{To begin your quiz,}} click on the \textbf{Quiz \#1} tab at the bottom of the screen.

\hypertarget{checking-your-learning-3}{%
\section*{Checking your Learning}\label{checking-your-learning-3}}
\addcontentsline{toc}{section}{Checking your Learning}

\begin{progress}
Before you move on to the next unit, check that you are able to:

\begin{itemize}
\item
  Define the key terminology concerning adulthood and aging.
\item
  Describe the key areas of growth experiences by emerging adults.
\item
  Explain age-related disorders such as Alzheimer's disease.
\item
  Describe how cognitive abilities change with age.
\item
  Apply effective communication principles to the challenge of improving your own relationships.
\item
  Analyze the stereotype that old age is a time of unhappiness.
\end{itemize}
\end{progress}

\hypertarget{personality}{%
\chapter{Personality}\label{personality}}

\hypertarget{overview-4}{%
\section*{Overview}\label{overview-4}}
\addcontentsline{toc}{section}{Overview}

In Unit 5 of the course, our attention will focus on personality. The study of personality attempts to answer the questions: \emph{Who are you?} and, \emph{How did you become who you are?} As human's are astoundingly complex, there is not any one theory that can capture, in totality, a description of who you are; hence, there are numerous approaches and theories that attempt to address these questions. In this unit, you will learn about some of the contemporary approaches to the study of Personality, and examine and review cultural, biological, Psychodynamic and Humanistic approaches to Personality.

\hypertarget{topics-4}{%
\subsection*{Topics}\label{topics-4}}
\addcontentsline{toc}{subsection}{Topics}

This unit is divided into the following topics:

\begin{enumerate}
\def\labelenumi{\arabic{enumi}.}
\tightlist
\item
  Psychoanalytic View\\
\item
  Humanistic View\\
\item
  Trait View\\
\item
  Social-Cognitive View
\end{enumerate}

\hypertarget{learning-outcomes-4}{%
\subsection*{Learning Outcomes}\label{learning-outcomes-4}}
\addcontentsline{toc}{subsection}{Learning Outcomes}

By the end of this unit, student's will be able to:

\begin{itemize}
\tightlist
\item
  Define the key terminology associated with contemporary approaches, cultural and biological approaches, and psychodynamic and humanistic approaches to personality.\\
\item
  Describe the behaviourist and social-cognitive views, evolutionary theories, and Freudian developmental and defensive explanations of personality.\\
\item
  Apply the ``Big Five'' personality traits, psychodynamic, and humanistic perspectives to understand personality.\\
\item
  Analyze the roots of violence and prejudice, roles of personality traits and psychological and physical states, and the genetic basis of personality in determining behaviour.\\
\item
  Assess claims that males and females have fundamentally different personalities and whether projective tests are valid measures of personality.
\end{itemize}

\hypertarget{activity-checklist-4}{%
\subsection*{Activity Checklist}\label{activity-checklist-4}}
\addcontentsline{toc}{subsection}{Activity Checklist}

Here is a checklist of learning activities you will benefit from in completing this unit. You may find it useful for planning your work:

\begin{reflect}
\textbf{Read and Reflect}

\begin{itemize}
\tightlist
\item
  Read \emph{Krause et al.~(2021). Revel for An Introduction to Psychological Science, 3rd Canadian Edition}\\
\item
  Review \href{PSYC106-CH12Personality-3rdEd.pptx}{\emph{Unit 5- Slides}}
\end{itemize}

CLICK HERE

An Introduction to Psychological Science - Chapter 12 - Personality

True or False?

\begin{enumerate}
\def\labelenumi{\arabic{enumi}.}
\tightlist
\item
  Freud believed that boys develop sexual desires for their mother when they are between 3 and 6 years of age.\\
\item
  One of the most reliable and valid measures of personality is the Rorschach inkblot test.\\
\item
  Dreams are disguised wish fulfillments that can be interpreted by skilled analysts.\\
\item
  Psychologists generally agree that painful experiences commonly get pushed out of awareness and into the unconscious.\\
\item
  Most Americans believe that self-esteem is very important for motivating a person to work hard and succeed.\\
\item
  Personality differences among dogs are as evident and as consistently judged as personality differences among humans.\\
\item
  Most people recognize that personality descriptions based on horoscopes are invalid.\\
\item
  From a few minutes' inspection of our living and working spaces, someone can, with reasonable accuracy, assess our emotional stability.\\
\item
  Older people are happiest when they do not have to take responsibility for everyday decisions that affect their lives.\\
\item
  The majority of people suffer from low self-esteem.
\end{enumerate}

Modules

\begin{itemize}
\tightlist
\item
  Contemporary Approaches to Personality\\
\item
  Cultural and Biological Approaches to Personality\\
\item
  Psychodynamic and Humanistic Approaches to Personality
\end{itemize}

Learning Objectives

\begin{itemize}
\tightlist
\item
  Know the key terminology associated with contemporary approaches to personality.\\
\item
  Understand the behaviourist and social-cognitive views of personality.\\
\item
  Apply self-report methods to understand your own personality.\\
\item
  Analyze the personality roots of violence and prejudice.\\
\item
  Analyze the relative roles of personality traits and psychological and physical states in determining behaviour.
\end{itemize}

Personality Exercise: Who am I?

\begin{itemize}
\tightlist
\item
  Describe your own personality by simply answering the question ``Who am I?'' on a piece of paper. Write at the top of the page ``I am . . .'' and then list about 20 characteristics that describe you. List what you consider to be some of your own positive and negative personality qualities.
\end{itemize}

Personality Exercise: Some Terms

\begin{itemize}
\tightlist
\item
  Sincere, honest, faithful, loyal, modest/unassuming, fair-minded, sly, greedy, pretentious, hypocritical, boastful, pompous, emotional, oversensitive, sentimental, fearful, anxious, vulnerable, brave, tough, independent, self-assured, stable, intellectual, creative, unconventional, innovative, shallow, absentminded\\
\item
  Outgoing, lively, extraverted, sociable, talkative, cheerful, active, shy, passive, withdrawn, introverted, quiet, reserved, patient, tolerant, peaceful, mild, agreeable, lenient, gentle, ill-tempered, stubborn, quarrelsome, choleric (easily angered), organized, disciplined, diligent, careful, thorough, precise, sloppy, negligent, reckless, lazy, irresponsible
\end{itemize}

Approaches to Studying Personality

\begin{itemize}
\tightlist
\item
  Personality (p.~458)\\
\item
  Idiographic approach (p.~458)\\
\item
  Serial killer\\
\item
  Person-centred\\
\item
  Nomothetic approach (p.~458)\\
\item
  Descriptive labels
\end{itemize}

The Trait Perspective

\begin{itemize}
\tightlist
\item
  Personality traits (p.~459)\\
\item
  Allport and Odbert\\
\item
  18,000 descriptors\\
\item
  Factor analysis (p.~459)
\end{itemize}

\includegraphics{assets/unit_5/slide_11.png}

\emph{Slide showing - The Big Five Personality Dimensions}

Beyond the Big Five: The Personality of Evil?

\begin{itemize}
\tightlist
\item
  Authoritarian Personality
\item
  HEXACO (p.~462)
\item
  Honesty-Humility
\item
  Right-wing authoritarianism (p.~463)
\item
  The Dark Triad (p.~462)
\item
  Machiavellianism
\item
  Psychopathy
\item
  Narcissism
\end{itemize}

Working the Scientific Literacy Model: Right-Wing Authoritarianism at the Group Level

\begin{itemize}
\tightlist
\item
  What do we know about RWA?\\
\item
  More RWA in society -- leads to prejudice, aggression\\
\item
  How can science determine how RWA affects groups?\\
\item
  Global Change Game\\
\item
  Can we critically evaluate this research?\\
\item
  External validity\\
\item
  Chance\\
\item
  Student participants\\
\item
  Why is this relevant?\\
\item
  21st century warning bell
\end{itemize}

\includegraphics{assets/unit_5/slide_15.png}

\emph{Slide showing - Personality Stability and Change over the Lifespan}

Personality Traits and States

\begin{itemize}
\tightlist
\item
  State (p.~466)\\
\item
  Four general aspects of situations

  \begin{itemize}
  \tightlist
  \item
    Locations\\
  \item
    Associations\\
  \item
    Activities\\
  \item
    Subjective states
  \end{itemize}
\end{itemize}

\includegraphics{assets/unit_5/slide_17.png}

\emph{Slide showing - Behavioural and Social-Cognitive Approaches to Personality}

Learning Objectives

\begin{itemize}
\tightlist
\item
  Know key terminology associated with cultural and biological approaches to personality.\\
\item
  Understand how evolutionary theories explain personality.\\
\item
  Apply your knowledge to arrive at accurate conclusions about the influences of biological and cultural factors on personality.\\
\item
  Analyze claims that males and females have fundamentally different personalities.\\
\item
  Analyze the genetic basis of personality.
\end{itemize}

Culture and Personality

\begin{itemize}
\tightlist
\item
  Personality structures in different cultures\\
\item
  Comparing personality traits between nations\\
\item
  Challenges in cross-cultural research\\
\item
  Response styles (p.~472)
\end{itemize}

\includegraphics{assets/unit_5/slide_20.png}

\emph{Slide showing - How Genes affect Personality}

Working the Scientific Literacy Model: From Molecules to Personality

\begin{itemize}
\tightlist
\item
  What do we know about specific genes and personality?

  \begin{itemize}
  \tightlist
  \item
    Genes code for brain chemicals related to personality

    \begin{itemize}
    \tightlist
    \item
      Serotonin\\
    \end{itemize}
  \end{itemize}
\item
  How do scientists study genes and personality?
\end{itemize}

\includegraphics{assets/unit_5/slide_22.png}

\emph{Slide showing - Genes, Serotonin, and Personality}

\begin{itemize}
\tightlist
\item
  Can we critically evaluate this evidence?

  \begin{itemize}
  \tightlist
  \item
    Multiple genes interact with environment\\
  \item
    Should not infer causality\\
  \end{itemize}
\item
  Why is this relevant?

  \begin{itemize}
  \tightlist
  \item
    Understanding of biological basis of disorders\\
  \item
    Better treatments
  \end{itemize}
\end{itemize}

The Role of Evolution in Personality: Animal Behaviour

\begin{itemize}
\tightlist
\item
  Big Five traits found in a number of species

  \begin{itemize}
  \tightlist
  \item
    Parus major

    \begin{itemize}
    \tightlist
    \item
      Some bold, others shy\\
    \end{itemize}
  \end{itemize}
\item
  Chimpanzees

  \begin{itemize}
  \tightlist
  \item
    Extraversion, conscientiousness, and agreeableness\\
  \end{itemize}
\item
  Octopuses

  \begin{itemize}
  \tightlist
  \item
    Activity, reactivity, and avoidance
  \end{itemize}
\end{itemize}

Why There Are So Many Different Personalities: The Evolutionary Explanation

\begin{itemize}
\tightlist
\item
  Core personality traits across cultures\\
\item
  Individuals select environments that match personality characteristics
\end{itemize}

MYTHS IN MIND: Men are from Mars, Women are from Venus

\begin{itemize}
\tightlist
\item
  Gender differences greatly exaggerated

  \begin{itemize}
  \tightlist
  \item
    Influenced by resource availability
  \end{itemize}
\end{itemize}

The Brain and Personality

\begin{itemize}
\tightlist
\item
  Four Humours (400 BC)
\item
  Phrenology (1700s)
\item
  Extraversion and arousal: brain regions (1967)
\item
  Approach/Inhibition model of motivation (1991)
\end{itemize}

Learning Objectives

\begin{itemize}
\tightlist
\item
  Know the key terminology related to psychodynamic and humanistic approaches to personality.\\
\item
  Understand how people use defense mechanisms to cope with conflicting thoughts and feelings.\\
\item
  Understand the developmental stages Freud used to explain the origins of personality.\\
\item
  Apply both psychodynamic and humanistic perspectives to explain personality.\\
\item
  Analyze whether projective tests are valid measures of personality.\\
\item
  Analyze the strengths and weaknesses of psychodynamic perspectives.
\end{itemize}

The Psychodynamic Perspective

\begin{itemize}
\tightlist
\item
  Unconscious processes and psychodynamics

  \begin{itemize}
  \tightlist
  \item
    Sigmund Freud (1800s)\\
  \item
    Key observations

    \begin{itemize}
    \tightlist
    \item
      Unconscious influences behaviour\\
    \item
      Personality forms in early childhood\\
    \item
      Mental representations shape behaviours
    \end{itemize}
  \end{itemize}
\end{itemize}

Unconscious Processes and Psychodynamics

\begin{itemize}
\tightlist
\item
  Unconscious mind (p.~483)

  \begin{itemize}
  \tightlist
  \item
    Impulses and drives\\
  \end{itemize}
\item
  Conscious mind (p.~483)

  \begin{itemize}
  \tightlist
  \item
    Aware of these
  \end{itemize}
\end{itemize}

\includegraphics{assets/unit_5/slide_31.png}

\emph{Slide showing - Personality Stability and Change over the Lifespan}

\includegraphics{assets/unit_5/slide_32.png}

\emph{Slide showing - Examples of Some Major Defence Mechanisms}

\includegraphics{assets/unit_5/slide_33.png}

\includegraphics{assets/unit_5/slide_34.png}

\includegraphics{assets/unit_5/slide_35.png}

\emph{Slide showing - Freud's Stages of Psychosexual Development}

\includegraphics{assets/unit_5/slide_36.png}

\emph{Slide showing - The Rorschach Inkblot Test}

\includegraphics{assets/unit_5/slide_37.png}

\emph{Slide showing - The Thematic Apperception Test}

\includegraphics{assets/unit_5/slide_38.png}

\emph{Slide showing - Figure Drawing as a Projective Test}

Working the Scientific Literacy Model: Perceiving Others as a Projective Test

\begin{itemize}
\tightlist
\item
  What do we know about the way people perceive others?

  \begin{itemize}
  \tightlist
  \item
    We make assumptions of others

    \begin{itemize}
    \tightlist
    \item
      May reflect our personality\\
    \end{itemize}
  \end{itemize}
\item
  How can scientists study how projection relates to personality?

  \begin{itemize}
  \tightlist
  \item
    Self-ratings on Big Five can been seen in perceptions of others\\
  \end{itemize}
\item
  Can we critically evaluate this research?

  \begin{itemize}
  \tightlist
  \item
    Small correlations\\
  \item
    Only with positive and negative attributions\\
  \end{itemize}
\item
  Why is this relevant?

  \begin{itemize}
  \tightlist
  \item
    Problems and controversy with projective tests

    \begin{itemize}
    \tightlist
    \item
      Need more objective measures
    \end{itemize}
  \end{itemize}
\end{itemize}

Alternatives to the Psychodynamic Approach

\begin{itemize}
\tightlist
\item
  Carl Jung (1875-1961)

  \begin{itemize}
  \tightlist
  \item
    Analytical psychology (p.~490)\\
  \item
    Personal, Collective unconscious (p.~490)\\
  \item
    Archetypes (p.~490)\\
  \end{itemize}
\item
  Alfred Adler (1870-1937) and Karen Horney (1885-1952)

  \begin{itemize}
  \tightlist
  \item
    Inferiority complex (p.~490)
  \end{itemize}
\end{itemize}

Humanistic Perspectives

\begin{itemize}
\tightlist
\item
  Carl Rogers, Abraham Maslow

  \begin{itemize}
  \tightlist
  \item
    Self-actualization (p.~491)\\
  \item
    Person-centred perspective (p.~491)
  \end{itemize}
\end{itemize}

\emph{Please note, the slides are intended to supplement the information found in your textbook. If you are having trouble viewing them, they can also be downloaded by scrolling to the bottom of the screen and clicking on the ``Unit 5- Slides'' link.}

\textbf{Read and Reflect}

\begin{itemize}
\tightlist
\item
  This Learning Activity is an opportunity to explore dreams and birth order, two of the constructs that help shape and reflect who we are as individuals. Keep in mind, though, there are many different theories and interpretations on the role dreams and birth order play.\\
\item
  The humanistic view, as it relates to personality, sees self-esteem as key ingredient for success in life. The articles in this activity illuminate how a healthy self-esteem can benefit personality development. It will also be helpful for preparation for discussion this week.\\
\item
  In this activity you will have the opportunity to analyze the power of the environment in influencing your perceived options. The concepts of Learned Helplessness vs.~Learned Optimism will be used to highlight how important environmental factors are.
\end{itemize}

\textbf{Practice, Read, and Reflect}

\begin{itemize}
\tightlist
\item
  In this activity you will have the opportunity to take a personality test (from a list of options). I would encourage you to interact with the results and ponder whether or not you think the test is accurate or inaccurate and whether or not the information provided seems unique to you or could it be applied to anyone given the right situation.
\end{itemize}

\textbf{Key Terms Quiz}

\begin{itemize}
\tightlist
\item
  Practice quiz to assess how well you know key terms from Chapter 12.\\
\item
  Not for formal evaluation.
\end{itemize}

\textbf{\emph{Learning Lab Preparation}}

\begin{itemize}
\tightlist
\item
  Each topic will provide a question or scenario for you to consider prior to attending your Learning Lab. Be sure to carefully consider each prompt as you will be expected to contribute to the group discussion.
\end{itemize}
\end{reflect}

\hypertarget{resources-6}{%
\subsection*{Resources}\label{resources-6}}
\addcontentsline{toc}{subsection}{Resources}

Here are some additional resources that will help you complete this unit:

\begin{itemize}
\tightlist
\item
  Krause, M., Corts, D., Smith, S. C., \& Dolderman, D. (2018). \emph{Revel for An Introduction to Psychological Science, 2nd Canadian Edition.} Pearson Ed.\\
\item
  Other resources will be provided online.
\end{itemize}

\hypertarget{psychoanalytic-view}{%
\section{Psychoanalytic View}\label{psychoanalytic-view}}

``We are sent into this world to acquire a personality and a character to take with us that can never be taken from us.'' \emph{(In a letter from a WW II Royal Air Force pilot to his mother)}

\hypertarget{personality-theories}{%
\subsection*{Personality Theories}\label{personality-theories}}
\addcontentsline{toc}{subsection}{Personality Theories}

In this chapter we consider the topic of personality theories. In doing this we examine several major perspectives. Within each of these there are many variations (e.g., Adler, Jung, and Freud within the psychoanalytic perspective). If you would like to know more about this area, you might be interested in taking the course ``Theories of Personality'' \emph{(Psychology 301 at TWU).}

Aside from all these more formal theories, each of us has an implicit personality theory. This is our personal idea of what people are like, including which personality characteristics go together. For example, we think of people who are attractive and intelligent as likable rather than unlikable. Our socialization leads us to assume certain traits go together. As you go through this chapter, you may find some approaches are closer to your own assumptions (implicit theory), and you may prefer these approaches to others. You may also find your implicit theory becomes more explicit. Finally, you may even modify your own theory in the face of new information.

\hypertarget{dreams}{%
\subsection*{Dreams}\label{dreams}}
\addcontentsline{toc}{subsection}{Dreams}

``And afterward, I will pour out my Spirit on all people. Your sons and daughters will prophesy, your old men will dream dreams, your young men will see visions'' \emph{(Joel 2:28)}.

God apparently spoke through dreams many times in the bible, and in Joel 2:28 He promised to do so in the future. Does God still (or again) speak through dreams? Are we missing out on a possible leading from God when we ignore our dreams? If you would like to read more on this subject you might be interested in books by Christians on this subject. For example, see John A. Sanford's Dreams, God's Forgotten Language, or Morton T. Kelsey's God, Dreams, and Revelation, or Abraham Schmitt's Before I Wake. The website below (cf.~Online Resources) contains some balanced advice. Also, voice your opinion or give your experience in the online discussion.

\hypertarget{birth-order}{%
\subsection*{Birth Order}\label{birth-order}}
\addcontentsline{toc}{subsection}{Birth Order}

You may be familiar with Adler's concept of the family constellation, which implies that position in the family has a powerful effect on goals, self-concept, and ways of relating to others.

As a test of this concept, think of a three-child family, all of whom were born within a few years (using larger families makes things too complicated). Try applying Adler's theory to predict which child in the family is likely to be:

\begin{itemize}
\tightlist
\item
  the best student\\
\item
  the most shy\\
\item
  the most helpful around the house\\
\item
  the most outgoing\\
\item
  the ``hell raiser''\\
\item
  the most ``helpless''\\
\item
  the charmer\\
\item
  the ``baby'' of the family
\end{itemize}

You will not always be right---but neither was Adler. The important thing is to note that Adler made psychology aware of the different expectations and social influences on different children in the same family. In some respects, children in the same position in different families are more alike than are children in differ­ent positions in the same family

\hypertarget{learning-activities-6}{%
\subsection*{Learning Activities}\label{learning-activities-6}}
\addcontentsline{toc}{subsection}{Learning Activities}

\begin{reflect}
\hypertarget{read-and-reflect-2}{%
\subsubsection*{Read and Reflect}\label{read-and-reflect-2}}
\addcontentsline{toc}{subsubsection}{Read and Reflect}

Having just introduced Personality Theories to the course, we begin to explore psychoanalytic views of human development. Dreams and birth order are but two of the constructs that help shape and reflect who we are as individuals- there are many different theories and interpretations on the role dreams and birth order play. Take a moment to consider the following articles:

\begin{itemize}
\tightlist
\item
  \href{https://www.cgg.org/index.cfm/fuseaction/Library.sr/CT/BQA/k/96/What-Is-Proper-Christian-Perspective-on-Dreams-Visions.htm}{\textbf{Christian Perspective on Dreams}}\\
\item
  \href{https://www.scientificamerican.com/article/ruled-by-birth-order/}{\textbf{Birth Order and Your Personality}}
\end{itemize}

\hypertarget{learning-lab-preparation-11}{%
\subsubsection*{Learning Lab Preparation}\label{learning-lab-preparation-11}}
\addcontentsline{toc}{subsubsection}{Learning Lab Preparation}

Prior to your Learning Lab, take some time to think about the following questions. You will be asked to share your thoughts in this week's Learning Lab:

\begin{itemize}
\tightlist
\item
  \textbf{\emph{Does God speak through dreams?}}\\
\item
  \textbf{\emph{What reasons do you have for your opinion? Can you give an example?}}
\end{itemize}
\end{reflect}

\hypertarget{humanistic-view}{%
\section{Humanistic View}\label{humanistic-view}}

\hypertarget{self-esteem}{%
\subsection*{Self-Esteem}\label{self-esteem}}
\addcontentsline{toc}{subsection}{Self-Esteem}

In 1890, William James gave a simple formula for self-esteem:

\includegraphics{assets/unit_5/Unit5_Topic2_Image.png}

\emph{James explained:}

Such a fraction may be increased as well by diminishing the denominator as by increasing the numerator. To give up pretensions is as blessed a relief as to get them gratified; and where disappointment is incessant and the struggle unending, this is what men will always do. . . . How pleasant is the day we give up striving to be young---or slenderThank Godwe say, those illusions are gone. \emph{(James, 1890) (Quoted from Invitation to Social Psychology by Ron Philipchalk)}

In his textbook, David Myers discusses the pros and cons of promoting self-esteem. Consider also the following:

\hypertarget{self-esteem-a-contrary-view}{%
\subsection*{Self-Esteem: A Contrary View}\label{self-esteem-a-contrary-view}}
\addcontentsline{toc}{subsection}{Self-Esteem: A Contrary View}

Self-esteem is a popular topic in North American society and researchers have expended a vast amount of effort studying it. Approximately 10,000 studies have explored the possible relationship between self-esteem and various human problems (Scheff et al., 1989). The assumption that low self-esteem is at the root of many personal and social problems has reached the level of a cultural truism. The California legislature (1990) has even established a task force to promote self-esteem, believing that enhanced self-esteem is a vaccine against drug abuse, teen-age pregnancy, welfare dependency, and other social ills. Sharon Neuman (1992) suggests that the need to enhance self-esteem has become a ``motherhood issue''---meaning that belief in its value is so widely accepted it goes unquestioned.

Recently, however, several authors have begun to challenge this focus on raising self-esteem. They point out that high self-esteem can sometimes be detrimental, if, for example, it leads people to set inappropriate, risky goals that are beyond their capabilities (Baumeister, et al., 1993a). In addition, the relationship between self-esteem and other measures of well-being is often weak or nonexistent (Burr \& Christensen, 1992; Hermans, 1992; Jackson, 1984). Perhaps most importantly, high self-esteem may be the result and not the cause of other positive tendencies (Alexander \& Baker, 1992).

Sharon Neuman (1992) argues that self-esteem must follow, not precede, real achievement. She claims that no number of self-esteem programs can replace the genuine satisfaction of a job well done. Wesley Burr and Clark Christensen (1992) suggest that the idea that we must esteem ourselves highly before we can value and love others is a ``Western myth.'' They add:

We would improve our thinking if we were to take the idea that ``we cannot love others until we love ourself'' and turn it 180 degrees to be: ``We cannot love ourself until we love others.'' We would further improve our thinking if we then concluded that this is a fairly irrelevant and unimportant idea anyway because it still implies that loving ourself is what is important. (p.~464)

Burr and Christensen suggest that self-esteem is a fruit and not a root of healthy emotional connections between people. \emph{(From Invitation to Social Psychology by Ron Philipchalk)}

\hypertarget{christianity-humanism}{%
\section{Christianity \& Humanism}\label{christianity-humanism}}

Some Christians object to the humanistic perspective because of its unqualified affirmation of human goodness and its apparent promotion of self-love. It is important to note, however, that humanistic psychology is not to be equated with the philosophy and social movement called humanism. Although it focuses on humans, so too do other approaches to personality. (The behavioral psychologist B.F. Skinner was a signatory to the Humanist Manifesto but a major opponent of humanistic psychology.) In his authoritative work on the history of psychology, Ernest Hilgard writes, ``Humanistic psychology as something contemporary is not to be confused with other forms of humanism, such as humanistic studies which refer to classical or liberal arts topics, or that form of humanism which sets itself against theological beliefs (1987, p.~504). In fact, J.I. Packer and Thomas Howard in their book Christianity: The true humanism (1985) argue that only Christianity gives a valid basis for assuming humans have value.

\hypertarget{learning-activities-7}{%
\subsection*{Learning Activities}\label{learning-activities-7}}
\addcontentsline{toc}{subsection}{Learning Activities}

\begin{reflect}
\hypertarget{read-and-reflect-3}{%
\subsubsection*{Read and Reflect}\label{read-and-reflect-3}}
\addcontentsline{toc}{subsubsection}{Read and Reflect}

As we explore a humanistic view, as it relates to personality, we start to see the development of self-esteem. This will be the focus of this unit's Learning Lab. In order to prepare for discussion this week, take a moment to consider the following resources:

\begin{itemize}
\tightlist
\item
  \href{https://kidshealth.org/en/kids/self-esteem.html}{\textbf{Kids Health: Self Esteem}}\\
\item
  \href{http://healthyselfesteem.org/}{\textbf{National Association for Self-Esteem}}\\
\item
  \href{https://www.apa.org/about/division/div32}{\textbf{Society for Humanistic Psychology}}
\end{itemize}

\emph{The purpose of these resources is to help you assess your self-esteem, help to build a healthy self-esteem, and to provide you with a credible site for learning more about Humanistic Psychology.}

\hypertarget{learning-lab-preparation-12}{%
\subsubsection*{Learning Lab Preparation}\label{learning-lab-preparation-12}}
\addcontentsline{toc}{subsubsection}{Learning Lab Preparation}

Prior to your Learning Lab, take some time to think about the following questions. You will be asked to share your thoughts in this week's Learning Lab:

\begin{itemize}
\tightlist
\item
  \textbf{\emph{If you wanted to study the psychologically healthiest individuals, whom would you study? How would you decide?}}
\end{itemize}
\end{reflect}

\hypertarget{trait-view}{%
\section{Trait View}\label{trait-view}}

\hypertarget{trait-theory}{%
\subsection*{Trait Theory}\label{trait-theory}}
\addcontentsline{toc}{subsection}{Trait Theory}

In the previous lesson we discussed implicit theories of personality. Many people's implicit theories include elements of trait theory. They believe that personality is made up of traits. For example, popular ideas of leadership and success often rest on this assumption (e.g., Traits for success).

Actually this view is quite old, going back to the ancient Greeks:

Almost 2,000 years ago a Greek physician named Galen suggested that there were four basic types of personality, based on four different ``humors,'' or fluids, in the body. Galen called these personality types sanguine, phlegmatic, choleric, and melancholic. When Eysenck combined his two basic personality scales, he found that he had a ``less humorous'' version of a classic theory. \emph{(from Understanding Human Behavior by Ron Philipchalk \& James McConnell)}

In psychology today there is a growing effort towards accurate measurement of personality traits, and investigation of the ways in which traits interact with the environment.

One of the most popular applications of trait theory is in the prediction of employee success. After looking at the ``Big 5,'' which trait do you think is the most important? Imagine you own a business and you have to make a quick decision to hire one of 5 very similar applicants. The only trait measure you have is each applicant's score on the ``Big 5.'' The only problem is that each one scores high on a different factor. Which one would you hire?

\hypertarget{measuring-personality}{%
\subsection*{Measuring Personality}\label{measuring-personality}}
\addcontentsline{toc}{subsection}{Measuring Personality}

The trait approach to personality is closely linked to the matter of psychological testing. The strength of this approach is that it provides quantifiable data for analysis and comparison. The weakness is the difficulty in measuring traits accurately. Remember our discussion of \emph{standardization, validity,} and \emph{reliability} of intelligence tests? Personality tests must also meet the same criteria. (An additional problem is that our situations guide us as well as our traits.) This is an important point to remember when you encounter the many so-called personality tests in the popular media.

\hypertarget{barnum-effect}{%
\subsection*{Barnum Effect}\label{barnum-effect}}
\addcontentsline{toc}{subsection}{Barnum Effect}

Do you have a desire to know what ``kind'' of person you are? This common desire can be used to manipulate you in a variety of ways. As discussed in the textbook, sometimes it is easier to make people believe that you are measuring their personality than it is to actually measure it. In fact, it is remarkably easy for people to be convinced that a personality profile describes them well. This phenomena is known as ``the Barnum effect,'' after the circus showman P. T. Barnum. The Barnum effect hearkens back to the late 1940s, when psychologist Bertram Forer gave research participants a personality test and then generated a personality description that subjects believed was based on their test responses. Even though all participants were given exactly the same personality description, they found the profile to be highly convincing and descriptive of them as an individual. When asked to rate how well the profile described them, on a scale ranging from 0 (very poor) to 5 (excellent), the average rating was an impressive 4.26 (Forer, 1949)You, too, can try this experiment. Read the personality description below and rate, on a scale ranging from 0 (very poor) to 5 (excellent), how well this description describes you:

\emph{You have a strong need for other people to like you and for them to ad­mire you. You have a tendency to be critical of yourself. You have a great deal of unused energy which you have not turned to your advantage. While you have personality weaknesses, you are generally able to compensate for them. Your sexual adjustment has presented some problems for you. Dis­ci­plined and controlled on the outside, you tend to be worrisome and in­se­cure inside. At times you have serious doubts as to whether you have made the right decision or done the right thing. You prefer a certain amount of change and variety and become dissatisfied when hemmed-in by restrictions and limitations. You pride yourself as being an independent thinker and do not accept other opinions without satisfactory proof. You have found it un­wise to be too frank in revealing yourself to others. At times you are extro­verted, affable, sociable, while at other times you are introverted, wary, and reserved. Some of your aspirations tend to be pretty unrealistic.}

\hypertarget{learning-activities-8}{%
\subsection*{Learning Activities}\label{learning-activities-8}}
\addcontentsline{toc}{subsection}{Learning Activities}

\begin{reflect}
\hypertarget{practice-read-and-reflect}{%
\subsubsection*{Practice, Read, and Reflect}\label{practice-read-and-reflect}}
\addcontentsline{toc}{subsubsection}{Practice, Read, and Reflect}

In this topic we have continued to develop our understanding of personality. This will be the focus of this unit's Learning Lab. In order to prepare for discussion this week, take a moment to consider the following resource:

\begin{itemize}
\tightlist
\item
  \href{https://www.queendom.com/}{\textbf{Queendom.com}}
\end{itemize}

\emph{Queendom.com} is a great resource for different kinds of tests. You will be looking specifically at Personality tests on this site. When you go to www.queendom.com, click on the \textbf{\emph{``TESTS, QUIZZES \& POLLS''}} tab and select \textbf{TESTS.} From the \textbf{\emph{PERSONALITY TESTS}} list select ONE of the following personality tests:

\begin{itemize}
\tightlist
\item
  DISC Personality Test\\
\item
  Big Five Personality Test\\
\item
  Self-Control \& Self-Monitoring Test\\
\item
  Happiness Test\\
\item
  Locus Of Control \& Attributional Style Test
\end{itemize}

Then interact with the results and ponder whether or not you think the test is accurate or inaccurate and whether or not the information provided seems unique to you or could it be applied to anyone given the right situation.

\hypertarget{learning-lab-preparation-13}{%
\subsubsection*{Learning Lab Preparation}\label{learning-lab-preparation-13}}
\addcontentsline{toc}{subsubsection}{Learning Lab Preparation}

Prior to your Learning Lab, take some time to think about the following questions.
You will be asked to share your thoughts in this week's Learning Lab:

\begin{itemize}
\tightlist
\item
  \textbf{\emph{What is the role of heredity in personality?}}\\
\item
  \textbf{\emph{Is our personality inherited and set for life? Can it be modified?}}
\end{itemize}

Author Tim LaHaye has taken the Greek idea of types and body humors and adapted it to a Christian perspective incorporating the concept of spiritual gifts.

\begin{itemize}
\tightlist
\item
  \textbf{\emph{Does the bible support an approach to personality that links individual differences to inherited traits?}}\\
\item
  \textbf{\emph{What is the role of the Holy Spirit in relationship to traits? - Are these different ``gifts'' (I Cor. 12)? Different ``fruit (Gal 5:22)? Can they be changed?}}
\end{itemize}
\end{reflect}

\hypertarget{social-cognitive-view}{%
\section{Social-Cognitive View}\label{social-cognitive-view}}

\hypertarget{you-your-environment}{%
\subsection*{You + Your Environment}\label{you-your-environment}}
\addcontentsline{toc}{subsection}{You + Your Environment}

Social-Cognitive Theory (SCT) was originally formulated by Albert Bandura at Stanford University. The fundamental premise of SCT is that models serve as the basis for learning new behaviours and that learning occurs in a social context with a dynamic and reciprocal interaction of the person, environment, and behavior; this is known as reciprocal determinism. Its emphasis is on external and internal social reinforcement and how these factors shape and maintain behavior in different environments. The theory also recognizes the influence of a person's past experiences, which factor into whether behavioral action will occur. These past experiences serve as the primary reinforcements and expectancies, shaping whether or not a person will engage in a specific behavior and answering why a person engages in that behaviour.

In summary, the Social-Cognitive view recognizes the mutual interdependence (``reciprocal determinism'') of individual qualities and situations. While we do have individual patterns of behaviour, we exhibit these according to the way we perceive our situation.

\hypertarget{learning-activities-9}{%
\subsection*{Learning Activities}\label{learning-activities-9}}
\addcontentsline{toc}{subsection}{Learning Activities}

\begin{reflect}
\hypertarget{read-and-reflect-4}{%
\subsubsection*{Read and Reflect}\label{read-and-reflect-4}}
\addcontentsline{toc}{subsubsection}{Read and Reflect}

Who we are is, in large part, shaped by the environment in which live. This will be the focus of this unit's Learning Lab. In order to prepare for discussion this week, take a moment to consider the following resources:

\begin{itemize}
\tightlist
\item
  \href{http://www.ldonline.org/article/6154/}{\textbf{Learned Helplessness}}\\
\item
  \href{https://web.stanford.edu/class/msande271/onlinetools/LearnedOpt.html}{\textbf{Learned Optimism Test}}
\end{itemize}

The above resources will help you to think about how your environment can impact your sense of agency (the ability to choose your behaviours to impact the environment) and how your sense of agency can influence your environment. This demonstrates the key idea of Social-Cognitive Theory, that you are in a constant, dynamic relationship with your environment and each factor is shaped by the other.

\hypertarget{key-terms-quiz}{%
\subsubsection*{Key Terms Quiz}\label{key-terms-quiz}}
\addcontentsline{toc}{subsubsection}{Key Terms Quiz}

In order to review some of the major terms from Chapter 12 in your textbook, take the following unmarked quiz. Although you will not be evaluated on these terms, they will assist you in the assessments for this course:

{[}h5p id=``81''{]}

\hypertarget{learning-lab-preparation-14}{%
\subsubsection*{\texorpdfstring{\textbf{Learning Lab Preparation}}{Learning Lab Preparation}}\label{learning-lab-preparation-14}}
\addcontentsline{toc}{subsubsection}{\textbf{Learning Lab Preparation}}

Prior to your Learning Lab, take some time to think about the following questions. You will be asked to share your thoughts in this week's Learning Lab:

\emph{Research indicates that an internal locus of control is associated with more positive psychological outcomes than an external locus of control.}

\begin{itemize}
\tightlist
\item
  \textbf{\emph{Does this mean that Christians (and others who believe in God) are less psychologically healthy because they believe an external force (God) is in control?}}
\end{itemize}
\end{reflect}

\hypertarget{assessment-5}{%
\section*{Assessment}\label{assessment-5}}
\addcontentsline{toc}{section}{Assessment}

\begin{assessment}
While there is no ``formal'' assignment that you will be responsible for submitting for Unit 5, you will be expected to participate in discussion during your Learning Lab. Your facilitator will be providing a participation mark based on your contributions.

Below is some information to consider prior to attending your Learning Lab:

\hypertarget{learning-lab}{%
\subsubsection*{\texorpdfstring{\textbf{Learning Lab}}{Learning Lab}}\label{learning-lab}}
\addcontentsline{toc}{subsubsection}{\textbf{Learning Lab}}

Your facilitator will be providing a participation mark based on your contributions for this unit. Below is some information to consider prior to attending your Learning Lab:

\emph{Active participation in group exercises, reflection, and critical discourse is an essential component of this course. You are expected to show respect for all members of the course, both in your speech and actions. Contribute by actively observing and listening, raising thoughtful questions, examining relevant issues, building on others' ideas, analyzing and evaluating the group's thinking, synthesizing key points, and expanding the group's perspectives. Take care not to dominate a conversation, giving space for others to speak. When in small groups help maintain the focus, flow, and quality of conversations, and take the initiative to invite others (particularly those who are quiet) to speak.}

\textbf{Rubric for Participation in Learning Labs}

\begin{longtable}[]{@{}
  >{\raggedright\arraybackslash}p{(\columnwidth - 4\tabcolsep) * \real{0.3333}}
  >{\raggedright\arraybackslash}p{(\columnwidth - 4\tabcolsep) * \real{0.3333}}
  >{\raggedright\arraybackslash}p{(\columnwidth - 4\tabcolsep) * \real{0.3333}}@{}}
\toprule\noalign{}
\begin{minipage}[b]{\linewidth}\raggedright
Emerging (0-64\%)
\end{minipage} & \begin{minipage}[b]{\linewidth}\raggedright
Developing (65-89\%)
\end{minipage} & \begin{minipage}[b]{\linewidth}\raggedright
Mastering (90-100\%)
\end{minipage} \\
\midrule\noalign{}
\endhead
\bottomrule\noalign{}
\endlastfoot
Nvr to almost never: Demonstrates active listening (as indicated by disengaged body language and no to rare comments that build on others' remarks),Initiates any contributions in class or small groups, Makes insightful or constructive comments, Helps maintain a supportive space for others to speak. & Sometimes to fairly often: Demonstrates active listening (as indicated by somewhat to often engaged body language and comments that build on others' remarks), Initiates a contribution at least once in a class or small group discussion; Makes insightful or constructive comments, Helps maintain a supportive space for others to speak. & Very often to nearly always: Demonstrates active listening (as indicated by fully engaged body language and comments that build on others' remarks), Initiates more than one contribution in a class or small group discussion, Makes insightful or constructive comments, Creates a space for others to speak and takes initiative to include others. \\
\end{longtable}
\end{assessment}

\hypertarget{checking-your-learning-4}{%
\section*{Checking your Learning}\label{checking-your-learning-4}}
\addcontentsline{toc}{section}{Checking your Learning}

\begin{progress}
Before you move on to the next unit, check that you are able to:

\begin{itemize}
\tightlist
\item
  Know the key terminology associated with contemporary approaches, cultural and biological approaches, and psychodynamic and humanistic approaches to personality.\\
\item
  Describe the behaviourist and social-cognitive views, evolutionary theories, and Freudian developmental and defensive explanations of personality.\\
\item
  Apply the Big Five personality traits, psychodynamic, and humanistic perspectives to understand personality.\\
\item
  Analyze the roots of violence and prejudice, roles of personality traits and psychological and physical states, and the genetic basis of personality in determining behaviour.\\
\item
  Analyze claims that males and females have fundamentally different personalities and whether projective tests are valid measures of personality.
\end{itemize}
\end{progress}

\hypertarget{social-psychology---part-1}{%
\chapter{Social Psychology - Part 1}\label{social-psychology---part-1}}

\hypertarget{overview-5}{%
\section*{Overview}\label{overview-5}}
\addcontentsline{toc}{section}{Overview}

No single factor has more impact on shaping the quality and direction of your life over virtually every aspect of your existence than relationships. In a way, everyone is a social psychologist. People constantly form ideas/opinions about why a person or a group of people are acting the way they are. Social psychology is similar to what people find themselves naturally doing except it uses scientific tools to study these hypotheses (the ideas/opinions). Social psychologist are primarily interested in examining the different ways that individual's effect groups of people and the way groups of people effect individual's. In this unit you will be introduced to some historical studies and main concepts in this field. Topics covered in this chapter include social influences on behaviour, social impressions and judgements, the relationship between attitudes and behaviour, and effective communication. Unit 6 (Part 1) looks at attributes and actions, conformity, obedience, and compliance. Unit 7 (Part 2) examines prejudice and aggression, and, though not specifically covered in the textbook, yet important to social psychology, attraction and altruism.

\hypertarget{topics-5}{%
\subsection*{Topics}\label{topics-5}}
\addcontentsline{toc}{subsection}{Topics}

This unit is divided into the following topics:

\begin{enumerate}
\def\labelenumi{\arabic{enumi}.}
\tightlist
\item
  Attributes and Actions\\
\item
  Conformity\\
\item
  Obedience\\
\item
  Compliance
\end{enumerate}

\hypertarget{learning-outcomes-5}{%
\subsection*{Learning Outcomes}\label{learning-outcomes-5}}
\addcontentsline{toc}{subsection}{Learning Outcomes}

By the end of this unit, student's will be able to:

\begin{itemize}
\tightlist
\item
  Define, and apply, the key terminology associated with social influence, social cognition, and on attitudes, behaviour, and effective communication.\\
\item
  Describe why individuals conform to others' behaviours, how individuals and groups can influence behaviours, and how we form first impressions and how these impressions influence us.\\
\item
  Apply your knowledge of the bystander effect to ensure that you will be helped if you are in an emergency, of social cognition to help overcome prejudice and discrimination, and of the central route to describe how a message should be designed.\\
\item
  Analyze whether guards who participate in abuse are inherently bad people, or if their behaviour is the product of social influences and whether people who commit discriminatory acts are necessarily prejudiced.\\
\item
  Describe how behaviours influence attitudes in terms of cognitive dissonance theory.
\end{itemize}

\hypertarget{activity-checklist-5}{%
\subsection*{Activity Checklist}\label{activity-checklist-5}}
\addcontentsline{toc}{subsection}{Activity Checklist}

\begin{reflect}
Here is a checklist of learning activities you will benefit from in completing this unit. You may find it useful for planning your work:

\textbf{Read and Reflect}

\begin{itemize}
\item
  Read \emph{Krause et al.~(2021). Revel for An Introduction to Psychological Science, 3rd Canadian Edition}
\item
  Review \emph{Unit 6 - Slides}
\end{itemize}

CLICK HERE

An Introduction to Psychological Science - Chapter 13 - Social Psychology

Biblical Word Study

\textsuperscript{2} John 1:6~(NIV)\\
\textsuperscript{6} And this is love: that we \emph{walk} in obedience to his commands. As you have heard from the beginning, his command is that you \emph{walk} in love.\\
Walk -- Gr.\(\pi\epsilon\rho\iota\pi\alpha\tau\epsilon\omega\) (peripateō) -- to tread about, walk about, and generally, to walk, to be walking; *to make one's way, make progress;

to frequent\emph{, stay in, a place;
}to regulate one's life, to conduct one's self*, of the standard to which one governs his life; denoting either the state in which one is living, or the virtue or vice to which he is given.

\begin{itemize}
\tightlist
\item
  Questions to ask those you live and work with:

  \begin{itemize}
  \tightlist
  \item
    What is my effect on you?
  \item
    What am I like to live with? Work with?
  \item
    What is something that is difficult for you to bring up with me?
  \end{itemize}
\item
  A question to ask your self:

  \begin{itemize}
  \tightlist
  \item
    Who am I in relationship with others?
  \item
    Questions like these can open the space to explore fear, worthlessness, shame and isolation, and can reveal strengths, gratitude, peace, joy, benefits, and maturity.
  \end{itemize}
\end{itemize}

True or False?

\begin{itemize}
\tightlist
\item
  T F 1. Compared with people in Western countries, those in East Asian cultures are more sensitive to situational influences on behaviour.\\
\item
  T F 2. To change people's racist behaviours, we first need to change their racist attitudes.\\
\item
  T F 3. Chimps are more likely to yawn after observing another chimp yawn.\\
\item
  T F 4. Most people would refuse to obey an authority figure who told them to hurt an innocent person.\\
\item
  T F 5. Studies of college and professional athletic events indicate that home teams win about 6 in 10 games.\\
\item
  T F 6. Individuals pull harder in a team tug-of-war than when they pull in a one-on-one tug-of-war.\\
\item
  T F 7. The higher the morale and harmony of a social group, the more likely are its members to make a good decision.\\
\item
  T F 8. Researchers project that, other things being equal, global warming of 4 degrees Fahrenheit (or about 2 degrees centigrade) would induce more than 50,000 additional assaults and murders in the United States alone.\\
\item
  T F 9. From research on liking and loving, it is clear that opposites do attract.\\
\item
  T F 10. We are less likely to offer help to a stranger if other bystanders are present.
\end{itemize}

\includegraphics{assets/unit_6/slide_7.png}

\emph{Slide showing - Modules for this chapter}

Learning Objectives

\begin{itemize}
\tightlist
\item
  Know the key terminology associated with social influence.\\
\item
  Understand why individuals conform to others' behaviours.\\
\item
  Understand how individuals and groups can influence behaviours.\\
\item
  Apply your knowledge of being a passive bystander or active altruist to understand your own likeliness to help.\\
\item
  Analyze whether people who harm others are fundamentally hurtful, mean people or if their behaviour is the product of social influences.
\end{itemize}

The Person and the Environment

\begin{itemize}
\tightlist
\item
  Kurt Lewin (1936): Behaviour is a function of the Person and the Environment

  \begin{itemize}
  \tightlist
  \item
    B = f(P,E)
  \end{itemize}
\end{itemize}

Mimicry, Norms, and Roles

\begin{itemize}
\tightlist
\item
  Mimicry (p.~495)

  \begin{itemize}
  \tightlist
  \item
    Important social and functional reasons for doing so\\
  \end{itemize}
\item
  Social Norms (p.~495)

  \begin{itemize}
  \tightlist
  \item
    Unwritten guidelines for how to behave in social contexts\\
  \end{itemize}
\item
  Ostracism (p.~496)

  \begin{itemize}
  \tightlist
  \item
    Powerful form of social pressure\\
  \end{itemize}
\item
  The Stanford Prison Study (Zimbardo)\\
\item
  Social roles (p.~496)
\end{itemize}

Group Dynamics: Social Loafing and Social Facilitation

\begin{itemize}
\tightlist
\item
  Social loafing (p.~497)

  \begin{itemize}
  \tightlist
  \item
    When individuals put less effort into tasks when working with others\\
  \end{itemize}
\item
  Social facilitation (p.~498)

  \begin{itemize}
  \tightlist
  \item
    When one's performance is affected by the presence of others
  \end{itemize}
\end{itemize}

\includegraphics{assets/unit_6/slide_13.png}

\emph{Slide showing - The Asch Conformity Experiment}

\includegraphics{assets/unit_6/slide_14.png}

\emph{Slide showing - Personal and Situational Factors Contribute to Conformity}

Groupthink

\begin{itemize}
\tightlist
\item
  Groupthink (p.~499)

  \begin{itemize}
  \tightlist
  \item
    When group members tend toward the same ideas to minimize conflict
  \end{itemize}
\end{itemize}

\includegraphics{assets/unit_6/slide_16.png}

\emph{Slide showing - The Milgram Experiment}

\includegraphics{assets/unit_6/slide_17.png}

\emph{Slide showing - The Milgram Experiment 2}

Obedience to Authority: The Milgram Experiment

\begin{itemize}
\tightlist
\item
  Variants of the experiment

  \begin{itemize}
  \tightlist
  \item
    Change of clothing\\
  \item
    Different location\\
  \item
    Proximity of experimenter and ``learner''\\
  \item
    Confederate dissenters
  \end{itemize}
\end{itemize}

The Bystander Effect

\begin{itemize}
\tightlist
\item
  The bystander effect (p.~503)\\
\item
  Diffusion of responsibility (p.~503)
\end{itemize}

\includegraphics{assets/unit_6/slide_20.png}

\emph{Slide showing - Diffusion of Responsibility}

Working the Scientific Literacy Model: The Bystander Effect

\begin{itemize}
\tightlist
\item
  What do we know about the bystander effect?\\
\item
  How can science study the bystander effect?\\
\item
  Latané \& Darley (1968)

  \begin{itemize}
  \tightlist
  \item
    Normative influences\\
  \item
    Informational influences\\
  \item
    Diffusion of responsibility\\
  \end{itemize}
\item
  Can we critically evaluate this evidence?

  \begin{itemize}
  \tightlist
  \item
    Bystander effects do happen\\
  \item
    But, often, people do help, even when it puts them at risk\\
  \end{itemize}
\item
  Why is this relevant?

  \begin{itemize}
  \tightlist
  \item
    Programs to encourage helping
  \end{itemize}
\end{itemize}

\includegraphics{assets/unit_6/slide_23.png}

\emph{Slide showing - Acts of Altruism}

\includegraphics{assets/unit_6/slide_24.png}

\emph{Slide showing - Acts of Altruism}

Learning Objectives

\begin{itemize}
\tightlist
\item
  Know the key terminology associated with social cognition.\\
\item
  Understand how we form first impressions and how these impressions influence us.\\
\item
  Apply your knowledge of attributions and biases to better understand how you tend to perceive yourself and others.\\
\item
  Analyze whether people who commit discriminatory acts are explicitly prejudiced.
\end{itemize}

Two Types of Cognitive Process

\begin{itemize}
\tightlist
\item
  Explicit processes (p.~507)\\
\item
  Implicit processes (p.~507)\\
\item
  Dual-process models (p.~507)
\end{itemize}

TPerson Perception

\begin{itemize}
\tightlist
\item
  Person perception (p.~507)

  \begin{itemize}
  \tightlist
  \item
    Schemas\\
  \item
    Thin slices of behaviour (p.~507-508)
  \end{itemize}
\end{itemize}

Self-Fulfilling Prophecies and Other Consequences of First Impressions

\begin{itemize}
\tightlist
\item
  Self-fulfilling prophecy (p.~508)

  \begin{itemize}
  \tightlist
  \item
    e.g., teachers' expectations of students
  \end{itemize}
\end{itemize}

The Self in the Social World

\begin{itemize}
\tightlist
\item
  Projecting the self onto others: False consensus and naïve realism (p.~509)\\
\item
  Self-serving biases (p.~510)

  \begin{itemize}
  \tightlist
  \item
    These arise out of a need to feel good about ourselves
  \end{itemize}
\end{itemize}

\includegraphics{assets/unit_6/slide_30.png}

\emph{Slide showing - Internal and External Attributions}

Attributions

\begin{itemize}
\tightlist
\item
  Fundamental attribution error (p.~510)

  \begin{itemize}
  \tightlist
  \item
    Cultural phenomenon\\
  \end{itemize}
\item
  Ingroups and outgroups (p.~511)\\
\item
  Ingroup bias (p.~511)
\end{itemize}

\includegraphics{assets/unit_6/slide_32.png}

\emph{Slide showing - Emotional Effects of Attribution}

Stereotypes, Prejudice, and Discrimination

\begin{itemize}
\tightlist
\item
  Stereotype (p.~512)\\
\item
  Prejudice (p.~512)\\
\item
  Discrimination (p.~512)
\end{itemize}

Prejudice in a Politically Correct World?

\begin{itemize}
\tightlist
\item
  Political correctness

  \begin{itemize}
  \tightlist
  \item
    Is it a neutral term?\\
  \end{itemize}
\item
  Prejudice is still prevalent\\
\item
  Implicit vs.~explicit processes in prejudice
\end{itemize}

MYTHS IN MIND: Are Only Negative Aspects of Stereotypes Problematic?

\begin{itemize}
\tightlist
\item
  ``Well-intended'' stereotypes

  \begin{itemize}
  \tightlist
  \item
    Gender

    \begin{itemize}
    \tightlist
    \item
      Hostile sexism\\
    \item
      Benevolent sexism
    \end{itemize}
  \end{itemize}
\end{itemize}

Working the Scientific Literacy Model: Explicit Versus Implicit Measures of Prejudice

\begin{itemize}
\tightlist
\item
  What do we know about measuring prejudice?

  \begin{itemize}
  \tightlist
  \item
    Explicit prejudice\\
  \item
    Implicit prejudice\\
  \end{itemize}
\item
  How can science study implicit prejudice?

  \begin{itemize}
  \tightlist
  \item
    The I A T (p.~514)
  \end{itemize}
\end{itemize}

\includegraphics{assets/unit_6/slide_37.png}

\emph{Slide showing - Working the Scientific Literacy Model: Explicit Versus Implicit Measures of Prejudice }

\begin{itemize}
\tightlist
\item
  Can we critically evaluate this evidence?

  \begin{itemize}
  \tightlist
  \item
    Validity of results

    \begin{itemize}
    \tightlist
    \item
      Knowledge of stereotypes vs.~personal stereotypes\\
    \end{itemize}
  \end{itemize}
\item
  Why is this relevant?

  \begin{itemize}
  \tightlist
  \item
    Stereotypes about every social status
  \end{itemize}
\end{itemize}

PSYCH @ The Law Enforcement Academy

\begin{itemize}
\tightlist
\item
  Video simulations

  \begin{itemize}
  \tightlist
  \item
    Mistaken shootings of innocent Black people\\
  \end{itemize}
\item
  Amadou Diallo

  \begin{itemize}
  \tightlist
  \item
    Mistakenly shot to death\\
  \end{itemize}
\item
  Police training

  \begin{itemize}
  \tightlist
  \item
    Shoot/do not shoot decisions
  \end{itemize}
\end{itemize}

Improving Intergroup Relations

\begin{itemize}
\tightlist
\item
  How do we overcome implicit processes?

  \begin{itemize}
  \tightlist
  \item
    Reprogramming through practice\\
  \item
    Contact hypothesis (p.~516)
  \end{itemize}
\end{itemize}

Learning Objectives

\begin{itemize}
\tightlist
\item
  Know the key terminology in research on attitudes, behaviour, and effective communication.\\
\item
  Understand how behaviours influence attitudes in terms of cognitive dissonance theory.\\
\item
  Apply your knowledge of cognitive dissonance to see how well your beliefs match your behaviours.\\
\item
  Analyze the difficulties communicators face in trying to convince the public to take action on important social and political issues.
\end{itemize}

Changing People's Behaviours

\begin{itemize}
\tightlist
\item
  How can we address climate change?

  \begin{itemize}
  \tightlist
  \item
    Technological\\
  \item
    Legal\\
  \item
    Economic\\
  \item
    Social
  \end{itemize}
\end{itemize}

\includegraphics{assets/unit_6/slide_43.png}

\emph{Slide showing - Central and Peripheral Routes to Persuasion}

Using the Central Route Effectively

\begin{itemize}
\tightlist
\item
  Make it personal

  \begin{itemize}
  \tightlist
  \item
    Construal-level theory (p.~520)\\
  \end{itemize}
\item
  Value Appeals
\end{itemize}

Working the Scientific Literacy Model: The Identifiable Victim Effect

\begin{itemize}
\tightlist
\item
  What do we know about communicating about tragedy and danger?
\item
  How can science explain the identifiable victim effect?

  \begin{itemize}
  \tightlist
  \item
    The experiential system (p.~522)\\
  \item
    The analytic system (p.~522)\\
  \end{itemize}
\item
  Can we critically evaluate this evidence?\\
\item
  Why is this relevant?

  \begin{itemize}
  \tightlist
  \item
    Connect issue to values of audience
  \end{itemize}
\end{itemize}

Preaching or Flip-Flopping One-Sided vs.~Two-Sided Messages

\begin{itemize}
\tightlist
\item
  Should you be ''preachy'' or a ``flip-flopper''?\\
\item
  Two-sided seen as more trustworthy

  \begin{itemize}
  \tightlist
  \item
    Attitude Inoculation (p.~523)
  \end{itemize}
\end{itemize}

Using the Peripheral Route Effectively

\begin{itemize}
\tightlist
\item
  Authority\\
\item
  Liking\\
\item
  Social validation\\
\item
  Reciprocity

  \begin{itemize}
  \tightlist
  \item
    Door-in-the-face technique (p.~524)\\
  \end{itemize}
\item
  Consistency

  \begin{itemize}
  \tightlist
  \item
    Foot-in-the-door technique (p.~524)
  \end{itemize}
\end{itemize}

The Attitude-Behaviour Feedback Loop

\begin{itemize}
\tightlist
\item
  Cognitive Dissonance Theory (p.~525)

  \begin{itemize}
  \tightlist
  \item
    Doomsday cults\\
  \item
    Lying for \$1 vs.~\$20
  \end{itemize}
\end{itemize}

\emph{Please note, the slides are intended to supplement the information found in your textbook. If you are having trouble viewing them, they can also be downloaded by scrolling to the bottom of the screen and clicking on the ``Unit 6 - Slides'' link.}

\begin{itemize}
\tightlist
\item
  This activity is an opportunity to examine attribution theory and cognitive dissonance. Put simply, attribution theory asserts that people explain other people's behaviour by looking at either their personal dispositions or their situation. Which lens you use to view people through powerfully shapes your thoughts and actions towards them. Next, cognitive dissonance involves mental discomfort when your attitudes and actions are not in alignments. The article on cognitive dissonance shares are three factors that influence the level of dissonance a person may feel.\\
\item
  Social influence is everywhere; hence, it is important to recognize when you are being influenced by propaganda, brainwashing, or other manipulative messages. The first article in this activity allows you to interact with some facts and uses of social influence, which can increase your ability to recognize when these techniques are being used on, or by, you. The second article looks more in-depth than your textbook at the concepts of conformity and obedience and lays out different examples of each of these topics.\\
\item
  This Activity focuses on conformity, compliance, and obedience, and a consideration of ethical principles. Doing research that involves the use of deception is a highly debated issue within psychology. The articles in this activity will help you to become more knowledgeable about Stanley Milgram and research that looks at obedience. There is also a resource that discusses what the ethics psychologists must follow along with a decision-making process when facing an ethical dilemma article. \emph{Do not feel compelled to read every article extensively (though, they are all interesting), a review of the material should be sufficient to answer the Learning Lab questions.}\\
\item
  In this activity you will gain further knowledge about strategies that are employed to promote compliance. The first article draws attention to the myriad ways social media is, for the good and for the bad, shaping culture today. The second article describes the difference between what constitutes mind control versus social conditioning. This information is certain to be eye-opening.
\end{itemize}

\textbf{\emph{Learning Lab Preparation}}

\begin{itemize}
\tightlist
\item
  Each topic will provide a question or scenario for you to consider prior to attending your Learning Lab. Be sure to carefully consider each prompt as you will be expected to contribute to the group discussion.
\end{itemize}
\end{reflect}

\hypertarget{resources-7}{%
\subsection*{Resources}\label{resources-7}}
\addcontentsline{toc}{subsection}{Resources}

Here are the resources you will need to complete this unit:

\begin{itemize}
\tightlist
\item
  Krause, M., Corts, D., Smith, S. C., \& Dolderman, D. (2018). \emph{Revel for An Introduction to Psychological Science, 2nd Canadian Edition.} Pearson Ed.\\
\item
  Other resources will be provided online.
\end{itemize}

\hypertarget{attributes-and-actions}{%
\section{Attributes and Actions}\label{attributes-and-actions}}

\hypertarget{the-ultimate-attribution-error}{%
\subsection*{The Ultimate Attribution Error}\label{the-ultimate-attribution-error}}
\addcontentsline{toc}{subsection}{The Ultimate Attribution Error}

If you study attribution further, perhaps in a ``Social Psychology'' course, you find that there are several errors in attribution common to our perceptions of behavior. The ultimate attribution error is a combination of biases that influences our interpretation of behavior in our friends and our foes. The ultimate attribution error is our tendency to use internal attributions to explain socially desirable actions by members of our own group and socially undesirable actions by members of an opposing group. The same bias leads us to use external attributions to explain socially undesirable actions of our own group and socially desirable actions of our foes (Hewstone, 1988; Pettigrew, 1979, 1980 ; Taylor and Jaggi, 1974 ).

A study by Donald Taylor and V. Jaggi (1974) illustrates this bias among south Indian subjects. Taylor and Jaggi found that Hindu subjects attributed socially desirable actions to the good character of Hindus, but attributed the same actions in Muslims to environmental circumstances. In addition, Hindu subjects blamed socially undesirable actions on Hindus' circumstances and Muslims' character. We commit the ultimate attribution error when we say, ``Our soldiers fight because they have to'' (external attribution), but ``Our enemies fight because they are warlike'' (internal attribution). cf galleys And, ``Our soldiers are naturally brave''; ``Our enemies are forced to do daring deeds.''

The ultimate attribution error tends to perpetuate our false stereotypes and prejudices. Our biased interpretations have a plausible ring because the reasons for our enemies' actions are usually somewhat ambiguous and open to various interpretations. In addition, our friends fall into the same error and support our views. Consequently, we continue to see behavior the way we expect or would like it to be, and people we don't like never get a chance to prove us wrong. \emph{(from Invitation to Social Psychology by Ronald Philipchalk)}

\hypertarget{actions-and-attitudes}{%
\subsection*{Actions and Attitudes}\label{actions-and-attitudes}}
\addcontentsline{toc}{subsection}{Actions and Attitudes}

``You can't legislate morality'' is a commonsense chestnut that social psycholo­gists have found not to be true. Attitudes do follow behavior. Eliot Aronson makes a persuasive argument that racial prejudiced attitudes and behaviors in the have actually responded dramatically and positively to civil rights legisla­tion and to court actions in the last several decades. In addition, he discusses sev­eral other methods that have been successfully used to reduce prejudice and discrimination, including his ``jigsaw'' technique, designed for use with grade school children. \emph{Aronson, E. (1992). The social animal (6th ed.). New York: W. H. Freeman.}

\hypertarget{learning-activity-5}{%
\subsection*{Learning Activity}\label{learning-activity-5}}
\addcontentsline{toc}{subsection}{Learning Activity}

\hypertarget{read-and-reflect-5}{%
\subsubsection*{Read and Reflect}\label{read-and-reflect-5}}
\addcontentsline{toc}{subsubsection}{Read and Reflect}

Attributes, and our perception of attributes, has been the focus of this lesson. This will be the focus of this unit's Learning Lab. In order to prepare for discussion this week, take a moment to consider the following resources:

\begin{itemize}
\tightlist
\item
  \href{http://healthyinfluence.com/wordpress/steves-primer-of-practical-persuasion-3-0/thinking/attribution/}{\textbf{Attribution Theory}}\\
\item
  \href{https://www.simplypsychology.org/cognitive-dissonance.html}{\textbf{Cognitive Dissnonance}}
\end{itemize}

These articles bring to light phenomena that tend to occur automatically (without conscious processing) for most people. Now that the concepts of attribution and cognitive dissonance are in your awareness it could be valuable to thoughtfully consider which attribution style you tend towards (dispositional or situational) and how that style influences how you treat others and how others respond to you. And, then ask, what am I doing with my cognitive dissonance? Do I tend to change my attitudes to fit my behaviours or do I adjust my behaviours to be alignment with my attitudes, values, and beliefs?

\hypertarget{learning-lab-preparation-15}{%
\subsubsection*{Learning Lab Preparation}\label{learning-lab-preparation-15}}
\addcontentsline{toc}{subsubsection}{Learning Lab Preparation}

Prior to your Learning Lab, take some time to think about the following scenario. You will be asked to share your thoughts in this week's Learning Lab:

\begin{itemize}
\tightlist
\item
  \textbf{\emph{Describe a time when you were forced to act in a certain way, and then later found you had come to enjoy what you were doing. That is, your attitude followed your actions.}}\\
\item
  \textbf{\emph{Could this happen with religious attitudes and beliefs too?}}
\end{itemize}

\hypertarget{conformity}{%
\section{Conformity}\label{conformity}}

``Oh, God, the terrible tyranny of the majority.'' \emph{(from Fahrenheit 451 by Ray Bradbury, p.~97)}

\hypertarget{conformity-vs.-compliance-obedience}{%
\subsection*{Conformity, vs.~Compliance \& Obedience}\label{conformity-vs.-compliance-obedience}}
\addcontentsline{toc}{subsection}{Conformity, vs.~Compliance \& Obedience}

In this lesson, and the next two, we will be looking at social influence. Because this is a large and interesting topic we will discuss three types of social influence separately, conformity, compliance, and obedience. Conformity, our topic for today, refers to going along with other people without anyone explicitly asking you to. Compliance refers to going along with the request of another person when you have a choice not to. Being influenced by a salesperson to buy something is an example of compliance. Obedience refers to going along with an order when no choice is offered, as in the army.

\hypertarget{norms}{%
\subsection*{Norms}\label{norms}}
\addcontentsline{toc}{subsection}{Norms}

Group expectations or norms exert a powerful influence on our behavior. We generally go along with what people expect of us not because we have to but because resistance is so uncomfortable. You can experience the power of group norms in controlling your behavior if you attempt to violate a norm for acceptable behavior (e.g., go the front of a line and edge in; ask someone for their seat on a bus or subway; sit close to a stranger in a public place when other seats are available).

\hypertarget{using-conformity}{%
\subsection*{Using Conformity}\label{using-conformity}}
\addcontentsline{toc}{subsection}{Using Conformity}

At the lowest level of social influence people are observed to go along with what other people are doing, often without being asked. Adults are frequently critical of adolescents for conforming to erratic youthful fashions, while they are often just as bound to their own conservative guidelines (why else would anyone where a necktie?). Conformity is a common and usually harmless part of everyday life; yet it can lead to complacent and even erroneous judgments. In a classic study of conformity approximately one-third of the subjects gave what they knew to be the wrong answer when four other ``subjects'' (actually confederates of the experimenter) gave the same wrong answer (Asch, 1952, 1955). Although frequently innocuous, conformity can be harmful when it is inappropriate.

Are Christians abusing the power of conformity when they pre- arrange for volunteers to ``go forward'' at an altar call in order to increase the number of non-Christians who will respond? Or are they merely being sensitive to the pressures that the non-Christian feels to conform to those who remain in their seats? Should Christians use this technique of influence? \emph{(from Psychology and Christianity by Ronald Philipchalk, p212)}

\hypertarget{conformity-culture}{%
\subsection*{Conformity \& Culture}\label{conformity-culture}}
\addcontentsline{toc}{subsection}{Conformity \& Culture}

Researcher Gary Lee and his associates found significant differences among 122 cultures in the emphasis given to conformity. Conformity differences de­velop as a result of socialization of children into parental values. Parents in cul­tures emphasizing conformity relied more heavily on physical punishment than parents in cultures emphasizing self-reliance.

\hypertarget{mach-test}{%
\subsection*{Mach Test}\label{mach-test}}
\addcontentsline{toc}{subsection}{Mach Test}

The ``Mach Test'' is a twenty-item inventory designed to reveal Machiavellian tenden­cies, a manipulative kind of social skill. Based on Machiavelli's writings, the in­stru­ment purports to identify people who, under social pressure, are able to retain their composure, while others are losing control. You can get an idea of your mach score, and thus presumably your tendency to use and be influenced by social influence, at:

\begin{itemize}
\tightlist
\item
  \href{https://openpsychometrics.org/tests/MACH-IV/}{\textbf{Mach Test}}
\end{itemize}

\hypertarget{learning-activity-6}{%
\subsection*{Learning Activity}\label{learning-activity-6}}
\addcontentsline{toc}{subsection}{Learning Activity}

\hypertarget{read-and-reflect-6}{%
\subsubsection*{Read and Reflect}\label{read-and-reflect-6}}
\addcontentsline{toc}{subsubsection}{Read and Reflect}

We investigated conformity, and the role it plays. This will be the focus of this unit's Learning Lab. In order to prepare for discussion this week, take a moment to consider the following resources:

\begin{itemize}
\tightlist
\item
  \href{http://www.workingpsychology.com/intro.html}{\textbf{Introduction to Social Influence}}\\
\item
  \href{https://webspace.ship.edu/cgboer/conformity.html}{\textbf{Conformity AND Obedience}}
\end{itemize}

\emph{Due to the pervasiveness of social influence, it cannot be overstated how important it is to understand the concepts and techniques associated with it. The purpose of these resources is to allow you to grow in your ability to be persuasive and to recognize when tools of influence are being used on you. It will also allow you to be more mindful of what/who you are conforming to and why it is that you are doing so. If you are really adventurous you may even want to take this knowledge to help you or someone else adopt a new attitude, belief, or action.}

\hypertarget{learning-lab-preparation-16}{%
\subsubsection*{Learning Lab Preparation}\label{learning-lab-preparation-16}}
\addcontentsline{toc}{subsubsection}{Learning Lab Preparation}

Prior to your Learning Lab, take some time to think about the following questions. You will be asked to share your thoughts in this week's Learning Lab:

\begin{itemize}
\tightlist
\item
  \textbf{\emph{What pressures for conformity exist in your community? On your campus?}}\\
\item
  \textbf{\emph{What happens to people who do not conform?}}\\
\item
  \textbf{\emph{Is conformity good or bad? What would society be like without conformity?}}
\end{itemize}

\hypertarget{obedience}{%
\section{Obedience}\label{obedience}}

Obedience refers to the strongest level of social influence, the level at which an order is given and no choice is assumed. For example, we are expected to obey the law; we are not assumed to have a choice.

The study of obedience has proven somewhat unsettling for those with an optimistic view of human nature. Under orders, people have been willing to \textbf{(a)} subdue a helpless protesting individual with electric shock labeled extremely dangerous (Milgram, 1963); \textbf{(b)} administer a drug to a patient at double the recommended dosage, when the drug had not been approved by the hospital and when, contrary to regulations, the orders were given by an unfamiliar physician over the phone (Hofling et al., 1966), \textbf{(c)} place cute puppies in a restraining harness and give them strong electric shock (Sheridan and King, 1972). However, rather than uncovering a sadistic or evil type of individual these studies indicate that in the momentum of the situation, and under legitimate authority, people will perform acts which they would otherwise find abhorrent.

The ominous parallel with so-called ``war crimes'' is clear. Soldiers from many countries and in various wars have committed heinous acts of cruelty and slaughter against innocent men, women, and children. Yet they were not particularly evil or sadistic men; they were simply following orders. The research on obedience raises the chilling possibility that anyone of us would do the same thing in the same situation.

The potentially destructive power of religious authority was clearly demonstrated in 1978. In Jonestown, Guyana, more than 900 followers of religious-cult leader Jim Jones obeyed his orders and committed suicide by drinking poison. Christian leaders need to beware of the awesome power of their position, particularly when they claim to speak as the voice of God. \emph{(from Psychology and Christianity by Ronald Philipchalk, p.~214-215)}

\hypertarget{avoiding-destructive-obedience}{%
\subsection*{Avoiding Destructive Obedience}\label{avoiding-destructive-obedience}}
\addcontentsline{toc}{subsection}{Avoiding Destructive Obedience}

How can you protect yourself from participating in destructive obedience? Actually, you may have already taken the first step. Growing evidence suggests that when people learn about research findings such as these, they tend to change their behavior to take this knowledge into account (Sherman, 1980). There are, however, several more active steps you can take to reduce your vulnerability to malevolent authority.

\textbf{\emph{First}}, you can question the legitimacy of the authority. Ask yourself if the person giving the orders really has the right to command you to do what he or she is asking. Special uniforms, insignias, and titles can create an aura of authority even where none exists. For example, Brad Bushman found that 72 \% of pedestrians gave change to a young man when ordered to by a woman in uniform, compared to 48 \% when the woman was dressed in business clothes. Bushman found that the nature of the uniform did not even have to be related to the order being given (Bushman, 1984, 1988).

\textbf{\emph{Second}}, ask yourself if your obedience might cause harm to someone else. Remember that even if you act under legitimate authority you are still responsible for any harm you produce.

\textbf{\emph{Third}}, talk to others about your misgivings. This can not only strengthen your personal resolve, but it may also convince others to join you as well. In one field experiment, 21 out of 22 nurses were prepared to obey an unfamiliar physician whose order violated hospital policy and possibly endangered the patients. The nurses had no chance to consult with anyone else (Hofling et al., 1966). In a second study, however, when the nurses were given the opportunity to contact their supervisors and other nurses, only 2 out of 18 were willing to obey (Rank \& Jacobson, 1977).

Milgram also demonstrated the effect of support for disobedience. In one experiment he employed two additional confederates. They posed as co-teachers who, along with the real subject, gave shocks to the ``learner.'' One confederate refused to continue at 150 volts, the other confederate broke off at 210 volts. With this type of additional support, 90 \% of the subjects refused to go all the way to the maximum shock---total obedience was reduced to 10 \% (Milgram, 1974).

\textbf{\emph{Finally}}, if you decide to defy authority, you might take a lesson from the conformity research and remember the power of calm, reasonable dissent. One of the most famous literary examples of defiance, Herman Melville's Bartleby, the Scrivener, illustrates this ``gentle rebellion:''

\emph{``Bartleby,''} said I, \emph{``Ginger Nut is away; just step round to the Post Office, won't you?} (it was but a three minute walk), \emph{and see if there is anything for me.''}

\emph{``I would prefer not to.''}

\emph{``You will not?''}

\emph{``I prefer not.''}

I staggered to my desk, and sat there in a deep study. My blind inveteracy returned. Was there any other thing in which I could procure myself to be ignominiously repulsed by this lean, penniless wight?---my hired clerk? What added thing is there, perfectly reasonable, that he will be sure to refuse to do?

\emph{``Bartleby''} - No answer

\emph{``Bartleby,''} in a louder tone. - No answer.

\emph{``Bartleby,''} I roared.

Like a very ghost, agreeably to the laws of magical invocation, at the third summons, he appeared at the entrance of his hermitage.

\emph{``Go to the next room, and tell Nippers to come to me.''}

\emph{``I prefer not to,''} he respectfully and slowly said, and mildly disappeared.

Bartleby manages to defy his employer and yet keep his job because his meek reply is so disarming. At times a more aggressive stance may be necessary, but remember, resistance is always possible. Obedience should not be blind. \emph{(from Invitation to Social Psychology by Ronald Philipchalk)}

\hypertarget{learning-activity-7}{%
\subsection*{Learning Activity}\label{learning-activity-7}}
\addcontentsline{toc}{subsection}{Learning Activity}

\hypertarget{read-and-reflect-7}{%
\subsubsection*{Read and Reflect}\label{read-and-reflect-7}}
\addcontentsline{toc}{subsubsection}{Read and Reflect}

This unit, we learned about obedience. We saw that obedience is strongest level of social influence. This will be the focus of this unit's Learning Lab. In order to prepare for discussion this week, take a moment to consider the following resources:

\begin{itemize}
\tightlist
\item
  \href{https://www.units.miamioh.edu/psybersite/cults/cco.shtml}{\textbf{Conformity, Compliance, and Obedience}}\\
\item
  \href{http://blogs.discovermagazine.com/neuroskeptic/2008/12/26/seven-things-you-didnt-know-about-milgram/\#.Xc8k9vZFyP9}{\textbf{Seven Things You Didn't Know About Stanley Milgram}}\\
\item
  \href{https://www.mtholyoke.edu/~apkokot/MilgramBio.htm}{\textbf{Stanley Milgram: His Life and Work}}\\
\item
  \href{https://cpa.ca/docs/File/Ethics/CPA_Code_2017_4thEd.pdf}{\textbf{Canadian Code of Ethics for Psychologists}}\\
\item
  \href{https://www.apa.org/ethics/code/index}{\textbf{Ethical Principles of Psychologists and Code of Conduct}}
\end{itemize}

\emph{As mentioned earlier, it is not necessary to read each resource thoroughly. What is important is to gain comprehension of the importance of using ethical guidelines when doing research on conformity, compliance, and obedience. Start to ponder how necessary is it to involve the use of deception in research. Ask yourself how you would feel if you were actually part of an experiment that was measuring compliance and you were deceived about what the researchers were actually doing? Also, while reading this information have you been able to recognize areas of your life where you are unquestionably following an authority figure? When answering questions like this I would encourage you to not be overly critical of yourself or others, rather, just notice where certain phenomena are present in your life and if you would like to do anything about it.}

\hypertarget{learning-lab-preparation-17}{%
\subsubsection*{Learning Lab Preparation}\label{learning-lab-preparation-17}}
\addcontentsline{toc}{subsubsection}{Learning Lab Preparation}

Prior to your Learning Lab, take some time to think about the following scenario and questions. You will be asked to share your thoughts in this week's Learning Lab:

\emph{When people read about Stanley Milgram's experiments on obedience to au­thor­ity, they are often puzzled at the ethics of this research. You have probably have heard of these studies before and already have some views about the ap­propriate­ness of the procedures. All psychologists follow ethical guidelines in their research established by national governing bodies (e.g., CPA, APA). At the heart of these state­ments about ethics are the responsibilities to: protect the subject's welfare, avoid deception, provide anonymity, and provide therapy or relief from any unin­tended results of the experiment. Each of these general principles applies to the Milgram studies. Do you believe that the studies could be conducted today, given what we know now about ethics and about Milgram's results?}

\begin{itemize}
\tightlist
\item
  \textbf{\emph{Does the severity of the problem studied---in this case destructive obedience---justify more extreme steps to manipulate and deceive experimental subjects?}}\\
\item
  \textbf{\emph{Under what circumstances do you personally think that deceiving subjects in a scientific experiment might be ethically warranted or justified?}}
\end{itemize}

\hypertarget{compliance}{%
\section{Compliance}\label{compliance}}

This fictitious letter, based upon actual appeals I have received, contains several examples of techniques of persuasion. Without criticizing the writer's motives, a social psychologist could identify examples of card stacking, plain folks, foot-in-the-door, and other effective procedures of persuasion that we will discuss below.

Compliance refers to an intermediate level of influence. Here we have the familiar situation where a request is made of us, but it is assumed that we have a choice; we don't have to comply with the persuasive communication. In the letter we see the use of several techniques which have been found to increase the effectiveness of a persuasive communication:

\begin{enumerate}
\def\labelenumi{\arabic{enumi}.}
\tightlist
\item
  \textbf{\emph{Card Stacking}} is the selection of only those examples which support the proponent's argument (Mr.~Smith's example is presented as if it were typical).\\
\item
  \textbf{\emph{The Band Wagon Effect}} gives the impression that many other people are involved, and that everyone is doing it (``thousands of others are standing with you'').\\
\item
  \textbf{\emph{A Testimonial}} is a personal example of the successful application of the proponent's argument (Mr.~Smith provides a concrete example of what might be expected).\\
\item
  \textbf{\emph{Plain Folks}} refers to the association of the person or product with simple ordinary people (``this poor country boy'').\\
\item
  \textbf{\emph{The foot-in-the-door technique}} is the attempt to get you to comply in a small way, knowing that this greatly increases your chance of complying to a larger request (in our example reading the letter, using the ``prayer square,'' or sending a very small donation are small requests which could be followed up with requests for larger gifts).\\
\item
  \textbf{\emph{Name Calling}} is a technique which persuades people to reject something which they may know nothing about (``the forces of evil'').\\
\item
  \textbf{\emph{Transfer}} is the association of a person or product with something that people already feel strongly about (``God,'' ``family,'' ``children'').
\end{enumerate}

These and other techniques have been shown to greatly increase the effectiveness of a persuasive communication. They are often quite apparent in appeals by religious leaders for financial support. Should Christians use these techniques of influence? \emph{(from Psychology and Christianity by Ronald Philipchalk, p.~211, 213)}

\hypertarget{learning-activity-8}{%
\subsection*{Learning Activity}\label{learning-activity-8}}
\addcontentsline{toc}{subsection}{Learning Activity}

\hypertarget{read-and-reflect-8}{%
\subsubsection*{Read and Reflect}\label{read-and-reflect-8}}
\addcontentsline{toc}{subsubsection}{Read and Reflect}

In this Topic we studied compliance and its use as a tool of influence. This will be the focus of this unit's Learning Lab. In order to prepare for discussion this week, take a moment to consider the following resources:

\begin{itemize}
\tightlist
\item
  \href{https://scholarworks.bgsu.edu/cgi/viewcontent.cgi?article=1004\&context=writ}{\textbf{Social Media and Its Stark Influence on Society }}\\
\item
  \href{http://www.csj.org/studyindex/studycult/cultqa3.htm}{\textbf{Cults: Questions and Answers}}
\end{itemize}

\emph{The purpose of the information being shared in this topic and through these resources is to help you recognize who and/or what you are allowing to be an influence in your life? Asking the question, ``Who is teaching me how to be influential and what direction is that influence taking me in?'' can be invaluable in helping you examine how you are spending your time and if you are truly satisfied in the direction of your life. Remember, tools of influence are just that -- tools. Any tool can be used towards benefit or detriment, it just depends on how it is being applied.}

\hypertarget{learning-lab-preparation-18}{%
\subsubsection*{Learning Lab Preparation}\label{learning-lab-preparation-18}}
\addcontentsline{toc}{subsubsection}{Learning Lab Preparation}

Prior to your Learning Lab, take some time to think about the following questions. You will be asked to share your thoughts in this week's Learning Lab:

\begin{itemize}
\tightlist
\item
  \textbf{\emph{Have you ever been taught a sales technique in your job experience? If so, explain\ldots{}}}
\end{itemize}

If you have not been taught a sales technique, you certainly have experienced many as a consumer. Can you give an example? Here is one:

\textbf{\emph{One of the standard ways that door-to-door salespersons will try to get you to buy something is the so-called foot-in-the-door technique: They will ask you to do them a small favor. Having done so, you will have to justify your own behavior to yourself by think­ing nice things about the seller. Hence, you are more likely to comply when they ask you to sign on the dotted line.}}

\hypertarget{assessment-6}{%
\section*{Assessment}\label{assessment-6}}
\addcontentsline{toc}{section}{Assessment}

\begin{assessment}
While there is no ``formal'' assignment that you will be responsible for submitting for Unit 6, you will be expected to participate in discussion during your Learning Lab. Your facilitator will be providing a participation mark based on your contributions. Below is some information to consider prior to attending your Learning Lab:

\hypertarget{learning-lab-1}{%
\subsubsection*{Learning Lab}\label{learning-lab-1}}
\addcontentsline{toc}{subsubsection}{Learning Lab}

Your facilitator will be providing a participation mark based on your contributions for this unit. Below is some information to consider prior to attending your Learning Lab:

\emph{Active participation in group exercises, reflection, and critical discourse is an essential component of this course. You are expected to show respect for all members of the course, both in your speech and actions. Contribute by actively observing and listening, raising thoughtful questions, examining relevant issues, building on others' ideas, analyzing and evaluating the group's thinking, synthesizing key points, and expanding the group's perspectives. Take care not to dominate a conversation, giving space for others to speak. When in small groups help maintain the focus, flow, and quality of conversations, and take the initiative to invite others (particularly those who are quiet) to speak.}

\textbf{Rubric for Participation in Learning Labs}

\begin{longtable}[]{@{}
  >{\raggedright\arraybackslash}p{(\columnwidth - 4\tabcolsep) * \real{0.3333}}
  >{\raggedright\arraybackslash}p{(\columnwidth - 4\tabcolsep) * \real{0.3333}}
  >{\raggedright\arraybackslash}p{(\columnwidth - 4\tabcolsep) * \real{0.3333}}@{}}
\toprule\noalign{}
\begin{minipage}[b]{\linewidth}\raggedright
Emerging (0-64\%)
\end{minipage} & \begin{minipage}[b]{\linewidth}\raggedright
Developing (65-89\%)
\end{minipage} & \begin{minipage}[b]{\linewidth}\raggedright
Mastering (90-100\%)
\end{minipage} \\
\midrule\noalign{}
\endhead
\bottomrule\noalign{}
\endlastfoot
Never to almost never: Demonstrates active listening (as indicated by disengaged body language and no to rare comments that build on others' remarks),Initiates any contributions in class or small groups, Makes insightful or constructive comments, Helps maintain a supportive space for others to speak. & Sometimes to fairly often: Demonstrates active listening (as indicated by somewhat to often engaged body language and comments that build on others' remarks), Initiates a contribution at least once in a class or small group discussion; Makes insightful or constructive comments, Helps maintain a supportive space for others to speak. & Very often to nearly always: Demonstrates active listening (as indicated by fully engaged body language and comments that build on others' remarks), Initiates more than one contribution in a class or small group discussion, Makes insightful or constructive comments, Creates a space for others to speak and takes initiative to include others. \\
\end{longtable}
\end{assessment}

\hypertarget{checking-your-learning-5}{%
\section*{Checking your Learning}\label{checking-your-learning-5}}
\addcontentsline{toc}{section}{Checking your Learning}

\begin{progress}
Before you move on to the next unit, check that you are able to:

\begin{itemize}
\item
  Define the key terminology associated with social influence, social cognition, and on attitudes, behaviour, and effective communication.
\item
  Describe why individuals conform to others' behaviours, how individuals and groups can influence behaviours, and how we form first impressions and how these impressions influence us.
\item
  Apply your knowledge of the bystander effect to ensure that you will be helped if you are in an emergency, of social cognition to help overcome prejudice and discrimination, and of the central route to describe how a message should be designed.
\item
  Analyze whether guards who participate in abuse are inherently bad people, or if their behaviour is the product of social influences and whether people who commit discriminatory acts are necessarily prejudiced.
\item
  Describe how behaviours influence attitudes in terms of cognitive dissonance theory.
\end{itemize}
\end{progress}

\hypertarget{title}{%
\chapter{Title}\label{title}}

\hypertarget{title-1}{%
\chapter{Title}\label{title-1}}

\hypertarget{title-2}{%
\chapter{Title}\label{title-2}}

\hypertarget{title-3}{%
\chapter{Title}\label{title-3}}

\hypertarget{title-4}{%
\chapter{Title}\label{title-4}}

\hypertarget{sample-unit-format}{%
\chapter*{Sample Unit Format}\label{sample-unit-format}}
\addcontentsline{toc}{chapter}{Sample Unit Format}

\hypertarget{overview-6}{%
\section*{Overview}\label{overview-6}}
\addcontentsline{toc}{section}{Overview}

In this first unit, we begin the course by\ldots{}

\hypertarget{topics-6}{%
\section*{Topics}\label{topics-6}}
\addcontentsline{toc}{section}{Topics}

\begin{enumerate}
\def\labelenumi{\arabic{enumi}.}
\tightlist
\item
  \emph{Topic}\\
\item
  \emph{Topic}\\
\item
  \emph{Topic}\\
\item
  \emph{Topic}
\end{enumerate}

\hypertarget{learning-outcomes-6}{%
\section*{Learning Outcomes}\label{learning-outcomes-6}}
\addcontentsline{toc}{section}{Learning Outcomes}

\begin{itemize}
\tightlist
\item
  \emph{Describe\ldots{}}
\item
  \emph{Contrast\ldots{}}
\item
  \emph{Analyze\ldots{}}
\item
  \emph{Determine\ldots{}}
\item
  \emph{Create\ldots{}}
\end{itemize}

\hypertarget{activity-checklist-6}{%
\section*{Activity Checklist}\label{activity-checklist-6}}
\addcontentsline{toc}{section}{Activity Checklist}

Here is a checklist of learning activities you will benefit from in completing this unit. You may find it useful for planning your work.

\begin{reflect}
{Unit 1 Learning Activities }

\begin{itemize}
\tightlist
\item
\item
\end{itemize}
\end{reflect}

\begin{assessment}
{Unit 1 Assessment}

Please check Moodle for details on what you are required to present for this unit.
\end{assessment}

\begin{feedback}
\textbf{Tips for Instructors:}
Learning activities are typically ungraded, and can be optional for students, however they are designed to help students learn the material and prepare for the assignments.
\end{feedback}

\hypertarget{resources-8}{%
\section*{Resources}\label{resources-8}}
\addcontentsline{toc}{section}{Resources}

Here are the resources you will need to complete this unit.

\begin{itemize}
\tightlist
\item
\item
\end{itemize}

\hypertarget{topic-1-title}{%
\section{Topic 1 Title}\label{topic-1-title}}

{[}Add content{]}

\begin{feedback}
\textbf{Tips for Instructors:} Topic content can be 1-2 paragraphs or several pages. Consider using instructional videos, graphics, charts, or other images to convey information and appeal to visual learners.
\end{feedback}

\hypertarget{activity-introductory-readings-video}{%
\subsection*{Activity: Introductory Readings \& Video}\label{activity-introductory-readings-video}}
\addcontentsline{toc}{subsection}{Activity: Introductory Readings \& Video}

\begin{reflect}
📗 Read \ldots{}

\hypertarget{questions-to-consider}{%
\subsubsection*{Questions to Consider}\label{questions-to-consider}}
\addcontentsline{toc}{subsubsection}{Questions to Consider}

After completing the activities above, consider the following questions:

\begin{itemize}
\tightlist
\item
  \ldots{}\\
\item
  \ldots{}
\end{itemize}
\end{reflect}

\emph{Note that the learning activities in this course are ungraded, unless specified. They are designed to help you succeed in your assessments in this course, so you are strongly encouraged to complete them.}

\begin{feedback}
\textbf{Tips for Instructors:}
As you add the details of the chapter/article/website to read, be sure to add the context for the reading (details about author and/or subject) and relate the reading to the unit learning outcomes.

The ``Questions to Consider'' feature gives students an opportunity to reflect on readings and make connections, encouraging higher order thinking. In addition to questions, ask students to create graphic organizers and jot down notes in a Reflective Journal.
\end{feedback}

\hypertarget{learning-activities-10}{%
\subsection*{Learning Activities}\label{learning-activities-10}}
\addcontentsline{toc}{subsection}{Learning Activities}

\hypertarget{questions-to-consider-1}{%
\subsection*{Questions to Consider}\label{questions-to-consider-1}}
\addcontentsline{toc}{subsection}{Questions to Consider}

\hypertarget{topic-2-title}{%
\section{Topic 2 Title}\label{topic-2-title}}

{[}add content{]}

\hypertarget{questions-to-consider-2}{%
\subsection*{Questions to Consider}\label{questions-to-consider-2}}
\addcontentsline{toc}{subsection}{Questions to Consider}

After completing the activities above, consider the following questions:

\begin{itemize}
\tightlist
\item
  \ldots{}\\
\item
  \ldots{}
\end{itemize}

\hypertarget{topic-3-title}{%
\section{Topic 3 Title}\label{topic-3-title}}

{[}add content{]}

\hypertarget{learning-activity-9}{%
\subsection*{Learning Activity}\label{learning-activity-9}}
\addcontentsline{toc}{subsection}{Learning Activity}

\begin{reflect}
does this need content
\end{reflect}

\hypertarget{questions-to-consider-3}{%
\subsubsection*{Questions to Consider}\label{questions-to-consider-3}}
\addcontentsline{toc}{subsubsection}{Questions to Consider}

After completing the activities above, consider the following questions:

\begin{itemize}
\tightlist
\item
  \ldots{}\\
\item
  \ldots{}
\end{itemize}

\hypertarget{unit-1-summary}{%
\section*{Unit 1 Summary}\label{unit-1-summary}}
\addcontentsline{toc}{section}{Unit 1 Summary}

{[}add content{]}

\begin{feedback}
\textbf{Tips for Instructors:}
Remind students of a few key points and how they apply to a greater context. You can mention the assignment or perhaps prepare them for what is in the next unit.
\end{feedback}

\hypertarget{assessment-7}{%
\section*{Assessment}\label{assessment-7}}
\addcontentsline{toc}{section}{Assessment}

\begin{assessment}
{Assessment}

Please check Moodle for details on what you are required to present for this unit.
\end{assessment}

\begin{feedback}
\textbf{Tips for Instructors:}
Use a variety of assessment techniques to gauge students understanding of the course learning outcomes. Assessment types include: essays (include drafts/outlines), quizzes, presentations, group projects, discussions, journals, blogs, e-portfolios, interviews, and media projects (infographics, graphic organizer, video, podcast).
\end{feedback}

\hypertarget{checking-your-learning-6}{%
\section*{Checking your Learning}\label{checking-your-learning-6}}
\addcontentsline{toc}{section}{Checking your Learning}

\begin{progress}
Now that you have completed the learning activities and assignments for this unit, check the unit learning outcomes below to see if you are able to do the following:

\begin{itemize}
\tightlist
\item
  \emph{Describe\ldots{}}\\
\item
  \emph{Contrast\ldots{}}\\
\item
  \emph{Analyze\ldots{}}\\
\item
  \emph{Determine\ldots{}}\\
\item
  \emph{Create\ldots{}}
\end{itemize}

Feel free to review topics more in depth or continue on to the next unit.
\end{progress}

\hypertarget{course-credits}{%
\chapter*{Course Credits}\label{course-credits}}
\addcontentsline{toc}{chapter}{Course Credits}

\hypertarget{course-contributors}{%
\section*{Course Contributors}\label{course-contributors}}
\addcontentsline{toc}{section}{Course Contributors}

\hypertarget{curriculum-developer}{%
\subsection*{Curriculum Developer}\label{curriculum-developer}}
\addcontentsline{toc}{subsection}{Curriculum Developer}

\begin{center}\rule{0.5\linewidth}{0.5pt}\end{center}

\hypertarget{course-instructors}{%
\subsection*{Course Instructors}\label{course-instructors}}
\addcontentsline{toc}{subsection}{Course Instructors}

\hypertarget{copyright-credits}{%
\section*{Copyright \& Credits}\label{copyright-credits}}
\addcontentsline{toc}{section}{Copyright \& Credits}

\textbf{Copyright © 2023 Trinity Western University. All rights reserved.}

The content of this course material is the property of Trinity Western University (TWU) and is protected by copyright law worldwide. This material may be used by students enrolled at TWU for personal study purposes only.

TWU seeks to ensure that any course content that is owned by others has been appropriately cleared for use in this course. Anyone wishing to make additional use of such third party material must obtain clearance from the copyright holder.

\hypertarget{course-development-team}{%
\subsection{Course Development Team}\label{course-development-team}}

Course Writer:
Instructional Designer:
Production Team:
Department Chair:
Dean:

Trinity Western University
22500 University Drive
Langley, BC, Canada \textbar{} V2Y 1Y1

\hypertarget{references}{%
\chapter*{References}\label{references}}
\addcontentsline{toc}{chapter}{References}

The following are key references used in this course. \textbf{\emph{Check with your course syllabus for required readings.}}

  \bibliography{book.bib}

\end{document}
